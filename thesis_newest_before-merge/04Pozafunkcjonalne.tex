\chapter{Wymagania pozafunkcjonalne}
\label{Chapter4}

\section{Wstęp}
\label{Chapter41}

W niniejszym rozdziale zostaną zaprezentowane i krótko opisane charakterystyki oraz wymagania pozafunkcjonalne obowiązujące dla systemu. Dodatkowo, podjęta zostaje tu próba weryfikacji, które wymagania udało się spełnić i jakie są perspektywy dalszego rozwoju projektu.

\section{Charakterystyki oprogramowania}
\label{Chapter42}

Projektując system, pod uwagę brane były następujące charakterystyki:

\begin{itemize}
\item{Analizowalność}
\item{Autentyczność}
\item{Bezpieczeństwo (wolność od ryzyka)}
\item{Charakterystyka czasowa}
\item{Dostępność personalna}
\item{Dostępność techniczna}
\item{Estetyka interfejsu Użytkownika}
\item{Funkcjonalna poprawność}
\item{Identyfikowalność}
\item{Integralność}
\item{Interoperacyjność}
\item{Kompletność kontekstowa}
\item{Łatwość adaptacji}
\item{Łatwość instalacji}
\item{Łatwość nauczenia się}
\item{Łatwość testowania}
\item{Łatwość zamiany}
\item{Łatwość zmiany}
\item{Niezaprzeczalność}
\item{Ochrona użytkownika przed błędami}
\item{Odporność na wady}
\item{Odtwarzalność}
\item{Poufność}
\item{Współistnienie}
\item{Zużycie zasobów}
\end{itemize}

Wybór tych, a nie innych charakterystyk został podyktowany decyzją zespołu zarządzającego, opartą na wiedzy pozyskanej w trakcie jednych z zajęć akademickich w ramach 1. roku studiów II stopnia. Charakterystyki te rozpatrywane były w fazach projektowania i implementacji z zastosowaniem różnych priorytetów. Ciężko w tym miejscu wyznaczyć, w kontekście implementacji, najważniejsze bądź najtrudniejsze do osiągnięcia wymagania z nimi związane, w szczególności w relacji wzajemnej między nimi. Tym niemniej, można dokonać pewnych obserwacji w kontekście projektowania, m.in. rozpatrując ich priorytetyzację przez zespół zarządzający. \\

Za bardzo ważną uznano interoperacyjność, czyli współpracę systemu iQuest z pozostałymi systemami Uczelnianymi. Kwestia ta jest niezbędna dla prawidłowego działania mechanizmów autoryzacji przyszłych ankieterów i respondentów, mających operować na systemie. \\

Ze względu na wagę potrzeb zaspakajanych przez system -- m.in. spełnienie wymogów ministerialnych związanych z monitorowaniem rozwoju absolwentów -- odporność na wady oraz odtwarzalność również zostały potraktowane z wysokim priorytetem. \\

Zważywszy na współczesne zastosowania sieci Internet, nie jest zaskakującym nie mniejsza istotność estetyki interfejsu użytkownika. Powstające dziś systemy muszą przyciągnąć potencjalnych użytkowników, umożliwiając im wygodne manewrowanie wśród ich funkcjonalności. Intuicyjność rozwiązania powinna sprawiać, że użytkownik końcowy nie będzie potrzebował poświęcać wiele czasu na naukę jego obsługi, nawet, jeśli zrezygnuje z zapoznania się z -- również przygotowaną w sposób przejrzysty -- instrukcją. \\

Zważywszy, że projekt ten jest istotny dla Uczelni, dołożono wszelkich starań, aby spełnić wszystkie wymagania pozafunkcjonalne -- nie tylko te opisane wyżej jako ważne. Zadanie to zostało w sporej części wykonane dzięki zastosowaniu sprawdzonego i znanego narzędzia w roli podstawy dla całego systemu. Mowa o platformie Moodle\footnote{Więcej informacji w rozdziale~\ref{Chapter6}, poświęconym implementacji i zastosowanym przy niej technologiom.}. Użycie jej pozwoliło m.in. na zapewnienie pełnej modułowości systemowi, dzięki czemu wprowadzanie jakichkolwiek zmian w dowolnej jego części jest uproszczone i nie wpływa na resztę kodu. Spełniona została także adaptatywność i przenośność systemu, oparta na uniwersalności języka PHP, jak też szerokiej gamie obsługiwanych przez Moodle systemów baz danych. \\

Zalety wykorzystania tak powszechnej technologii są jednak znacznie szersze. Moodle posiada swój własny standard kodowania, którego przy realizowaniu projektu starano się w pełni przestrzegać. To sprawia, że w połączeniu z wytycznymi DRO (\definicja{Dział Rozwoju Oprogramowania Politechniki Poznańskiej}), dokumentacją Moodle i dokumentacją systemu iQuest, nakład pracy wymagany do wdrożenia się, celem rozwoju projektu, został zminimalizowany. \\

Warto zaznaczyć, że w trakcie prac nad projektem, bardzo silny nacisk stawiano na prawidłowe zarządzanie uprawnieniami użytkowników. Są one weryfikowane przy każdym działaniu podejmowanym przez użytkowników. Zależą one od przydzielonej im w systemie roli. Szczegóły tej kwestii swobodnie definiować może administracja. Ważnym jest, że osoby o roli ankietera mogą modyfikować i analizować jedynie utworzone przez siebie, lub udostępnione im bezpośrednio, badania. Podobnie działają uprawnienia przydzielone do roli respondenta -- pozwalając dokładnie jeden raz odpowiedzieć na pytania w ankietach skierowanych do grup docelowych, do których należy użytkownik. \\

Charakterystyką, której od samego początku nie traktowano priorytetowo, była szybkość działania systemu. W ogólnym sensie, wyświetlanie danych realizowane jest w relatywnie krótkim czasie, zależy to jednak od tego, gdzie są one przechowywane, czy ich pobranie wymaga połączenia z systemem zewnętrznym oraz czy istnieją one już w bazie danych systemu, czy dopiero należy je tam zamieścić. Tym niemniej, wszelkie związane z tą kwestią wymagania zostały spełnione. \\

Instalacja systemu realizowana jest trójstopniowo. Pierwszym krokiem jest umieszczenie na serwerze docelowym plików platformy Moodle i ich instalacja oraz wstępna konfiguracja. W tak przygotowanym środowisku, umieszczane są następnie pliki systemu iQuest, które -- dzięki wbudowanym mechanizmom platformy Moodle -- administrator może w bardzo prosty sposób zainstalować i aktywować. Aktualizacja systemu przebiega bardzo podobnie, i polega jedynie na przeprowadzeniu standardowego dla Moodle procesu aktualizacji, poprzedzonego zmianą odpowiednich plików na serwerze, w ramach którego działa system. W trakcie testów na urządzeniu przenośnym typu netbook\footnote{Asus Automobili Lamborghini VX6, CPU: Intel Atom D525 1,7GHz, RAM: 4GB, HDD: 320GB 5400 RPM, Środowisko: korzystający z \definicja{WUBI} system \definicja{Ubuntu 12.04 LTS}.} pełniącym rolę serwera HTTP, serwera bazy danych oraz klienta systemu, cały proces instalacji nie przekraczał 40 minut. Zaobserwowanym głównym czynnikiem ograniczającym wydajność procesu była szybkość środowiska wirtualnego oraz zastosowanego dysku twardego.

Nierozpatrzone wyżej charakterystyki zostały uznane przez zespół zarządzający za mniej priorytetowe. Autorzy niniejszej pracy dyplomowej nie uznali za konieczne zagłębiania się w tę kwestię. Możliwe jednak, że wnioski zawarte w rozdziale \ref{Chapter8}, odpowiadają choć częściowo na to zagadnienie.

\section{Wymagania pozafunkcjonalne i ich weryfikacja}
\label{Chapter43}

\subsection{Wymagania pozafunkcjonalne}
\label{Chapter431}

Wymagania pozafunkcjonalne przypisane do wyżej opisanych charakterystyk oprogramowania przedstawione zostały w tabeli \ref{tab:reqs}. Dodatkowo, określono za pomocą etykiety \textbf{ważność (W)} oraz \textbf{trudność implementacji (TI)} dla każdego z wymagań. Wartości w ramach tego oznaczenia wyrażone są w następującej skali:

\begin{itemize}
\item{H (ang.~\definicja{high}) -- wysoka}
\item{M (ang.~\definicja{medium}) -- średnia}
\item{L (ang.~\definicja{low}) -- niska}
\end{itemize}

Analizy, zakończonej etykietowaniem, dokonał zespół zarządzający, rozpatrując to wedle wytycznych przedstawionych w trakcie zajęć akademickich na 1. roku studiów II stopnia.

\newpage
\begin{center}
\begin{longtable}{ | p{4cm} | p{9cm} | c | c | }
\hline
\textbf{Charakterystyka} & \textbf{Wymaganie} & \textbf{W} & \textbf{TI} \\ \hline
%
Analizowalność & Komentarze w kodzie źródłowym powinny być w języku angielskim. & M & L \\ \hline
Analizowalność & Kod źródłowy sytemu powinien być utworzony zgodnie ze standardami Moodle. & M & L \\ \hline
Analizowalność & System powinien zawierać testy jednostkowe. & M & M \\ \hline
Analizowalność & System powinien rejestrować stack trace i rodzaj błędu (fatal, warning). & H & L \\ \hline
Analizowalność & Wraz z kodem źródłowym systemu należy dostarczyć dokumentację. & H & M \\ \hline
Analizowalność & System powinien logować niewłaściwe wywołania metod. & H & L \\ \hline
%
Autentyczność & Gdy student staje się absolwentem, należy umożliwić mu dalsze logowanie się do systemu bez użycia eKonta. & H & M \\ \hline
%
Bezpieczeństwo (wolność od ryzyka) & Dane (opinie respondentów) przechowywane w systemie muszą być uzyskiwane poprzez wbudowane mechanizmy ankietowania. & H & L \\ \hline
Bezpieczeństwo (wolność od ryzyka) & Projekt systemu należy poddać analizie z wykorzystaniem metody ATAM. & M & M \\ \hline
%
Charakterystyka czasowa & Wyświetlenie ankiety powinno trwać nie dłużej niż 4 sekundy. & H & L \\ \hline
Charakterystyka czasowa & Generowanie raportów powinno odbywać się ze średnio nie dłużej niż 1 godzinę. & M & M \\ \hline
Charakterystyka czasowa & Należy zdefiniować klasy operacji, w zależności od czasu ich trwania. Klasy:
\begin{itemize}
\item{bez komunikatu potwierdzającego wykonanie}
\item{z potwierdzeniem wykonania}
\item{wykonywane na serwerze w tle}
\end{itemize} & M & M \\ \hline
%
Dostępność personalna & Przewidzieć, na poziomie architektury, możliwość rozbudowy np. o interfejs dla osób niedowidzących. & L & M \\ \hline
%
Dostępność techniczna & System może mieć przerwę serwisową, lecz musi wówczas prezentować specjalny ekran informujący o czasie jej trwania. & H & L \\ \hline
%
Estetyka Interfejsu Użytkownika & Środowisko ma być przyjazne i czytelne dla użytkownika końcowego. & H & M \\ \hline
%
Funkcjonalna poprawność & Wszystkie wartości mają być prezentowane z dokładnością do 2~miejsc po przecinku. & H & L \\ \hline
%
Identyfikowalność & System ma umożliwiać identyfikowanie podmiotów (osobno: administratorów, ankieterów, respondentów), podejmujących konkretne działania: tworzenie ankiet, odpowiadanie w 
ankietach, itp.. & H & M \\ \hline
%
Integralność & Baza danych powinna być chroniona przed nieuprawnionym dostępem [modyfikacją\slash usunięciem] w następujący sposób: logowanie za pomocą loginu i hasła. & H & L \\ \hline
Integralność & System powinien być odporny na następujące próby nielegalnego dostępu: nieuprawniony dostęp fizyczny do serwera. & M & L \\ \hline
Integralność & Należy chronić\slash szyfrować dane przesyłane z i do systemu. & M & L \\ \hline
%
Interoperacyjność & System ma wymieniać potrzebne dane z systemami uczelnianymi: eKonto, eDziekanat, ePoczta. Dane mają być aktualne. & H & M \\ \hline
Interoperacyjność & System ma pobierać dane z systemu eDziekanat w następujący sposób: SOAP. & H & M \\ \hline
Interoperacyjność & System ma przesyłać dane o wiadomościach do wysłania do systemu ePoczta w następujący sposób: SOAP. & H & M \\ \hline
Interoperacyjność & System ma pobierać dane do autoryzacji z systemu eKonto w następujący sposób: SOAP. & H & M \\ \hline
Interoperacyjność & System ma przesyłać wyniki ankiet do systemu raportowania. & H & M \\ \hline
%
Kompletność kontekstowa & System ma działać w przeglądarkach: IE 7.0+, Firefox 15, Opera 12. & H & M \\ \hline
Kompletność kontekstowa & Należy przygotować raport jak system zachowuje się na platformach mobilnych. & L & L \\ \hline
%
Łatwość adaptacji & Należy utworzyć raport z łatwości adaptacji oraz gdzie znajdują się adaptery. & M & L \\ \hline
%
Łatwość instalacji & System musi umożliwiać łatwą aktualizacje, przy założeniu, że wersja platformy Moodle pozostaje bez zmian. & H & H \\ \hline
%
Łatwość nauczenia się & Interfejs użytkownika (dla ankietowanych) powinien być całkowicie intuicyjny. & H & M \\ \hline
Łatwość nauczenia się & Interfejs użytkownika (dla ankietera) może wymagać nieznacznego doszkolenia obsługujących. & M & L \\ \hline
%
Łatwość testowania & ,,Atrapy'' (ang.~\definicja{mock}) systemów zewnętrznych (m.in. eKonto, ePoczta). & M & M \\ \hline
Łatwość zamiany & Należy umożliwić przełączanie systemu między trybem testowym i produkcyjnym. & M & L \\ \hline
%
Łatwość zamiany & System powinien umożliwiać wczytanie wszystkich danych z poprzedniej wersji. & H & M \\ \hline
Łatwość zamiany & Procedura wymiany oprogramowania powinna trwać nie dłużej niż 2 dni i odbywać się w następujący sposób: zgodność z instrukcją. & M & L \\ \hline
Łatwość zamiany & Podczas projektowania systemu, należy brać pod uwagę możliwość wprowadzenia wielojęzyczności interfejsu. & M & M \\ \hline
%
Łatwość zmiany & System powinien być przygotowany na wprowadzenie następujących zmian: nowe typy raportów, nowe typy pytań, modyfikacje interfejsów systemów zewnętrznych. & M & M \\ \hline
%
Niezaprzeczalność & System musi posiadać logi (zalogowanie w systemie, stworzenie\slash edycja\slash usunięcie\slash wysłanie\slash wypełnienie ankiety), aby móc udokumentować skąd pochodzą dane. & H & L \\ \hline
%
Ochrona użytkownika przed błędami & Dodatkowe potwierdzenie chęci wykonania operacji nieodwracalnych (nawet dla administratora), lub możliwość przywrócenia usuniętych danych przez jakiś czas. & H & M \\ \hline
Ochrona użytkownika przed błędami & Dla dużych ankiet, zatwierdzenie odesłania jej przez ankietowanego. & M & L \\ \hline
Ochrona użytkownika przed błędami & Potwierdzenie przed rozesłaniem ankiet. & M & L \\ \hline
Ochrona użytkownika przed błędami & Lista operacji wykonywanych w tle. & M & L \\ \hline
%
Odporność na wady & Gdy nastąpi awaria innych systemów np. eKonto, należy poinformować użytkownika o błędzie i uniemożliwić mu dalsze działanie w systemie. & H & L \\ \hline
%
Odtwarzalność & Odtwarzanie całego systemu w czasie nieprzekraczającym 3h. & M & L \\ \hline
Odtwarzalność & Kopia zapasowa bazy danych wykonywana z częstotliwością raz na dobę. & H & L \\ \hline
Odtwarzalność & Dostępność instrukcji odtwarzania. & H & L \\ \hline
Odtwarzalność & Wykonywanie operacji w transakcjach tam, gdzie to możliwe. & H & L \\ \hline
%
Poufność & Uwierzytelnianie i autoryzacja dla każdej operacji. & M & L \\ \hline
%
Współistnienie & Jedynym kryterium działania systemu ankiet (bez raportów) jest działająca przeglądarka. & H & L \\ \hline
Współistnienie & System generowania raportów może być osobną aplikacją, działającą na specjalnych: sprzęcie i oprogramowaniu. & L & L \\ \hline
Współistnienie & Wszystkie potrzebne dane z zewnętrznych systemów należy przechowywać lokalnie i synchronizować. & M & H \\ \hline
Współistnienie & Należy kolejkować zadania (np.~e-maile) w systemie, na czas braku możliwości ich wysłania (np.~awaria ePoczty). & H & M \\ \hline
%
Zużycie zasobów & System nie powinien wykorzystywać więcej serwerów HTTP niż jeden. & M & L \\ \hline
Zużycie zasobów & System powinien działać poprawnie przy nawet 10.000 użytkowników zalogowanych jednocześnie. & H & M \\ \hline
Zużycie zasobów & Zadania, które nie muszą zostać wykonane natychmiastowo, powinny być kolejkowane przez system. & M & M \\ \hline
Zużycie zasobów & Możliwość pracy przez ankietera\slash respondenta na średniej klasy laptopie (CPU 2GHz, RAM 4GB). & H & L \\ \hline
%
\caption{Wymagania pozafunkcjonalne}\label{tab:reqs}
\end{longtable}
\end{center}

\subsection{Weryfikacja wymagań pozafunkcjonalnych}
\label{Chapter432}

%%%%%%%%%%%%%%%%%%%%%%%%%%%%%%%%%%%%%%%