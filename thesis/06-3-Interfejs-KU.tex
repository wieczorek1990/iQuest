%W ramach systemu iQuest, elementy interfejsu ułożone są według następującego schematu:
%\begin{description}
%\item Na ekranie wyświetlana jest lista sekcji.
%\item Wewnątrz każdej sekcji znajduje się lista badań. Pod nią prezentowana jest lista dostępnych materiałów.
%\end{description}

\section{Interfejs}
\label{Chapter65}

W trakcie projektowania graficznego interfejsu użytkownika, głównym problemem okazał się wybór odpowiedniego narzędzia. Celem jaki postawiono, była maksymalna zgodność tworzonych elementów z różnymi wersjami \textit{Moodle} -- zarówno wcześniejszymi, jak i późniejszymi. Zdecydowano, aby starać się korzystać z gotowych interfejsów programowania aplikacji \textit{(API)} dostarczonych przez \textit{Moodle}, tj. \textit{Page API}, \textit{Form API}, oraz \textit{Access API}. Wszystkie interfejsy zostały napisane przy użyciu języka \textit{PHP} -- są wykonywane po stronie serwera. Konieczne okazało się też wykonanie niektórych skryptów po stronie klienta. Dlatego w projekcie wykorzystano również język skryptowy \textit{Java Script}.

\subsection{Bezpieczeństwo}

Potencjalnie, z systemu może korzystać wielu użytkowników, zarówno studentów jak i pracowników. Każda z tych osób może być uprawniona do wykonywania innych czynności w systemie, niezależnie od siebie. Wiąże się to z jednej strony z uwierzytelnianiem użytkowników, zaś z drugiej -- ich autoryzacją. Odbywa się to za pomocą wbudowanego w \textit{Moodle} mechanizmu ról. \\

Zapewnienie bezpieczeństwa w systemie wymagane jest, aby użytkownik przez pomyłkę nie wykonał czynności, do których nie został uprawniony. Sprowadza się to do ograniczenia mu dostępnych opcji do tych, które może używać. Dla przykładu, użytkownik może zobaczyć i użyć jedynie tych odnośników, które prowadzą do zasobów, do których posiada uprawnienia. \\

Zabezpieczenie nie ogranicza się jednak jedynie do ograniczenia dostępności opcji. Należy bowiem założyć, że użytkownik może przypadkowo lub z intencją próbować pozyskać zasoby bez autoryzacji. Z tego względu zapewnienie bezpieczeństwa wiąże się także z odmową dostępu do zabronionych zasobów. W tym celu sprawdzano dane, które przychodzą do serwera, m.in. tworząc zmienne wiązane w zapytaniach \textit{SQL}, co pozwala na obronę przed atakami typu \textit{SQL-injection}. Ostatecznie, bezpośrednio przed wyświetleniem treści użytkownikowi, sprawdzane jest, czy jest on uprawniony do ich czytania. Dzięki temu, niepowołany użytkownik nie uzyska dostępu do niedozwolonych treści nawet przez wykorzystanie adresu prowadzącego bezpośrednio do żądanego zasobu.