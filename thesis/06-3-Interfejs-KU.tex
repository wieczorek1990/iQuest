\section{Interfejs}
\label{Chapter65}

Jedną z części pracy było zaprojektowanie graficznego interfejsu użytkownika. Głównym problemem jaki się pojawił, był wybór odpowiedniego narzędzia. Celem jaki postawiono, była maksymalna zgodność projektowanych elementów z różnymi wersjami \emph{Moodle} -- zarówno wcześniejszymi, jak i późniejszymi. Zdecydowano, aby starać się korzystać z gotowych interfejsów programowania aplikacji \emph{(API)} dostarczonych przez \emph{Moodle}, tj. \emph{Page API}, \emph{Form API}, oraz \emph{Access API}. Wszystkie interfejsy są napisane przy użyciu języka PHP -- są wykonywane po stronie serwera. Konieczne okazało się też wykonanie niektórych skryptów po stronie klienta. Dlatego w projekcie wykorzystano również język skryptowy \emph{Java Script}.