\section{Interfejs}
\label{Chapter65}

Jedną z części pracy było zaprojektowanie graficznego interfejsu użytkownika. Głównym problemem jaki się pojawił, był wybór odpowiedniego narzędzia. Celem jaki postawiono, była maksymalna zgodność projektowanych elementów z różnymi wersjami \emph{Moodle} -- zarówno wcześniejszymi, jak i późniejszymi. Zdecydowano, aby starać się korzystać z gotowych interfejsów programowania aplikacji \emph{(API)} dostarczonych przez \emph{Moodle}, tj. \emph{Page API}, \emph{Form API}, oraz \emph{Access API}. Wszystkie interfejsy są napisane przy użyciu języka PHP -- są wykonywane po stronie serwera. Konieczne okazało się też wykonanie niektórych skryptów po stronie klienta. Dlatego w projekcie wykorzystano również język skryptowy \emph{Java Script}.

\subsection{Bezpieczeństwo}

Potencjalnie z systemu może korzystać wielu użytkowników, zarówno studentów jak i pracowników. Każda z tych osób może być uprawniona do wykonywania innych czynności w systemie. Wiąże się to z jednej strony z uwierzytelnianiem użytkowników, z drugiej, ich autoryzacją. Odbywało się to za pomocą wbudowanego w \emph{Moodle} mechanizmu ról.

Zapewnienie bezpieczeństwa w systemie polega po pierwsze na tym, aby użytkownik przez pomyłkę nie wykonał czynności, do których nie został uprawniony. Sprowadzało się to do tego, aby ograniczyć użytkownikowi wyświetlane opcji tylko do tych, które może używać. Np. tylko tych odnośników, które prowadzą do dozwolonych dla użytkownika zasobów.

Po drugie, zapewnienie bezpieczeństwa wiąże się z odmową dostępu do zabronionych zasobów użytkownikom, których intencją było dostanie się do nich. Dlatego sprawdzano dane, które przychodzą do serwera. Np. utworzeno zmienne wiązane w zapytaniu \emph{SQL} w celu obrony przed atakiem \emph{SQL injection}. Co więcej przed wyświetleniem treści użytkownikowi sprawdzano, czy użytkownik może je przeczytać. Dzięki temu niepowołany użytkownik nie przeczyta niedozwolonych treści wpisując bezpośrednio adres do żądanej podstrony w pasku adresu przeglądarki.