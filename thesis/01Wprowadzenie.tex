\chapter{Wprowadzenie}
\label{Chapter1}

\section{Opis problemu i koncepcja jego rozwiązania}
\label{Chapter11}

\subsection{Problem}
\label{Chapter111}

W~kontekście każdej organizacji, zrzeszającej większą grupę ludzi, takiej jak~zakład pracy, czy~jednostka szkolnictwa wyższego, jaką~jest Politechnika Poznańska, niezbędne jest systematyczne badanie efektywności programów nauczania, sposobu kształcenia oraz celowości inwestycji. Przykładowo, brak wiedzy o~potrzebach studentów i~wykładowców dotyczących wyposażenia laboratoriów, sal wykładowych, programów komputerowych, może prowadzić do zbędnych zakupów, a~tym~samym do~ponoszenia niepotrzebnych kosztów. Najprostszym i~najbardziej popularnym narzędziem prowadzenia badań jest ankieta\footnote{\cite{Wiki:BA}\cite{IP:Awbi}}. Pozwala ona na~nieinwazyjne, anonimowe i~masowe pozyskiwanie opinii respondentów na~zadany temat, a~dzięki zastosowaniu odpowiednich metod statystycznych, także na~ocenę zdania całej populacji. \\

Biorąc pod uwagę ostatnie lata (2008-2012)\footnote{Mowa o~okresie, który jest znany autorom niniejszej pracy dyplomowej.}, w~zgodzie ze~statutem\footnote{Zapis dotyczący dokonywania oceny nauczyciela akademickiego z~uwzględnieniem oceny studentów\cite{AP:SPP11}.}, Politechnika Poznańska wykorzystywała szerokie spektrum narzędzi do~pozyskiwania wiedzy o~efektach podejmowanych działań. Jednak z~uwagi na~brak odpowiedniego systemu zapewniającego wymierność wyników (np.~uniemożliwienie wielokrotnego udziału pojedynczej osoby w~ankiecie, zablokowanie możliwości wypełnienia ankiety przez studenta innego wydziału\slash kierunku,~itp.), uzyskane informacje mogły przyczyniać~się do~błędnej interpretacji sytuacji. Dla~zobrazowania dotychczasowego systemu badania, posłuży przykład ankietowania studentów Wydziału Informatyki\footnote{Do 2010 roku -- Wydziale Informatyki i~Zarządzania.}. Student był~poddawany badaniom ankietowym w~formie elektronicznej na~poziomach uczelni i~wydziału, formie papierowej przeprowadzanej przez Samorząd Studentów na~potrzeby uczelni, oraz~dodatkowym ankietom w~ramach niektórych zajęć dydaktycznych. \\

Wyniki ankiet realizowanych w~ramach wyżej wymienionych systemów prowadzenia badań nie~były ze~sobą porównywane, co~sprawiało, że~nie~mogły dać pełnego obrazu panującej sytuacji. Samo wdrożenie badań nie~umożliwiało wyciągnięcia konstruktywnych wniosków. Brak odpowiednich mechanizmów analizy ich~wyników, generował zagrożenie wystąpienia spadku jakości usług świadczonych przez uczelnię, a~co za~tym idzie, także jej prestiżu.

\subsection{Proponowane rozwiązanie}
\label{Chapter112}

Punktem wyjścia jest stworzenie jednego, globalnego, prostego w~obsłudze systemu o~nazwie \textit{iQuest}, służącego do~prowadzenia badań ankietowych dla~różnych grup docelowych, obejmujących zarówno aktualnych studentów oraz~absolwentów Politechniki Poznańskiej. Będzie on~współpracować z uczelnianym systemem raportowania, umożliwiającym wygodne pozyskiwanie informacji na~zadany temat. \\

Dla~zapewnienia bezproblemowego wdrożenia systemu oraz~jego dostępności, zostanie on~wykonany za~pomocą technologii internetowych. Umożliwi~to jego obsługę z~użyciem dowolnej popularnej przeglądarki internetowej dostępnej na~rynku. \\

Ważnym elementem proponowanego rozwiązania jest objęcie badaniami również absolwentów uczelni. W~zamian za~wzięcie udziału w badaniach, uzyskają oni dostęp do~unikalnych materiałów dydaktyczno-naukowych. \\

Przeprowadzane badania będą anonimowe. Jednakże, w~celu zapewnienia prawidłowości i~miarodajności wyników, niezbędne jest wprowadzenie mechanizmów uwierzytelniania użytkowników systemu. Ich~autoryzacja będzie obejmować jedynie kwestię dostępu do badań i~materiałów -- dla ankietera, wyniki będą pozbawione danych identyfikujących poszczególnych respondentów.

\section{Ograniczenia i zagrożenia dla projektu}
\label{Chapter12}

Analizie podano jedynie ograniczenia i~zagrożenia związane ze stroną implementacyjną projektu, z~uwzględnieniem następujących definicji:

\begin{description}
\item[Ograniczenie] -- granica, której przekroczenie jest niemożliwe lub zakazane. Obejmuje np.~ograniczenia technologiczne oraz~narzucone z~góry przez osoby o~kompetencjach decyzyjnych w~kwestiach związanych z~projektem.
\item[Zagrożenie] -- sytuacja lub~fakt mogący negatywnie wpłynąć na~realizację projektu.
\end{description}

\subsection{Ograniczenia}
\label{Chapter121}

System \textit{iQuest} realizowany jest dla~Wydziału Informatyki. W~związku z~tym musi~on spełniać szerokie spektrum wymagań ze~strony takich organów, jak Dział Rozwoju Oprogramowania (DRO), Biuro Analiz i~Rozwoju Usług (BAiRU), czy Dział Obsługi i~Eksploatacji (DOiE). Jednym z~warunków jest~konieczność stosowania rozwiązań ,,otwartych'' (ang. \definicja{open}) oraz ,,wolnych'' (ang. \definicja{free}), co~znacząco zawęża możliwe do~zastosowania technologie i~zasoby. \\

Za poboczne ograniczenie, ale~mające istotny wpływ na~przebieg prac, należy uznać z~góry narzucony harmonogram, oraz restrykcje co~do wielkości zespołu programistów.

\subsection{Zagrożenia}
\label{Chapter122}

Mimo braku wcześniejszego doświadczenia zespołu przy realizacji tego rodzaju projektów, zespół dynamicznie przystosowywał~się do~sytuacji, wykazując umiejętność szybkiego uczenia~się. W~dalszym ciągu zagrożeniem pozostaje fakt, że~wykonawcami projektu są~studenci uczelni technicznej, a~nie wykwalifikowani specjaliści w~dziedzinie technologii internetowych, posiadający wieloletnią praktykę w~realizacji podobnych zadań. \\

Poważną przeszkodą jest niekompletność dokumentacji niektórych systemów i~technologii niezbędnych w~projekcie\footnote{Rozwinięcie tej~kwestii znajduje~się w~rozdziałach \ref{Chapter6}. oraz \ref{Chapter8}.} oraz~niespodziewane zmiany w~specyfikacji projektu i~wymaganiach klienta, występujące w~trakcie jego realizacji.

\subsection{Wpływ ograniczeń i zagrożeń na projekt}
\label{Chapter123}

Narzucona decyzja o~zastosowaniu konkretnych typów technologii (rozwiązania ,,wolne i~otwarte''), dokumentowanych nie~przez wyspecjalizowanych fachowców, lecz tzw.~,,społęczność'' (ang.~\definicja{community}), wymusiła konieczność oparcia wielu działań na~metodzie ,,prób i~błędów'', co~znacząco wpłynęło na~czas realizacji zadań poszczególnych członków zespołu projektowego. \\

Wspomniana w~poprzedniej sekcji (\ref{Chapter122}. -- Zagrożenia) kwestia zmieniających~się wymagań i~specyfikacji sprawiła jednak najwięcej problemów. Niejednokrotnie zdarzały~się sytuacje, kiedy prace projektowe były już w~zaawansowanym stadium, a~nowe wytyczne powodowały konieczność rozpoczynania pracy od~podstaw. Było~to główną przyczyną powstawania opóźnień w~realizacji projektu.

\section{Cele projektu}
\label{Chapter13}

Celem projektu \textit{iQuest} jest zbudowanie jasnego, prostego, przystępnego systemu umożliwiającego prowadzenie badań wśród studentów i~absolwentów uczelni. System ten powinien:
\begin{itemize}
\item{zapewnić spełnienie przez Uczelnię zapisów Ustawy ,,Prawo o Szkolnictwie Wyższym'' dotyczących monitorowania rozwoju absolwentów Uczelni\cite{AP:PoSW05}}
\item{ujednolicić uczelniany system pozyskiwania informacji}
\item{oferować dużą elastyczność:
\begin{itemize}
\item{przy definiowaniu różnorodnych ankiet}
\item{przy tworzeniu i~hierarchizacji grup respondentów}
\item{przy zachęcaniu do~uczestnictwa w~niej przez~potencjalnych Respondentów}
\end{itemize}}
\item{integrować~się z~uczelnianym systemem raportowania}
\item{odciążyć pracowników uczelni oraz~Samorząd Studentów z~obowiązków związanych z~przeprowadzaniem konwencjonalnych (,,papierowych'') ankiet}
\end{itemize}

\section{Osoby realizujące projekt}
\label{Chapter14}

Projekt realizowany jest w~ramach Software Development Studio (SDS), służącego realizacji projektów informatycznych dla~Wydziału Informatyki Politechniki Poznańskiej. W~rolę zespołu zarządzającego wcielają~się studenci studiów magisterskich (specjalizacja \definicja{Software Engineering}). Zespół programistów tworzą studenci ostatniego semestru studiów inżynierskich, dla~których udział w~SDS jest~ściśle związany z~pracą dyplomową inżynierską. Wkład poszczególnych osób w~projekt znajduje~się w~dodatku \ref{Chapter101}. \\

Poniżej przedstawiono wykaz kompetencji poszczególnych zespołów:
\begin{description}
\item[Zespół zarządzający] -- dwóch studentów, pełniących w~głównej mierze role \emph{kierownika projektu} oraz~\emph{architekta}, ale~także wykonujący obowiązki \emph{analityka}. Kompetencje:
\begin{itemize}
\item Kontakt bezpośredni z~klientem oraz~jego przedstawicielami.
\item Decydowanie o~kierunku rozwoju projektu (w~tym: wybór technologii, rozwiązań,~itp.).
\item Utworzenie dokumentacji architektury projektu.
\item Tworzenie i~przydzielanie zadań do zespołu programistów.
\end{itemize}
\item[Zespół programistów] -- czterech studentów, pełniących role \emph{programistów}. Kompetencje:
\begin{itemize}
\item Przyjmowanie i~realizacja zadań przydzielonych zespołowi.
\item Zgłaszanie problemów z~implementacją założeń architektonicznych.
\item Opiniowanie decyzji w~procesie ich~podejmowania.
\end{itemize}
\end{description}

\section{Struktura pracy}
\label{Chapter15}

Praca została podzielona umownie na~trzy wzajemnie uzupełniające~się części. Część pierwsza, obejmująca rozdziały \ref{Chapter1}., \ref{Chapter8}. i~\ref{Chapter9}. oraz~sekcję \ref{Chapter62}. rozdziału \ref{Chapter6}., odnosi~się do~całości projektu z~punktu widzenia realizujących~go~osób. Opisane~są w~niej założenia, napotkane problemy oraz~wyciągnięte wnioski. Część druga -- rozdziały od~\ref{Chapter2}. do~\ref{Chapter5}. -- dotyczy charakterystyki projektu oraz~jego architektury. Oparto~ją na~danych pozyskanych od~zespołu zarządzającego projektem\footnote{\cite{Redmine:ProjDocs}}, dostępnych dla zespołu programistów w~ramach platformy \textit{Redmine}. Część trzecia, zawierająca pozostałe rozdziały, tj.~\ref{Chapter6}. i~\ref{Chapter7}. (z~wyłączeniem sekcji \ref{Chapter62}.), przedstawia projekt od~strony implementacji. Opisuje szczegółowo budowę systemu iQuest.