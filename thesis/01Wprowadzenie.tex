\chapter{Wprowadzenie}
\label{Chapter1}

\section{Opis problemu i koncepcja jego rozwiązania}
\label{Chapter11}

\subsection{Problem}
\label{Chapter111}

Podstawą każdej działalności jest warunkująca ją potrzeba. Niemożliwą jest jednak analiza potrzeb, bez wcześniejszego ich zbadania. Najłatwiejszą i najbardziej rozpowszechnioną metodą pozyskiwania wiedzy na jakiś temat jest pytanie o to osób związanych ze sprawą. Kiedy pytań jest wiele, a liczba potencjalnych respondentów osiąga poziom kilkudziesięciu tysięcy osób, zadawanie pytań wymaga usystematyzowania. Podobnie wygląda kwestia metodyki ich dostarczania do odpowiadających oraz gromadzenia odpowiedzi. Rozwiązaniem tego zagadnienia jest równie proste, co skuteczne narzędzie badań, pozwalające na gromadzenie danych - ankieta. \\

Ankieta pozwala na nieinwazyjne, anonimowe i masowe pozyskiwanie opinii respondentów na zadany temat, a dzięki zastosowaniu różnych metod statystycznych, także na ocenę zdania całej populacji. W kontekście każdej jednostki, zrzeszającej większą grupę osób, jak zakład pracy, czy jednostka szkolnictwa wyższego, jaką jest Politechnika Poznańska, ankietowanie jest niezwykle potrzebne, celem zapewnienia prawidłowej pracy. Bez informacji zwrotnej (ang.~\definicja{feedback}) praktycznie nie ma możliwości precyzyjnej analizy poprawności podejmowanych działań, co w efekcie zawsze będzie prowadzić do spiętrzającej się fali problemów. Za przykład może posłużyć sytuacja, gdy nie posiadając wiedzy o nieodpowiednim (zdaniem studentów i prowadzących zajęcia) wyposażeniu sali laboratoryjnej, władze uczelni ustalają harmonogram zajęć, zwiększający obciążenie tego pomieszczenia. Efektem takiego działania byłby spadek jakości kształcenia studentów korzystających z tego laboratorium. \\

W ostatnich latach\footnote{Mowa o okresie 2008-2013, który jest znany autorom niniejszej Pracy Dyplomowej.} Politechnika Poznańska posiadała szerokie spektrum narzędzi do pozyskiwania wiedzy o swoim działaniu i oferowanych usługach, funkcjonujących w zgodzie ze Statutem Politechniki Poznańskiej\footnote{Zapis odnośnie dokonywania oceny nauczyciela akademickiego z uwzględnieniem oceny studentów\cite{AP:SPP11}.}. Niestety, gros z nich wzajemnie się wykluczał. Przykładowy student na Wydziale Informatyki\footnote{Do 2010 roku -- Wydziale Informatyki i Zarządzania} poddawany był ankietom: elektronicznym na poziomie Uczelni, jak i Wydziału, ,,papierowym'' na poziomie Uczelni oraz Samorządu Studentów, a ponadto dodatkowej ankietyzacji w ramach niektórych zajęć dydaktycznych. \\

Żaden spośród tych ,,systemów'' ankietyzacji nie posiadał odpowiednich zabezpieczeń, wymaganych do zapewnienia wymierności ich wyników (jak np.~zagwarantowanie, aby nikt nie wziął w ankiecie udziału więcej niż jeden raz, czy uniemożliwienie wypełnienia ankiety przeznaczonej dla innego wydziału, kierunku). Brak wymiernych efektów udziału w tych badaniach, podobnie jak ich mnogość i dezorientacja zwiżana z pytaniem ,,kto tak naprawdę pozyskuje informacje'', działały tu na niekorzyść ankietujących. Ważnym jest też fakt, że istniała tylko jedna, globalna grupa docelowa -- studenci Politechniki Poznańskiej. \\

Ustawa o Szkolnictwie Wyższym wymaga od uczelni wyższych badania nie tylko aktualnej społeczności studenckiej, lecz także jej absolwentów\cite{AP:PoSW05}. Sprawdzane powinny być takie czynniki, jak rozwój karier czy satysfakcja z zapewnianych usług. Niestety, na Politechnice Poznańskiej brak jest odpowiednich narzędzi do prowadzenia tego rodzaju analiz. Nawiązując więc do wcześniejszych twierdzeń, potencjalna katastrofa (jak niezauważony spadek jakości usług, prowadzący do zmniejszenia satysfakcji odbiorców ze względu na brak podjęcia odpowiednich działań i w efekcie spadku prestiżu Uczelni) to jedynie kwestia czasu. \\

\subsection{Proponowane rozwiązanie}
\label{Chapter112}

Najłatwiejszym i -- w opinii autorów niniejszego dokumentu -- najlepiej popartym logicznymi argumentami rozwiązaniem, jest stworzenie jednego, jednolitego, globalnego i prostego w obsłudze systemu prowadzenia badań i ankiet dla różnych grup docelowych, obejmujących wszystkich aktualnych i byłych studentów Politechniki Poznańskiej. Aby zapewnić dostępność oraz prostotę wdrożenia systemu, który na to pozwoli, zostanie on wykonany za pomocą technologii internetowych, co umożliwi też jego obsługę z użyciem dowolnej popularnej przeglądarki internetowej dostępnej na rynku. \\

W trakcie analizy problemu ustalono, że koniecznym będzie także zapewnienie swojego rodzaju ,,zachęty'' dla potencjalnych respondentów -- o ile bowiem studentowi może zależeć na rozwoju jego uczelni, o tyle absolwent nie będzie czerpał z tego tytułu żadnych wymiernych korzyści. Wybrane dla systemu iQuest rozwiązanie obejmuje więc też element zachęcający do korzystania z systemu, jakim jest udzielanie im dostępu do unikalnych materiałów dydaktyczno-naukowych. \\

Aby zapewnić, że pozyskane w badaniach dane są prawidłowe, w proponowanym rozwiązaniu znajdą się systemy autoryzacji osób ankietowanych. Jest to jednak jedyny element wymagający sprawdzenia tożsamości respondenta -- same wyniki będą dla ankietera całkowicie anonimowe. \\

\section{Ograniczenia i zagrożenia dla projektu}
\label{Chapter12}

Największym ograniczeniem, a zarazem zagrożeniem dla całego projektu jest kwestia zachęty studentów i absolwentów do korzystania z nowego systemu ankietowania. Popularność całego systemu zależy w głównej mierze właśnie od tego, jakie materiały będą za jego pośrednictwem udostępniane oraz jakie będą warunki uzyskania dostępu do nich. \\

Rozpatrując jednak tę kwestię ze strony implementacyjnej, sporym zagrożeniem okazuje się wymaganie stosowania rozwiązań ,,otwartych'' i ,,wolnych'', ograniczające spektrum możliwych do wykorzystania technologii i zasobów. W tej kategorii, za poważne ograniczenie należałoby także uznać wyznaczony z góry termin zakończenia prac. Na projektowanie, implementację i wdrożenie rozwiązania przewidziany jest jedynie okres od września 2012, do lutego 2013. Uzupełniając tę kwestię o fakt braku wcześniejszego doświadczenia zespołu przy realizacji projektów opartych na technologiach internetowych, spełnienie wszystkich wymagań związanych z projektem może okazać się utrudnione. \\

Ostatecznie jednak, kwestia zachęty dla respondentów -- choć istotna dla powodzenia systemu w przyszłości -- w fazie tworzenia może być potraktowana jako poboczna. Kwestię ograniczonego czasu można natomiast nadrobić, stosując odpowiednie metody zarządzania czasem. Największym zagrożeniem pozostaje więc brak wcześniejszego doświadczenia w tego typu projektach oraz ograniczenia dotyczące możliwych do zastosowania technologii. \\

\section{Cele projektu}
\label{Chapter13}

Celem projektu \textit{iQuest} jest zbudowanie systemu umożliwiającego łatwe przeprowadzanie ankiet wśród studentów i absolwentów uczelni. System ten powinien:
\begin{itemize}
\item{zapewnić spełnienie przez Uczelnię zapisów Ustawy ,,Prawo o Szkolnictwie Wyższym'' dotyczących monitorowania rozwoju absolwentów Uczelni\cite{AP:PoSW05}}
\item{ujednolicić uczelniany system pozyskiwania informacji}
\item{oferować dużą elastyczność:
\begin{itemize}
\item{przy definiowaniu różnorodnych ankiet}
\item{przy tworzeniu i hierarchizacji grup respondentów}
\item{przy zachęcaniu do uczestnictwa w niej przez potencjalnych Respondentów}
\end{itemize}}
\item{oferować rozbudowane możliwości raportowania}
\item{odciążyć pracowników uczelni oraz Samorząd Studentów z obowiązków związanych z przeprowadzaniem konwencjonalnych (,,papierowych'') ankiet}
\end{itemize}

\section{Omówienie pracy}
\label{Chapter14}

Niniejszy dokument, będący pracą inżynierską w ramach studiów I-stopnia na kierunku Informatyka na Wydziale Informatyki Politechniki Poznańskiej, dotyczy systemu iQuest, służącego do badania opinii studentów i absolwentów Uczelni, realizującego zadania przytoczone w rozdziale \ref{Chapter13}.Praca ma formę dokumentacji technicznej dla osób mających w przyszłości rozwijać, wdrażać i obsługiwać system, choć jej odbiorcami są również członkowie komisji egzaminacyjnej. \\

Cała praca podzielona została umownie (w trakcie tworzenia) na trzy uzupełniające się części, zawierające po kilka rozdziałów. Część pierwsza, zawierająca rozdziały \ref{Chapter1}., \ref{Chapter8}. i \ref{Chapter9}. oraz sekcję \ref{Chapter62}. rozdziału \ref{Chapter6}., odnosi się do całości projektu z punktu widzenia jego autorów. W tej części opisane są założenia, napotkane problemy i wyciągnięte z pracy wnioski. Część druga, zawierająca rozdziały od \ref{Chapter2}. do \ref{Chapter5}., dotyczy charakterystyki projektu oraz jego architektury. Tę część oparto na danych pozyskanych od zespołu zarządzającego projektem\cite{Redmine:ProjDocs}, dostępnych dla zespołu w ramach platformy \definicja{Redmine}. Część trzecia, zawierająca pozostałe rozdziały, tj. \ref{Chapter6}. i \ref{Chapter7}, przedstawia projekt od strony implementacji. Zawiera szczegółowe informacje odnośnie budowy systemu iQuest. \\

Wraz z dalszym zagłębianiem się w treść, następuje przejście na coraz głębszy poziom abstrakcji. Zabieg ten miał na celu zagwarantowanie osobie czytającej stopniowe zagłębianie się w szczegóły projektu. Jednocześnie, czytający ma pewność, że otwierając konkretny rozdział niniejszego dokumentu, uzyska dostęp do najistotniejszych dla niego informacji.