\chapter{Opis implementacji}
\label{Chapter6}

\section{Wstęp}
\label{Chapter61}

Realizacja projektu \textit{iQuest} rozciągała się na okres trwający 5 miesięcy. W tym czasie zrealizowano dwa kolejne wydania. \\

W pierwszym miesiącu -- w trakcie praktyk studenckich odbywanych przez 3 spośród 4 członków zespołu programistów -- utworzono pierwszą wtyczkę do platformy \textit{Moodle}. Był to moduł logowania przez \textit{eKonto}. Następne 3 miesiące trwało utworzenie pierwszego wydania, obejmującego podstawowe mechanizmy tworzenia badań i ankiet, oraz ich przeprowadzania. Ostatni miesiąc -- przeznaczony na drugie wydanie -- zaowocował powstaniem mechanizmów zarządzania grupami docelowymi, tworzenia predefiniowanych zestawów odpowiedzi do pytań, oraz wieloma innymi dodatkami, bezpośrednio związanymi z wymaganiami funkcjonalnymi i pozafunkcjonalnymi, wyznaczonymi dla projektu. \\

W trakcie realizacji, zdarzały się dynamiczne zmiany podejścia do poszczególnych elementów systemu, co skutkowało zatwierdzaniem nowych decyzji projektowych i wymagało przebudowania gotowych już elementów. Dla zespołu programistów, było to uciążliwe i wymagało poświęcenia znaczących nakładów czasowych. \\

W dalszej części niniejszego rozdziału znajduje się analiza systemu \textit{iQuest} w kontekście implementacji. Rozpoczęty opisem napotkanych problemów i zastosowanych rozwiązań wywód, kontynuowany będzie opisem zastosowanych technologii. Wówczas nastąpi przejście do wyjaśnienia kwestii związanych z implementacją interfejsu i logiki, a także mechanizmów wiążących je ze sobą.

\section{Napotkane problemy i ich rozwiązania}
\label{Chapter62}

\subsection{Wstęp}
\label{Chapter621}
%KB

Problemy oraz ich rozwiązania zostały posortowane chronologicznie, zgodnie z kolejnością, w jakiej pojawiały się w czasie realizacji projektu. Celem ułatwienia ich analizy, wcześniej zamieszczono krótki wstęp teoretyczny opisujący budowę platformy \textit{Moodle}.

\subsection{Platforma \textit{Moodle}}
\label{Chapter622}
%KU

Po zalogowaniu do systemu użytkownik musi wybrać \textit{kurs}. Jest on największą częścią platformy \textit{Moodle} i przeważnie kojarzony jest z ,,przedmiotem''. Na kurs składa się kilka lub kilkanaście \textit{sekcji}. Odpowiadają one najczęściej konkretnym zajęciom, wydarzeniom lub np.~tygodniom. Najmniejszą jednostką w \textit{Moodle} jest \textit{aktywność}, będąca podstawowym typem modułów rozszerzających funkcjonalność platformy. \textit{Aktywnościami} są np.~\textit{Fora}, \textit{Głosowania}, czy \textit{Czat}. \\
Problem ten przewijał się przez cały czas implementacji systemu \textit{iQuest}.
Istnieje również inny typ modułów: \textit{zasoby}. Są to m.in.~własne \textit{strony internetowe}, \textit{pliki}, \textit{adresy URL}. Na potrzeby projektu została wyróżniona grupa \textit{materiały}. Zalicza się do niej wszystkie moduły inne niż \textit{iQuest}, czyli inne niż badania i związane z nimi ankiety. Moduły grupowane są w sekcje. Różnica pomiędzy tym, czy użytkownik znajduje się lub nie, w konkretnym elemencie platformy, jak moduł, czy kurs, nazywana jest \textit{kontekstem}. O ułożeniu i wyświetlaniu elementów na stronie decyduje \textit{formater}, definiując w ten sposób interfejs użytkownika.

\subsection{Inicjalizacja bazy danych}
\label{Chapter623}

Moduł iQuest do prawidłowego działania wymaga rozszerzenia istniejącej bazy danych platformy \textit{Moodle} o dodatkowe tabele, przechowujące niezbędne do spełnienia założonej funkcjonalności dane. \\

Do zaimportowania bazy danych przygotowanej przez Architekta, wykorzystano narzędzie wbudowane w platformę \textit{Moodle}: \definicja{XMLDB}. Gwarantuje ono bezobsługową instalację modułu w przyszłości. W trakcie pracy z tym narzędziem, znaleziony został błąd, uniemożliwiający zaimportowanie kluczy obcych do bazy. W efekcie, programiści musieli ręcznie utworzyć wszystkie klucze obce, przewidziane przez Architekta, co wymagało znaczących nakładów czasowych.

\subsection{Instalacja modułu}
\label{Chapter624}
Postanowiono, że wraz z instalacją modułu \textit{iSurvey} zawierającego logikę systemu iQuest, powinien automatycznie tworzyć się odpowiedni \definicja{kurs} związany jedynie z nim. Rozwiązanie to zmniejsza nakład czas wymagany do przygotowania platformy do użytku, oraz zapobiega pomyłkom związanym z ręcznym tworzeniem i konfiguracją systemu. \\

Konsekwencje takiego podejścia wyszły na jaw dopiero po jego zrealizowaniu. Okazało się, że podejście to uniemożliwia instalację modułu jednocześnie z całą platformą \textit{Moodle}, ponieważ w trakcie procesu instalacji platformy \textit{Moodle}, dodatkowe moduły instalowane są przed mechanizmami pozwalającymi na tworzenie \definicja{kursu}. Z tego względu, moduł \textit{iSurvey} należy dodawać do wcześniej zainstalowanej platformy.
\subsection{Formularze}
\label{Chapter625}

Projektując graficzny interfejs użytkownika, prędzej czy później pojawia się potrzeba wyboru narzędzia do projektowania formularzy. Rozważano kilka możliwości. Pierwszym, najbardziej naturalnym odruchem była idea zastosowania czystego języka \textit{HTML}. Pod uwagę brane było także zastosowanie wbudowanych w \textit{Moodle} \textit{API} -- \textit{Form API} oraz \textit{Output API}, jak też użycie zewnętrznych \textit{API}, nie związanych bezpośrednio z platformą. \\

Zastosowanie \textit{HTML} oraz zewnętrznych \textit{API} zostało odrzucone. Decyzję tę podjęto ze względu na obawę, że pisanie interfejsu w całości od nowa okaże się zbyt pracochłonne. Z uwagi na dość mocno ograniczony czas, nie chciano wyważać otwartych już drzwi, tworząc coś, co już wcześniej zostało przez kogoś zrealizowane, nawet kosztem tego, że interfejs nie wyglądał dokładnie tak, jak go zaprojektowano -- na pierwszym miejscu stawiano jego kompletność. Zastosowanie zewnętrznych \textit{API} było natomiast niezgodne z założeniem, mówiącym o korzystaniu z interfejsów \textit{Moodle} wszędzie tam, gdzie to możliwe. Niepożądanym było, aby użytkownik poczuł, że programowana wtyczka nie jest integralną częścią platformy. Co więcej, różnice w stosunku do oryginalnego wyglądu platformy mogłyby sprawić, że interfejs oceniony zostałby jako nieintuicyjny. To zaważyło na decyzji odnośnie zastosowania \textit{Output API} wbudowanego w \textit{Moodle}. \\

Wyżej wspomniane API jest zestawem funkcji, wprowadzonych wraz wersją $2.0$ Moodle. Umożliwiają one wstawianie na stronę standardowych elementów formularza, takich jak: etykiety, przyciski, linki, tabele, itp. Niestety, z nieznanych dla osób tworzących ten dokument przyczyn, to doskonale wyposażone i w pełni udokumentowane API zostało usunięte z Moodle wraz z aktualizacją do wersji $2.2$. Co bardziej niezrozumiałe, niektóre elementy można nadal stosować, lecz niemożliwym okazało się odnalezienie dotyczącej tego dokumentacji. \\

Ostatecznie, realizacja interfejsu musiała odbyć się z użyciem \textit{Form API}. Ku rozczarowaniu zespołu programistów, ma on znacznie uboższą dokumentację. Zdarza się, że w funkcji są omówione np. tylko trzy pierwsze argumenty, podczas gdy reszta jest pominięta -- tak jakby kompletnie nie istniała. \textit{Form API} różni się też od poprzednich API tym, że jest oparte na modelu obiektowym. To sprawia, że zawiera szereg zalet, jak np.~fakt posiadania mechanizmu pozwalającego na weryfikację danych wprowadzanych przez użytkownika z użyciem \textit{JavaScript}. Mechanizm ten waliduje dane po stronie klienta, pozwalając na przesłanie do serwera jedynie poprawnych informacji. \\

Podsumowując, mimo niekompletnej dokumentacji, jako narzędzie implementacji formularzy wybrano \textit{Form API}. Choć zawiera ono sporo zalet, nie wszystkie potrzebne elementy udało się stworzyć korzystając jedynie z niego. Stosowano wówczas język \textit{HTML}. 

\subsection{Role}
\label{Chapter626}

Jedną z cech projektu, jest podział użytkowników na ankieterów i respondentów. W systemie Moodle istnieje mechanizm do zarządzania rolami, który wydawał się adekwatny do użycia w tym przypadku. Rola jest to zbiór \textit{możliwości} (ang. \definicja{capability}), które można rozumieć, jako prawa do wykonania, określonego przez programistę, fragmentu kodu. Zdecydowano więc o zastosowaniu tego gotowego rozwiązania.

\subsection{Formater kursu}
\label{Chapter627}

Jednym z problemów jakie napotkano, była konieczność wyświetlania respondentom i ankieterom tylko określonych modułów. Ankieterzy powinni zobaczyć tylko te badania, które utworzyli, lub które im udostępniono, wraz z innymi aktywnościami i zasobami. Respondenci powinni natomiast zobaczyć tylko te badania, w których mogą wypełnić wziąć udział, a także materiały, do których pozyskali prawa do ich odczytu.

Do rozwiązania problemu zdecydowano się użyć formatera kursu. Narzędzie to, jako integralna część Moodle, wydawało się najlepszym rozwiązaniem spośród dostępnych. W krótkim czasie okazało się jednak, że niekompletność dokumentacji, czy nawet merytorycznych dyskusji na temat w Internecie znacząco uprzykrza wykonanie zadania. Całą pracę wejścia, polegającą na poznaniu narzędzia, wykonano studiując kod źródłowy domyślnych formaterów dostępnych w Moodle.

Pierwotnie zakładano wyświetlanie użytkownikowi dwóch sekcji. Jednej z odpowiednimi badaniami, drugiej z materiałami. Należało także ograniczyć ankieterowi możliwość dodawania w pierwszej sekcji modułów innych niż iQuest, oraz dokładnie odwrotnego działania w drugiej z nich. Okazało się to nieosiągalne bez ingerowania w wewnętrzny kod platformy.

Przyczyną były uaktualnienia zastosowane w Moodle. Kod PHP wyświetlania typów modułów jest nadpisywany przez \textit{JavaScript}. W ten sposób, z poziomu funkcji PHP odpowiedzialnych za wyświetlanie listy modułów w danej sekcji, nie da się kontrolować, które moduły zostaną wyświetlane, a które nie. Mówiąc prościej, programista może jedynie wybrać, jakie moduły będą wyświetlane we wszystkich sekcjach w danym kursie, nie mogąc decydować, co można wykonywać w każdej sekcji z osobna.

Rozwiązaniem było umieszczenie listy badań oraz listy materiałów w jednej sekcji. Można w niej dodać jakikolwiek moduł. Dopiero przy wyświetlaniu moduły dzielone są na dwie listy: listę badań i listę materiałów. Dzięki temu cel został osiągnięty -- użytkownik zobaczy tylko te moduły, które ma prawo wyświetlać. Co więcej, będą one odpowiednio posegregowane, aby użytkownik szybko mógł znaleźć to, czego szuka.

\subsection{Tworzenie badania}
\label{Chapter628}

Kolejną trudnością w projekcie było połączenie utworzonej dla systemu iQuest wtyczki z platformą Moodle. Głównie sprowadzało się to do wykorzystania interfejsu graficznego Moodle w sposób niwelujący uczucie zmiany systemu u użytkownika. Zarówno wygląd, jak i sposób wykorzystywania funkcjonalności, powinny być zgodne ze standardem Moodle. Dzięki takiemu podejściu, osoba korzystająca wcześniej z platformy, a pragnąca używać wtyczki iQuest, nie będzie musiała zmieniać swoich przyzwyczajeń. Co więcej, w projekcie duży nacisk został postawiony na zachęcanie respondentów do wypełnienia ankiety, co było dodatkową motywacją do zaprojektowania przyjaznego użytkownikom interfejsu.

Wstępna wersja interfejsu, zaprojektowana przez Architekta, działała wedle następującego schematu: ankieter wyrażał chęć utworzenia nowego badania poprzez kliknięcie odpowiedniego przycisku. Wówczas mógł dodać do badania ankietę z katalogu, ewentualnie utworzyć nową. W kolejnych krokach, użytkownik definiował szczegóły badania, takie jak: nazwa, grupa docelowa, czas rozpoczęcia i zakończenia itp. Niestety, realizacja takiego rozwiązania okazała się niemożliwa.

Problem stwarzało dodawanie ankiety w trakcie procesu tworzenia badania, jeszcze przed jego zakończeniem. Zaczynając generowanie badania od zdefiniowania ankiety, nie można było jej od razu do niego dodać -- badanie to bowiem jeszcze nie istniało. W takim wypadku należałoby przechowywać informację, że po utworzeniu badania ma dodać się do niego ankieta\footnote{Przykładowo można w tym celu wykorzystać dodatkowy parametr w adresie \textit{URL}, choć stwarzałoby to potencjalny problem dotyczący kwestii liczby przekierowań, przez które musiałby on być przekazywany.}. Dodatkowo, w Moodle, przy kreowaniu nowego modułu, użytkownikowi wyświetlany jest domyślny formularz, w którym podaje się parametry potrzebne do zbudowania instancji tego modułu. Przyjmując, że badanie jest kojarzone z modułem, nie ma możliwości, aby przed zakończeniem tworzenia badania wstawić wewnątrz dodatkowy formularz.

Z tego względu, zamieniona została kolejność tworzenia badania i ankiety. Najpierw użytkownik tworzy badanie, czyli moduł realizujący ankietę. Dopiero wówczas ma możliwość załączenia do niego ankiety. Podejście to ma kilka zalet: jest to zgodne z procedurami charakterystycznymi dla Moodle, a co za tym idzie, bardziej intuicyjne dla użytkownika obeznanego z platformą, a jednocześnie pozwala na łatwe dodanie ankiety do badania. Ponadto ankieter może zrezygnować z komponowania ankiety przy kreowaniu badania, odkładając to -- znacznie bardziej czasochłonne -- zadanie na później.

\subsection{Tworzenie ankiety}
\label{Chapter629}

Przy tworzeniu ankiety pojawił się dość specyficzny problem implementacyjny. Wynikał on z faktu, że ankietę definiować można zarówno z poziomu kursu, jak i z poziomu badania. Powstało pytanie: Jak przetwarzać dane pochodzące z różnych, niezależnych od siebie kontekstów?

Standardowo, we wtyczkach Moodle, elementy odpowiedzialne za wyświetlanie informacji na ekranie znajdują się w pliku \textit{view.php}. Pojawił się pomysł, aby rozszerzyć strukturę o dwa dodatkowe pliki: \textit{mod.php} oraz \textit{course.php}. Do pliku \textit{mod.php} trafiać miały dane z kontekstu modułu. Drugi plik zajmować miałby się przetwarzaniem danych z kontekstu kursu. Taki podział gwarantował większy porządek w kodzie źródłowym. Porządek był ważny, ponieważ Moodle nie jest tu zgodny ze wzorcem Model-View-Controller. W związku z tym istotne jest aby efektywnie zarządzać kodem źródłowym, żeby mała jego zmiana nie wymagała zmiany wielu elementów.

Niestety wprowadzone zmiany okazały się niewystarczające. Występowało niepotrzebne powielanie kodu. Wydzielono jeszcze jeden plik, w którym przetwarzano dane otrzymane z formularzy i zapisywano je do bazy danych. Później zwracano sterowanie do plików wspomnianych plików, w zależności od kontekstu.

Dzięki utworzeniu trzech dodatkowych plików, kod źródłowy stał się bardziej przejrzysty. Wartość takiego rozwiązania można zauważyć dopiero, gdy zachodzi konieczność znalezienia błędu lub wprowadzenia modyfikacji do kodu. Przy dobrym zarządzaniu kodem mała zmiana wymaga nieznacznych tylko poprawek.

\subsection{Hierarchia CSS}
\label{Chapter62a}

Na wielu poziomach serwisu borykano się z problemem hierarchii plików \textit{CSS}. Twórcy platformy Moodle po, jak zapewniają, gruntownym przemyśleniu sprawy i rozważeniu wszystkich możliwości, ustalili następującą hierarchię kaskadowych arkuszy stylów:
\begin{description}
\item Najważniejsze są pliki umieszczone w katalogu \textit{theme}, odnoszące się do całej platformy.
\item Następnie uwzględniane są reguły z pliku \textit{styles.css}, umieszczonego w katalogu konkretnej wtyczki.
\end{description}
Główną wadą tego podejścia jest fakt, że nie można we wtyczce nadpisać właściwości, która została zdefiniowana w katalogu \textit{theme}. Aby zmienić choćby jedną właściwość z tego katalogu należy utworzyć nowy \definicja{wygląd}, kopiując oryginalny i zmieniając tę jedną właściwość. Następnie administrator platformy musi ustawić ten wygląd w swoim systemie (co wiąże się z dodatkową operacją, jeśli poprzedni wygląd był wyglądem domyślnym). Problem ten nie mógł zostać rozwiązany i przewijał się przez cały czas implementacji systemu iQuest.

\subsection{Testy jednostkowe i akceptacyjne}
\label{Chapter62b}

Realizacja wszystkich testów została pierwotnie powierzona jednemu z członków zespołu programistów. Ze względu na brak precyzyjnej architektury logiki we wczesnej fazie projektu, początkowe utrzymanie testów okazało się być bardzo czasochłonne. \\

Przyczyną takiego stanu był także stały rozwój systemu. Realizowany bez zastosowania metody rozwoju w oparciu o testowanie, \textit{iQuest} ze znaczną szybkością ewoluował niezależnie od testów. Nowe parametry, nowe wartości wyjściowe oraz zmiana dostępności poszczególnych funkcji sprawiały, że utrzymywanie testów zajmowało nawet kilkudziesięciokrotnie więcej czasu, niż ich utworzenie od nowa. \\

Problem ten występował jednak jedynie w trakcie pierwszego wydania projektu. Przy drugim wydaniu, działalność związaną z testami jednostkowymi w całości przejął programista logiki aplikacji, co znacząco zmniejszyło czasochłonność ich realizacji. \\

Większy problem tyczył się testów akceptacyjnych. Już na początku realizacji projektu, wyszło na jaw, że eksport z \textit{Selenium IDE} (standard \textit{HTML}) do \textit{Eclipse IDE} (standard \textit{Java}) nie jest zadaniem prostym. Ze względu na jego czasochłonność, w szczególności przy utrzymywaniu testów, zdecydowano o pominięciu tego kroku. \\

\subsection{Mapowanie obiektowo-relacyjne}
\label{Chapter62c}
%LW

Mapowanie obiektowo-relacyjne pozwala uprościć operacje na danych przechowywanych w bazie danych poprzez udostępnienie ich programiście w postaci obiektowej. System \textit{iQuest} operuje na klasach takich jak: ankieta, badanie, grupa docelowa, członek grupy docelowej, uprawnienie dostępu, pytanie (i potomne), odpowiedź, zadanie, praca w tle, etc. Początkowo architekt stworzył diagram klas na którym każda klasa miała wyróżnione publiczne metody \textit{insert, update, delete}. Niestety takie rozwiązanie spowodowało powielenie dużej ilości kodu związanego z interakcją z bazą danych. W ramach refaktoryzacji podjęto się zadania stworzenia klas, które wzorem nowoczesnych systemów \textit{ORM} uproszczą projektowanie nowych klas reprezentujących dane. Ze względu na silną integrację istniejącego już kodu z mechanizmami \textit{Moodle} w grę nie wchodziły gotowe rozwiązania. Autorskie rozwiązanie korzysta z mechanizmów \textit{Moodle Data manipulation API} oraz mechanizmu refleksji języka \textit{PHP}, by pozwolić programiście korzystającemu z tego rozwiązania na proste pobieranie i manipulację obiektami przechowywanymi w bazie danych. Diagram UML przedstawia się następująco:

\newpage
\begin{figure}[H]
\begin{center}
\includepdf{figures/lw/orm.pdf} 
\end{center}
\caption{iQuest ORM}\label{fig:iquest-orm}
\end{figure}
\newpage

Klasa danych dziedziczy po klasie \textit{record} oraz implementuje statyczną metodę \textit{get\_mapping} interfejsu \textit{irecord}, by uzyskać dostęp do metod komunikacji z bazą danych. Metoda \textit{get\_mapping} pozwala zdefiniować mapowanie danej klasy na odpowiednią relację w bazie danych. Należy przy tym podać nazwę tabeli, mapowanie dla pól klasy składające się z mapowania nazw (klasa \textit{names\_mapping}), tj.~nazwy w klasie i bazie danych oraz typu, który zadecyduje o metodzie pobrania/zapisania danej (typem może być także klasa potomna klasy \textit{record}). Dodatkowo można uwzględnić istnienie kluczy obcych, których poprawność będzie sprawdzana, jeżeli utworzymy obiekt klasy \textit{reference\_mapping}. W przypadku dziedziczenia wystarczy zdefiniować przy mapowaniu sposób jego obsługi (m.in. jakie pole określa typ klasy). Najważniejszy kod znajduje się w metodzie \textit{get\_instance}, która pobiera konstruktor danej klasy, poprzez refleksję tworzy obiekt i na podstawie pobranych z bazy danych informacji, ustawia resztę pól. Metody \textit{insert, update, delete} pobierają reprezentację obiektu oczekiwanego przez metody \textit{Data manipulation API} oraz wykonują żądane operacje.\\
Zastosowane rozwiązanie znacząco poprawiło czytelność kodu poprzez zastosowanie zasady DRY (ang. \definicja{Don't repeat yourself}). Projektowano je, mając na uwadze rozwiązanie, z którym programista logiki miał już wcześniej styczność, tj. implementację \textit{ActiveRecord} z \textit{Ruby on Rails}. W trakcie pracy nad projektem doceniono stosowanie konwencji nazewniczych, których obecność znacząco upraszcza projektowanie klas mapujących dane.

\subsection{Inwencja programistów}
\label{Chapter62d}

W trakcie rozwoju oprogramowania pojawiło się kilka małych niejasności, które należało rozwiązać. Kilka razy wykazano również inicjatywę i zaproponowano rozwiązania, które stały się ostatecznie częścią projektu. \\

Pierwszą ideą było zagospodarowanie przestrzeni w widoku badania. Po utworzeniu badania i dodania do niego ankiety, ankieterowi ukazuje się widok badania. Architekt nie zaproponował jak ma on wyglądać. Dał tylko pewne wskazówki. Zaznaczył, że z tego widoku, ankieter ma mieć możliwość usunięcia ankiety z badania oraz edytowania jej. Żeby spełnić wymagania, na stronie wystarczyło pokazać odnośniki: ,,edytuj'' i ,,usuń z badania''. Praktycznie cała strona pozostawała pusta. Sytuacja taka jest niedopuszczalna, gdyż zawsze istnieją przydatne informacje, które można w takim miejscu wyświetlić. Ustalono, że najbardziej naturalnym będzie zamieścić w tym miejscu podstawowe statystyki badania. \\

W pierwszej wersji zaimplementowano tylko proste parametry, takie jak: ile czasu zostało do zakończenia badania, ile osób liczy grupa docelowa oraz ile osób już odpowiedziało. Wraz z rozwojem systemu, dodano kolejną tabelę z danymi. Wyświetla się ona, gdy choć jedna osoba odpowie na któreś pytanie. Możemy na jej podstawie przeanalizować, jak kształtowały się odpowiedzi w pytaniach zamkniętych. Nie zdecydowano się wyświetlać odpowiedzi na pytania otwarte, ze względu na ich różnorodność -- negatywnie wpływałoby to na czytelność strony. Ideą tabeli było pokazanie skróconych informacji o badaniu. Dokładny, rozbudowany raport, można wygenerować z użyciem systemu \textit{Jasper Report}. \\

Pozyskanie i podliczenie odpowiedzi dla danego pytania wiązało się z zastosowaniem odpowiedniego algorytmu. Teoretycznie najprostszym rozwiązaniem byłoby sprawdzanie liczby krotek związanych z danym pytaniem w tabeli \textit{answers}. To rozwiązanie jest jednak nieoptymalne, gdyż wiąże się z wielokrotnymi odwołaniami bazy danych. Lepszym wyborem było użycie wbudowanych mechanizmów systemu zarządzania bazą danych celem optymalizacji odwołania do tablic. Przy użyciu wyrażenia ,,GROUP BY'' opracowano zapytanie, które od razu zwraca liczbę odpowiedzi respondentów w dany sposób, co pozytywnie wpłynęło na szybkość działania algorytmu. \\

Kolejna kwestia tyczy się odnośników, zwiększających intuicyjność pracy z systemem. Zarówno w katalogu jak i widoku badania umieszczono przycisk ,,pokaż''. Służy on do wyświetlenia ankiety w taki sam sposób, w jaki widzi ją respondent. Dzięki temu, że ankieter może zobaczyć układ pytań, łatwiej mu zdecydować o np.~ podziale na strony. W widoku katalogu pojawił się także przycisk pozwalający na dodanie nowej ankiety. Znajduję się on zarówno w dolnej, jak i górnej części tabeli katalogu, co zwalnia użytkownika z konieczności mozolnego przewijania strony. \\

W założeniach projektu ustalono, że raz udzielona odpowiedź na pytanie jest nieedytowalna. Z tego względu, dodano udoskonalenie, które polega na tym, że respondent nie musi od razu wypełnić całej ankiety. Może to robić stopniowo - za każdym razem jednak zostaną mu wyświetlone tylko te pytania, na które jeszcze nie odpowiedział. Pozwala to także uniknąć sytuacji, w której respondent przeoczy jakieś pytania. Jeśli respondent nie wypełni całej ankiety, to badanie nie zniknie z widoku kursu. Dopiero po wypełnieniu całej ankiety, badanie już więcej nie pojawi się w kursie.

\section{Użyte technologie}
\label{Chapter63}

\subsection{\textit{Moodle}}
\label{Chapter631}

\textit{Moodle} (ang. \definicja{Modular Object-Oriented Dynamic Learning Environment}) -- stanowi podstawę systemu \textit{iQuest}. Jest to popularna (ponad 63 miliony użytkowników) platforma e-learningowa o otwartym kodzie źródłowym, napisana w języku \emph{PHP}. Wyboru dokonano ze względu na kilka czynników:
\begin{itemize}
\item{Propozycję Kierownika Projektu oraz następującą po niej decyzję Architekta, wynikającą z faktu, iż \textit{Moodle} posiada już implementację wielu wymaganych w \textit{iQuest} mechanizmów, jak np. konta użytkowników, system ról i uprawnień.}
\item{Oczekiwania Klienta, wynikające z popularności platformy \textit{Moodle} wśród systemów uczelnianych.}
\item{Modułowość \textit{Moodle}, umożliwiającą pisanie rozszerzeń.}
\end{itemize}

\subsection{\textit{PHP}}
\label{Chapter632}

\textit{PHP} -- platforma \textit{Moodle} jest napisana właśnie w tym języku programowania. Z tego względu, jest to technologia zastosowana w większości rozszerzeń utworzonych przez zespół textit{iQuest}, korzystających z interfejsów programowania aplikacji tej platformy. Ponadto \textit{PHP} jest jednym z najpopularniejszych języków programowania aplikacji internetowych, posiada doskonałą dokumentację\footnote{\cite{Man:PHP}} oraz jest cały czas rozwijany.

Dodatkowo, język \textit{PHP} jest dość przyjazny dla programisty, gdy chodzi o komunikację z bazą danych. Przy realizacji zadań z tym związanych, odnoszono się zarówno do wspomnianej wyżej dokumentacji, jak też do literatury fachowej, w tym ,,PHP i MySQL - Księga przykładów'' autorstwa E. Quigley oraz M. Gargenta\footnote{\cite{EQMG:PiMKp07}}.

\subsection{\textit{PHPUnit}}
\label{Chapter633}

Ze względu na fakt, iż programiści \textit{Moodle'a} wykonują testy jednostkowe kodu wykorzystując do tego celu \textit{PHPUnit}, zdecydowano się skorzystać z przygotowanego przez nich oprogramowania. \textit{Moodle} udostępnia dwie klasy do testowania -- \textit{basic\_testcase} i \textit{advanced\_testcase}, przy czym druga wymieniona służy do testów, które wchodzą w interakcję z bazą danych. Korzystanie z tych klas dodatkowo uprasza fakt istnienia świetnej dokumentacji technicznej\footnote{\cite{Man:PHPUnit}}.

\subsection{\textit{Selenium}}
\label{Chapter634}

\textit{Selenium} -- szybko rozwijający się zestaw narzędzi do testów akceptacyjnych. Był to naturalny wybór zwłaszcza, że zostało ono przybliżone programistom na zajęciach z Inżynierii Oprogramowania w trakcie toku studiów. \textit{Selenium} składa się m.in. z następującego oprogramowania\footnote{\cite{Man:Selenium}}:
\begin{itemize}
\item{\textit{Selenium IDE} -- zintegrowane środowisko programistyczne dla skryptów \textit{Selenium} -- zaimplementowane jako rozszerzenie dla przeglądarki internetowej \textit{Firefox}. Pozwala na: nagrywanie i odtwarzanie sekwencji kroków, wykonywanych podczas pracy z przeglądarką, eksport skryptów do kodu języków programowania (np. \textit{Java}).}
\item{\textit{Selenium Client Drivers} (\textit{Java}) -- sterownik klienta dla języka \textit{Java}, pozwalający na wykonywanie skryptów \textit{Selenium} z poziomu języka \textit{Java}},
\item{\textit{HtmlUnit Driver} -- Implementacja klasy \textit{WebDriver}, która emuluje zachowanie przeglądarki. Pozwala na uruchamianie skryptów \textit{Selenium} bez korzystania z przeglądarki internetowej.}
\end{itemize}

\subsection{\textit{PostgreSQL}}
\label{Chapter635}

System zarządzania bazą danych \textit{PostgreSQL} został wybrany ze względu na wymaganie pozafunkcjonalne -- pracownicy \textit{Działu Rozwoju Oprogramowania Politechniki Poznańskiej} korzystają z tej właśnie bazy danych. Jest to baza danych o otwartym kodzie źródłowym, zgodna ze standardami, ciągle rozwijana, wysoce konfigurowalna.

\subsection{\textit{Eclipse IDE}}
\label{Chapter636}

Wybór \textit{Eclipse IDE} jako stosowanego dla projektu \textit{iQuest} zintegrowanego środowiska programistycznego wynika z faktu, iż oprogramowanie to jest dostępne za darmo. Dodatkową zaletą \textit{Eclipse} jest modularność tego rozwiązania, dzięki czemu dostępny jest w nim dodatek \textit{PHP Development Tools}, upraszczający pracę z technologią \textit{PHP}. Udostępnia m.in. narzędzia do analizy poprawności składniowej pisanego kodu, formatery kodu, wyszukiwanie fraz w wielu plikach, kontekstowe podpowiedzi i nawigację.

\subsection{\textit{SVN}}
\label{Chapter637}

\textit{Subversion} został wybrany jako podstawowy system kontroli wersji ze względu na wymagania pozafunkcjonalne. Zespół eksploatacji, który docelowo przejmie zarządzanie artefaktami związanymi z projektem, wykorzystuje właśnie \textit{SVN}. Główne funkcjonalności tego systemu to: atomowe publikowanie zmian, historia operacji na plikach (zmiana nazwy, skopiowanie, przeniesienie, modyfikacja, usunięcie), wersjonowanie plików i folderów, łatwy dostęp do informacji o zmianach.

\subsection{\textit{Redmine}}
\label{Chapter638}

Systemu zarządzania projektami \textit{Redmine} wykorzystywany był od samego początku istnienia projektu. Jest to narzędzie bardzo przydatne w wymianie informacji pomiędzy członkami zespołu, integrujące się m.in. z repozytorium kodu, bazą wiedzy o projekcie, listą zagadnień, forum.  Technologia ta została narzucona, ze względu na sposób organizacji pracy w \textit{Software Development Studio} na Politechnice Poznańskiej.

\subsection{\textit{JasperReports}}
\label{Chapter639}

Ze względu na wymagania pozafunkcjonalne, zdecydowano się skorzystać z mechanizmów raportowania oferowanych przez \textit{JasperReports}. Jest to najbardziej popularny silnik raportowania o otwartym kodzie źródłowym (wersja Community). Pozwala na generację raportów, których treść jest określona z dokładnością co do piksela. Generowane raporty można eksportować do popularnych formatów dokumentów, np. \textit{HTML}, \textit{PDF}, \textit{Excel}, \textit{Word}.

\subsection{\textit{JetBrains PhpStorm}}
\label{Chapter63a}

\textit{PhpStorm} jest komercyjnym \textit{IDE} dla języka \textit{PHP} tworzonym przez firmę \textit{JetBrains}, dostępnym dla studentów na licencji edukacyjnej. Dla celów niniejszej pracy dyplomowej, wykorzystano funkcjonalność odpowiedzialną za tworzenie diagramów UML z kodu. Wygenerowane w trybie \textit{Organic} diagramy można zobaczyć w dalszej części pracy (schematy \ref{rys:back-end1}., \ref{rys:back-end2}., \ref{fig:tests1}., \ref{fig:tests2}., zamieszczone w rozdziałach \ref{Chapter6}. i \ref{Chapter7}.).

\subsection{\textit{HTML} oraz \textit{CSS}}
\label{Chapter63b}
Po stronie użytkownika, system \textit{iQuest} prezentowany jest za pośrednictwem interpretowanego przez jego przeglądarkę internetową kodu w języku \textit{HTML}. Jego wygląd, wraz z umiejscowieniem elementów, określa natomiast arkusz stylów \textit{CSS}. Realizując zadania związane z tymi technologiami, odnoszono się do profesjonalnej literatury branżowej\footnote{M.in. ,,Wstępu do CSS3 i HTML5'' autorstwa Bartosza Danowskiego\cite{BD:WdCiH11}}.
\subsection{\textit{JavaScript}}
\label{Chapter63c}

Formularze wymagające częstej interakcji z klientem, np.~formularz umożliwiający tworzenie nowej ankiety oraz~funkcje związane z~walidacją pól uzupełnianych przez~klienta zostały napisane w~\textit{JavaScript}. Obsługa strony po~stronie klienta jest dla~użytkownika znacznie wygodniejsza, gdyż~nie~wymaga częstego przeładowywania całej strony. Dodatkowo, ogranicza to~obciążenie łącza zarówno w~lokalizacji serwera, jak i~klienta, co~jest korzystne dla~obu~stron.

\section{Ogólna struktura projektu}
\label{Chapter64}

Zgodnie z ideą trójwarstwowej architektury, opisanej w rozdziale \ref{Chapter5}., system \textit{iQuest} podzielono na trzy warstwy, w tym: prezentacji i logiki biznesowej. Obie z nich są ze sobą ściśle powiązane. Szczegóły z tym związane przedstawiają rozdziały \ref{Chapter65}-\ref{Chapter67}.

\section{Interfejs}
\label{Chapter65}

Jedną z części pracy było zaprojektowanie graficznego interfejsu użytkownika. Głównym problemem jaki się pojawił, był wybór odpowiedniego narzędzia. Celem jaki postawiono, była maksymalna zgodność projektowanych elementów z różnymi wersjami \emph{Moodle} -- zarówno wcześniejszymi, jak i późniejszymi. Zdecydowano, aby starać się korzystać z gotowych interfejsów programowania aplikacji \emph{(API)} dostarczonych przez \emph{Moodle}, tj. \emph{Page API}, \emph{Form API}, oraz \emph{Access API}. Wszystkie interfejsy są napisane przy użyciu języka PHP -- są wykonywane po stronie serwera. Konieczne okazało się też wykonanie niektórych skryptów po stronie klienta. Dlatego w projekcie wykorzystano również język skryptowy \emph{Java Script}.
\section{Logika (back-end)}
\label{Chapter66}

Jednym z zadań w ramach pracy było zaprogramowanie odpowiedniej logiki biznesowej rozwiązującej zadania stawiane przed zaprojektowanym systemem. Najważniejszym zadaniem z perspektywy back-end'u jest interakcja z bazą danych. Poza tym system posiada: procesor zadań wykonywanych w tle oparty na \emph{cron}; moduł odpowiadający za komunikację z systemem uczelanianym \emph{ePoczta}; moduł logowania zdarzeń. W trakcie implementacji zdecydowano się nie tworzyć osobnego mechanizmu do przechowywania ustawień w bazie danych i skorzystaliśmy z istniejącego już w \emph{Moodle}. Jednym z wymagań pozafunkcjonalnych było wykorzystanie bazy danych \emph{PostgreSQL}. Platforma \emph{Moodle} korzysta z mechanizmu \emph{XMLDB}, co pozwala na ominięcie wielu problemów pojawiających się przy migracjach pomiędzy różnymi systemami baz danych. Niestety kosztem wykorzystania tego mechanizmu jest konieczność pracy z interfejsami programowania aplikacji dostarczanymi przez platformę \emph{Moodle}, m.in. \emph{Data manipulation API} (zarządzanie danymi), \emph{Access API} (zarządzanie dostępem, rolami, prawami).\\
\section{Powiązanie back-endu z interfejsem}
\label{Chapter67}

%Ja to robiłem - KU
\emph{Moodle} jako paradygmat programowania stosuje podejście proceduralne. Natomiast logika zaprogramowanej wtyczki podejście obiektowe. Zaprojektowano więc mechanizm łączący te dwa sposoby programowania.

Mechanizm łączący stosuje podejście proceduralne. Zaimplementowano szereg dodatkowych funkcji, które operują na danych zwracanych przez formularze. Zamiast zapisywać je bezpośrednio do bazy danych, tworzą najpierw obiekty, które dopiero później zapisywane są do bazy. Cały proces odbywa się po stronie serwera i jest zapisany w języku PHP. 

Mimo większej złożoności, zastosowano ten sposób, aby w jak największej części projektu użyć programowania obiektowego, które pozwala na lepszą organizację kodu, a przez to np. na szybsze wykrycie ewentualnych błędów oraz minimalizację powielania kodu. Oczywiście są to tylko niektóre z zalet programowania obiektowego.

\subsection{Instalacja \textit{iQuest}}
\label{Chapter68}
Instalacja systemu realizowana jest trójstopniowo. Pierwszym krokiem jest umieszczenie na serwerze docelowym plików platformy \textit{Moodle} i ich instalacja oraz wstępna konfiguracja. W tak przygotowanym środowisku, umieszczane są następnie pliki systemu \textit{iQuest}, które -- dzięki wbudowanym mechanizmom platformy \textit{Moodle} -- administrator może w bardzo prosty sposób zainstalować i aktywować. Aktualizacja systemu przebiega bardzo podobnie, i polega jedynie na przeprowadzeniu standardowego dla \textit{Moodle} procesu aktualizacji, poprzedzonego zmianą odpowiednich plików na serwerze, w ramach którego działa system.