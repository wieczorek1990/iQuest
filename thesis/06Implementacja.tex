\chapter{Opis implementacji}
\label{Chapter6}

\section{Wstęp}
\label{Chapter61}

{\color{red}Wersja bez redakcji.}

%Wprowadzenie. Struktura tego rozdziału nie jest z góry określona, gdyż mocno zależy to od specyfiki projektu. Generalnie w poszczególnych podrozdziałach każdy powinien opisać swoją część z takiego technicznego punktu widzenia. Piszecie, jak zrealizowaliście poszczególne wymagania, jak to wygląda ,,pod maską'', oczywiście też trzeba przyjąć jakiś poziom szczegółowości. W bardzo szczególnych przypadkach chyba może się zdarzyć, że trzeba będzie załączyć fragment jakiegoś kodu źródłowego czy konfiguracji -- generalnie ma to być opisane w taki sposób, że jako osoba nieznająca systemu siadam i wiem, jak i co zrobiliście. Oczywiście, to moje dywagacje, być może osoby związane z uczelnią zlinczują mnie za ten fragment.
%
\subsection{Użyte technologie}
W poniższym rozdziale postaram się wyjaśnić wybór odpowiednich technologii użytych w pracy.
\subsubsection{PHP}
\emph{PHP} to język w którym napisany jest system \emph{Moodle}, dlatego też jest językiem większości naszych rozszerzeń.
\subsubsection{PHPUnit}
Ze względu na fakt, iż programiści \emph{Moodle'a} testują jednostkowo kod tejże platformy z użyciem \emph{PHPUnit} zdecydowaliśmy się na to samo. \emph{Moodle} udostępnia dwie klasy do testowania, tj. \emph{basic\_testcase} i \emph{advanced\_testcase}, przy czym ta druga służy do testów, które wchodzą w interakcję z bazą danych.
\subsubsection{Selenium}
\emph{Selenium} to szybko rozwijające się narzędzie do testów akceptacyjnych. Był to naturalny wybór zwłaszcza, że zostało ono nam przybliżone już na zajęciach z Inżynierii Oprogramowania na 6. semestrze studiów.
\subsubsection{PostgreSQL}
Ze względu na wymaganie pozafunkcjonalne wybraliśmy system zarządzania bazą danych \emph{PostgreSQL}.
\subsubsection{Eclipse}
Ze względu na darmową dostępność jako zintegrowane środowisko programowania wybraliśmy \emph{Eclipse} z dodatkiem \emph{PHP Development Tools}.
\subsubsection{SVN}
Ze względu na wymaganie pozafunkcjonalne wybraliśmy system zarządzania wersjami Subversion.
\subsubsection{Redmine}
Już od początku pracy korzystaliśmy z systemu zarządzania projektami \emph{Redmine}. To narzędzie bardzo przydatne w wymianie informacji pomiędzy członkami zespołu, integrujące się m.in. z repozytorium kodu.
\subsubsection{JasperReports}
Ze względu na wymaganie pozafunkcjonalne zdecydowaliśmy się skorzystać z mechanizmów raportowania oferowanych przez \emph{JasperReports}.
%MT
\subsubsection{JavaScript}
Formularze wymagające częstej interakcji z klientem, np. formularz umożliwiający tworzenie nowej ankiety, oraz funkcje związane z walidacją pól uzupełnianych przez klienta zostały napisane w \emph{JavaScript}. Obsługa strony po stronie użytkownika zapobiega frustracji, związanej z częstym przeładowywaniem całej strony.

\section{Ogólna struktura projektu}
\label{Chapter63}

Sekcja druga.

\section{Interfejs}
\label{Chapter64}

\subsection{Wprowadzenie}
Jedną z części pracy było zaprojektowanie graficznego interfejsu użytkownika. Głównym problemem jaki się pojawił, był wybór odpowiedniego narzędzia. Celem jaki postawiono, była maksymalna zgodność projektowanych elementów z różnymi wersjami \emph{Moodle} -- zarówno wcześniejszymi, jak i późniejszymi. Zdecydowano, aby starać się korzystać z gotowych interfejsów programowania aplikacji \emph{(API)} dostarczonych przez \emph{Moodle}, tj. \emph{Page API}, \emph{Form API}, oraz \emph{Access API}. Wszystkie interfejsy są napisane przy użyciu języka PHP -- są wykonywane po stronie serwera. Konieczne okazało się też wykonanie niektórych skryptów po stronie klienta. Dlatego w projekcie wykorzystano również język skryptowy \emph{Java Script}.

\section{Logika (back-end)}
\label{Chapter65}

Jednym z zadań w ramach pracy było zaprogramowanie odpowiedniej logiki biznesowej rozwiązującej zadania stawiane przed zaprojektowanym systemem. Najważniejszym zadaniem z perspektywy back-end'u jest interakcja z bazą danych. Poza tym system posiada: procesor zadań wykonywanych w tle oparty na \emph{cron}; moduł odpowiadający za komunikację z systemem uczelanianym \emph{ePoczta}; moduł logowania zdarzeń. W trakcie implementacji zdecydowaliśmy się nie tworzyć osobnego mechanizmu do przechowywania ustawień w bazie danych i skorzystaliśmy z istniejącego już w \emph{Moodle}. Jednym z wymagań pozafunkcjonalnych było wykorzystanie bazy danych \emph{PostgreSQL}. Platforma \emph{Moodle} korzysta z mechanizmu \emph{XMLDB}, co pozwala na ominięcie wielu problemów pojawiających się przy migracjach pomiędzy różnymi systemami baz danych. Niestety kosztem wykorzystania tego mechanizmu jest konieczność pracy z interfejsami programowania aplikacji dostarczanymi przez platformę \emph{Moodle}, m.in. \emph{Data manipulation API}.

\section{Powiązanie back-endu z interfejsem}
\label{Chapter66}

Dalsze opisy.