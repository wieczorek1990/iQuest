
\chapter{Opis procesów biznesowych}
\section{Aktorzy}
\section{Obiekty biznesowe}
\section{Biznesowe przypadki użycia}

Rozdział teoretyczny --- przegląd literatury naświetlający stan wiedzy na dany temat. 

Przegląd literatury naświetlający stan wiedzy na dany temat obejmuje rozdziały pisane na podstawie
literatury, której wykaz zamieszczany jest w części pracy pt.~\emph{Literatura} (lub inaczej \emph{Bibliografia},
\emph{Piśmiennictwo}). W tekście pracy muszą wystąpić odwołania do wszystkich pozycji zamieszczonych w
wykazie literatury. \textbf{Nie należy odnośników do literatury umieszczać w stopce strony.} Student jest
bezwzględnie zobowiązany do wskazywania źródeł pochodzenia informacji przedstawianych w pracy,
dotyczy to również rysunków, tabel, fragmentów kodu źródłowego programów itd. Należy także podać
adresy stron internetowych w przypadku źródeł pochodzących z Internetu.

\chapter{Wymagania funkcjonalne}
\section{Wstęp - diagram przypadków użycia}
\section{Administrator iQuest}
\subsection{UC1}
\subsection{UC2}
\section{Ankieter iQuest}
\subsection{UC3}
\subsection{UC4}
\section{Respondent iQuest}
\subsection{UC5}
\subsection{UC6}

\chapter{Wymagania pozafunkcjonalne}
\section{Wstęp}
\section{Charakterystyki oprogramowania}
\section{Wymagania pozafunkcjonalne i ich weryfikacja}

\chapter{Architektura systemu}
\section{Wstęp}
\section{Opis ogólny architektury}
\subsection{Perspektywa fizyczna}
\subsection{Perspektywa implementacyjna}
\section{Komunikacja pomiędzy modułami}
\section{Schemat bazy danych}
\section{Perspektywa kodu}