\chapter{Opis procesów biznesowych}
\label{Chapter2}

System \textit{iQuest}, będący przedmiotem niniejszej Pracy Dyplomowej, jest nie tylko projektem edukacyjnym, lecz również pełnoprawnym zadaniem biznesowym. Wykonywany dla Dziekana Wydziału Informatyki Politechniki Poznańskiej, traktowany jest dokładnie tak samo, jak w pełni profesjonalne zlecenia, z którymi jego uczestnikom przyjdzie się zmierzyć w przyszłości. Z tego względu, konieczna jest jego analiza w kontekście powiązanych procesów biznesowych.

\section{Aktorzy}
\label{Chapter21}

W systemie zdefiniowani są następujący aktorzy:
\begin{itemize}
\item System -- opisywany system, iQuest.
\item Administrator -- zarządza sprawami technicznymi, związanymi platformą Moodle. Funkcję mogą pełnić osoby mające podstawową wiedzę informatyczną, znający mechanizmy Moodle'a.
\item Administrator Bazy Danych -- zarządza sprawami technicznymi, związanymi z prawami do grup docelowych, ich tworzeniem i utrzymaniem. Funkcję mogą pełnić Pracownicy Uczelni\slash Dziekanatu oraz Administratorzy Systemów.
\item Ankieter -- tworzy ankiety, wskazuje grupy docelowe i rozsyła ankiety. Może też przeglądać raporty. Funkcję mogą pełnić: Prowadzący zajęcia, Pracownik Dziekanatu.
\item Respondent -- odpowiada na otrzymane ankiety. Funkcję mogą pełnić: Absolwenci, Studenci.
\end{itemize}

\section{Obiekty biznesowe}
\label{Chapter22}

W ramach systemu iQuest, zdefiniowane są cztery obiekty biznesowe. Mowa o Badaniu, Ankiecie, Grupie Docelowej i Raporcie.

\subsection{Badanie}

Jest to Ankieta wraz z wybranymi: grupą docelową i czasem trwania. Badanie determinują następujące atrybuty:

\begin{itemize}
\item Nazwa Badania,
\item Data rozpoczęcia,
\item Data zakończenia,
\item Okresowość,
\item Grupa docelowa,
\item Przypisana Ankieta.
\end{itemize}

\subsection{Ankieta}

Jest tworzona przez Ankieterów i wysyłana do Respondentów. Raz utworzona Ankieta zostaje zapisana w Katalogu Ankiet. Ankietę charakteryzują następujące atrybuty:

\begin{itemize}
\item Nazwa Ankiety,
\item Wstęp,
\item Podsumowanie,
\item Przypisane Pytania.
\end{itemize}

\subsection{Grupa Docelowa}

Grupa studentów lub absolwentów, do których skierowana jest ankieta. Atrybuty:

\begin{itemize}
\item Studenci\slash Absolwenci
\end{itemize}

\subsection{Raport}

Zebrane odpowiedzi z jednego lub z kilku badań. Może zawierać wykresy, zestawienia.


%\subsection{Katalog Ankiet}
%
%Katalog Ankiet zawiera zbiór wszystkich Ankiet dostępnych dla danego Ankietera iQuest. Ankiety mogą być z poziomu Katalogu Ankiet współdzielone, duplikowane, oglądane, edytowane i//lub usuwane, w zależności od aktualnego statusu. Dla przykładu, nowo-utworzoną Ankietę bez odpowiedzi można bez problemu usunąć lub edytować, podczas gdy taka, na którą udzielono już odpowiedzi, dostępna jest jedynie do odczytu, duplikacji i współdzielenia.

%\subsection{Pytanie}
%
%Pytanie jest elementarną jednostką Ankiety. Samo może składać się jedynie z nazwy (w przypadku pytań otwartych), lub nazwy i dostępnych odpowiedzi (dla Pytań zamkniętych). Pytanie w ogólności charakteryzują:

%\begin{itemize}
%\item Treść Pytania,
%\item Rodzaj Pytania,
%\item Dostępne odpowiedzi (dla Pytań zamkniętych).
%\end{itemize}

\pagebreak
\section{Biznesowe przypadki użycia}
\label{Chapter23}

Poniżej przedstawione zostały biznesowe przypadki użycia. Obejmują one dwa główne zagadnienia: zbieranie informacji oraz zarządzanie Grupami Docelowymi.

\subsection{BC01: Zbieranie informacji o Absolwentach}

\ucsection{BC01: Zbieranie informacji o Absolwentach}{Ankieter, Respondent}
{Ankieter chce ankietować Absolwentów}
{Ankieta, Raport}{\ucactions{
\ucaction{1. Ankieter tworzy Ankietę (UC01)}
\ucaction{2. Ankieter wybiera Absolwentów, do których chce rozesłać Ankietę (UC03)}
\ucaction{3. Ankieter uruchamia Ankietę (UC04)}
\ucaction{4. System powiadamia Respondentów o Ankiecie}
\ucaction{5. Respondent wypełnia Ankietę (UC05)}
\ucaction{6. Ankieter sprawdza podsumowanie Ankiety (UC06)}
}}
%{\ucextensions{
%\ucaction{3.A Opis sytuacji wyjątkowej 1}
%\ucaction{3.A.1 Pierwszy krok sytuacji wyjątkowej 1}
%\ucaction{3.B Opis sytuacji wyjątkowej 2}
%\ucaction{3.B.1 Pierwszy krok sytuacji wyjątkowej 2}
%\ucaction{3.B.2 Drugi krok sytuacji wyjątkowej 2}
%}}
{}

\subsection{BC02: Zbieranie informacji o Studentach}

\ucsection{BC02: Zbieranie informacji o Studentach}{Ankieter, Respondent}
{Ankieter chce ankietować Studentów}
{Ankieta, Raport}{\ucactions{
\ucaction{1. Ankieter tworzy Ankietę (UC01)}
\ucaction{2. Ankieter wybiera Studentów, do których chce rozesłać Ankietę (UC03)}
\ucaction{3. Ankieter uruchamia Ankietę (UC04)}
\ucaction{4. System powiadamia Respondentów o Ankiecie}
\ucaction{5. Respondent wypełnia Ankietę (UC05)}
\ucaction{6. Ankieter sprawdza podsumowanie Ankiety (UC06)}
}}
{}

\subsection{BC03: Zarządzanie Grupami Docelowymi}

\ucsection{BC03: Zarządzanie Grupami Docelowymi}{Administrator Bazy Danych}
{Ankieter chce ankietować Studentów}
{Ankieta, Raport}{\ucactions{
\ucaction{1. Ankieter zgłasza potrzebę stworzenia Grupy Docelowej Administratorowi Bazy Danych}
\ucaction{2. Administrator Bazy Danych podaje nazwę Grupy Docelowej, którą zamierza utworzyć}
\ucaction{3. Administrator Bazy Danych dodaje/usuwa członków Grupy Docelowej}
\ucaction{4. Administrator Bazy Danych potwierdza chęć stworzenia Grupy Docelowej}
\ucaction{5. System tworzy Grupę Docelową}
\ucaction{6. Ankieter może korzystać z Grupy Docelowej}
}}
{}