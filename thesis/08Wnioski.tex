\chapter{Zebrane doświadczenia}
\label{Chapter8}

%Tu wbijamy doświadczenia - czyli czegośmy się nauczyli. Każdy item powinien mieć do siebie min. jedno zdanie uzasadnienia / przykładu.
Podczas pracy zebraliśmy następujące doświadczenia bezpośrednio związane z zastosowanymi technologiami:
\begin{itemize}
\item Implementowanie testów jednostkowych z użyciem PHPUnit -- w celu kontroli poprawności pisanego kodu back-end'u w języku PHP,
\item Implementowanie logiki aplikacji w języku PHP na podstawie UML -- na podstawie diagramów UML oraz rozmów z architektem zaimplementowano kod back-end'u w języku PHP,
\item Implementowanie usług internetowych (protokół \emph SOAP) -- dla komunikacji z usługami uczelni wykorzystano klienty dostarczone przez Dział Rozwoju Oprogramowania w języku PHP, dla komunikacji z serwerem raportowania wykorzystano API \emph{external services} platformy Moodle (PHP) po stronie systemu iQuest oraz Apache Axis (Java) po stronie serwera JasperReports,
\item Implementowanie schematu bazy danych w formacie XMLDB na podstawie UML -- wykorzystano interfejs do zarządzania XMLDB dostarczany przez Moodle; drobne poprawki wprowadzano ręcznie w pliku XML definiującym schemat bazy danych,
\item Implementowanie modułów uwierzytelniania systemu Moodle -- na podstawie przykładowych oraz istniejących modułów uwierzytelniania stworzono moduł uwierzytelniania przez eKonto oraz moduł dla absolwentów,
\item Rozszerzanie funkcjonalności oprogramowania o otwartym kodzie źródłowym -- w celu spełnienia wymagań projektowych dodano m.in. okresowe sprawdzanie ważności sesji eKonto do mechanizmu zarządzania sesją Moodle, zmodyfikowano wygląd strony logowania, panelu użytkownika,
\item Konfigurowanie systemów operacyjnych Ubuntu i OpenSUSE -- W celu zdalnej konfiguracji wykorzystywano protokół SSH; Należało zainstalować i skonfigurować wymagane oprogramowanie (w tym Apache, cron, PostgreSQL, PHP, Check Point's Linux SNX), przygotować katalogi repozytoriów kodu,
\item Programowanie z użyciem Eclipse PDT -- wykorzystano zintegrowane środowisko programistyczne w celu zwiększenia produktywności programistów,
\item Konfigurowanie JasperServer -- dodano dialekt zapytań JoSQL, własne źródło danych (pomocne okazały się informacje z projektu \emph{Business Intelligence Server} dostępnego na Redmine),
\item Projektowanie raportów JasperReports -- obejmowało zaprojektowanie raportów w JasperReports Studio,
\item Korzystanie z klientów VPN (firmy CheckPoint) -- w celu uzyskania dostępu do sieci wewnętrznej zobligowano zespół do zestawiania połączenia VPN,
\item Korzystanie z systemu zarządzania projektami Redmine -- obejmuje zarządzanie zagadnieniami, bazą wiedzy, repozytorium, plikami, korzystanie z dostępnych metod komunikacji: komunikaty, forum, komentarze,
\item Korzystanie z systemów kontroli wersji SVN i Git -- SVN wykorzystano jak repozytorium kodu, repozytorium Git służyło zespołowi podczas pracy nad niniejszą pracą,
\item Pisanie dokumentacji technicznej oraz użytkownika,
\item Implementowanie modułów aktywności systemu Moodle,
\item Projektowanie testów akceptacyjnych z użyciem Selenium IDE,
\item Implementowanie formularzy z wykorzystaniem JavaScript,
\end{itemize}
%Tu kończymy wbijać doświadczenia.

%Poniżej wrzucamy jakiś opis do tych doświadczeń. Takie jakby podsumowanie ich w formie zdaniowej.
Bardzo istotna cześć doświadczeń związanych z projektem wiąże się z pracą zespołową.
Nauczyliśmy się, że dobrego członka zespołu wyróżnia: sumienność, punktualność, odpowiedzialność, dokładność, terminowość, prawdomówność, uczciwość, asertywność, zdolność kreatywnego myślenia, kultura bycia oraz komunikatywność. Praca w grupie nad dużym projektem wymaga dobrej koordynacji prac. Dostępność architekta i kierownika projektu znacząco ułatwiła pracę nad projektem. Dzięki podziałowi zadań, każdy członek zespołu mógł pogłębić swoje zainteresowania. Wraz z upływem każdy członek zespołu znalazł sobie zbiór typów zagadnień, które potrafił zrealizować szybciej i lepiej ze względu na wcześniej pozyskane doświadczenie. Uzyskane doświadczenia z pewnością przydadzą się nam w pracy zawodowej.
%I kończymy te doświadczenia...

%Tu zaczynamy bawić się we wnioski. Wymieniamy siebie i mówimy co nam się podobało, co nie, i w ogóle czego się spodziewaliśmy, jak to poszło, co sądzimy o SDS, itp. Można je tak jak jest niżej, wymieniać od myślników, można sobie wywalić pod nazwiskiem "itemize" i pobawić się w pisanie normalnego tekstu. Jeśli ktoś chce, może napisać co by zrobił, gdyby posiadając dzisiejszą wiedzę o tym projekcie i powiązanych z nim technologiach, znów miał wybrać tę pracę inżynierską. Jak by się to potoczyło? Co by zrobił, a czego nie? Jak mi starczy czasu to też taką rozkminę tu chyba dorzucę ;)
Wnioski:
\begin{description}
\item Krzysztof Borowiak:
\begin{itemize}
\item ...
\end{itemize}
\item Maciej Trojan
\begin{itemize}
\item Niewątpliwą zaletą pracy nad dużym projektem było zrozumienie jak ważną role odgrywa
dobre zaplanowanie i zaprojektowanie oprogramowania przed przystąpieniem do bezpośredniego
programowania.
\item Praca w zespole z określonym podziałem na role, umożliwiła szybkie rozwiązywanie
problemów napotkanych w trakcie wytwarzania oprogramowania. Ponadto podział ten wymusił
zagłębienie się w dziedzinach powiązanych z pełnioną rolą, co znacząco wpłynęło na
rozwijanie umiejętności.
\item Aby projekt został zrealizowany w wyznaczonym terminie, powstające zagadnienia należy
przydzielać do konkretnych osób oraz egzekwować ich wykonanie.
\end{itemize}
\item Krzysztof Urbaniak:
\begin{itemize}
\item ...
\end{itemize}
\item Łukasz Wieczorek:
\begin{itemize}
\item Zatwierdzaj swoje zmiany (ang. \emph{commit}) dopiero, gdy funkcjonalność, którą miałeś zaimplementować, jest kompletna i przetestowana jednostkowo oraz akceptacyjnie,
\item Przygotuj wcześniej logikę, z której będą korzystać programiści interfejsu, by nie opóźniać ich pracy,
\item Jeśli nie wiesz, jak zaimplementować daną funkcjonalność zapytaj architekta -- zwykle dysponuje on większą wiedzą ogólną,
\item Werbalizacja problemu bardzo często pomaga w jego rozwiązaniu,
\item Wcześniejsza refaktoryzacja znacznie ułatwia dalszą pracę nad kodem,
\item Wielokrotnie powtarzane fragmenty kodu należy jak najszybciej zrefaktoryzować, by uchronić się od poprawiania podobnych konstrukcji składniowych,
\item ,,Nie należy mnożyć bytów ponad potrzebę'',
\item Testy jednostkowe ułatwiają refaktoryzację kodu -- nawet jeśli ich utrzymywanie jest kosztowne, warto je pisać,
\item Podstawą efektywnej pracy jest kontrola czasu poświęcanego na realizację przydzielonych zdań,
\item Dobra komunikacja w zespole to podstawa sukcesu,
\end{itemize}
\end{description}
%I po robocie.
