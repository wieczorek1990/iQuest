\chapter{Zebrane doświadczenia}
\label{Chapter8}

Podczas pracy zebraliśmy następujące doświadczenia bezpośrednio związane z zastosowanymi technologiami:
\begin{itemize}
\item{Implementowanie testów jednostkowych z użyciem PHPUnit,}
\item{Implementowanie logiki aplikacji w języku PHP na podstawie UML,}
\item{Implementowanie formularzy z wykorzystaniem JavaScript,}
\item{Implementowanie usług internetowych (protokół \emph{SOAP}),}
\item{Implementowanie schematu bazy danych w formacie XMLDB na podstawie UML,}
\item{Implementowanie modułów uwierzytelniania systemu Moodle,}
\item{Implementowanie modułów aktywności systemu Moodle,}
\item{Rozszerzanie funkcjonalności oprogramowania o otwartym kodzie źródłowym, w celu spełnienia wymagań projektowych,}
\item{Programowanie z użyciem Eclipse PDT,}
\item{Projektowanie testów akceptacyjnych z użyciem Selenium IDE,}
\item{Konfigurowanie serwera Apache dla PHP,}
\item{Zdalne konfigurowanie systemów operacyjnych Ubuntu i OpenSUSE,}
\item{Konfigurowanie JasperServer,}
\item{Projektowanie raportów JasperReports,}
\item{Korzystanie z klientów VPN (firmy CheckPoint),}
\item{Korzystanie z systemu zarządzania projektami Redmine,}
\item{Korzystanie z systemów kontroli wersji SVN i Git,}
\item{Pisanie dokumentacji technicznej oraz użytkownika,}
\end{itemize}

Bardzo istotna cześć doświadczeń związanych z projektem wiąże się z pracą zespołową.
Nauczyliśmy się, że dobrego członka zespołu wyróżnia: sumienność, punktualność, odpowiedzialność, dokładność, terminowość, prawdomówność, uczciwość, asertywność, zdolność kreatywnego myślenia, kultura bycia oraz komunikatywność. Praca w grupie nad dużym projektem wymaga dobrej koordynacji prac. Dostępność architekta i kierownika projektu znacząco ułatwiła pracę nad projektem. Dzięki podziałowi zadań, każdy członek zespołu mógł pogłębić swoje zainteresowania. Wraz z upływem każdy członek zespołu znalazł sobie zbiór typów zagadnień, które potrafił zrealizować szybciej i lepiej ze względu na wcześniej pozyskane doświadczenie. Uzyskane doświadczenia z pewnością przydadzą się nam w pracy zawodowej.

Wnioski:
\begin{itemize}
\item Łukasz Wieczorek:
\begin{itemize}
\item Zatwierdzaj swoje zmiany (ang. \emph{commit}) dopiero, gdy funkcjonalność, którą miałeś zaimplementować, jest kompletna i przetestowana jednostkowo oraz akceptacyjnie,
\item Przygotuj wcześniej logikę, z której będą korzystać programiści interfejsu, by nie opóźniać ich pracy,
\item Jeśli nie wiesz, jak zaimplementować daną funkcjonalność zapytaj architekta -- zwykle dysponuje on większą wiedzą ogólną,
\item Werbalizacja problemu bardzo często pomaga w jego rozwiązaniu,
\item Wcześniejsza refaktoryzacja znacznie ułatwia dalszą pracę nad kodem,
\item Wielokrotnie powtarzane fragmenty kodu należy jak najszybciej zrefaktoryzować, by uchronić się od poprawiania podobnych konstrukcji składniowych,
\item ,,Nie należy mnożyć bytów ponad potrzebę'',
\item Testy jednostkowe ułatwiają refaktoryzację kodu -- nawet jeśli ich utrzymywanie jest kosztowne, warto je pisać,
\item Podstawą efektywnej pracy jest kontrola czasu poświęcanego na realizację przydzielonych zdań,
\item Dobra komunikacja w zespole to podstawa sukcesu,
\end{itemize}
\end{itemize}