\chapter{Zapewnianie jakości i konserwacja systemu}
\label{Chapter7}

\section{Testy i weryfikacja jakości oprogramowania}
\label{Chapter71}

\subsection{Testy jednostkowe}
\label{Chapter711}

%Tutaj piszemy o testach jednostkowych -- jak je zrobiliśmy oraz jakie były problemy, wnioski, które się pojawiły.

\subsection{Testy integracyjne}
\label{Chapter712}

%Podrozdział opcjonalny (bo nie wiem, czy wszyscy robią takie testy). Tutaj piszemy o testach integracyjnych -- jak je zrobiliśmy oraz jakie były problemy, wnioski, które się pojawiły.

\subsection{Testy akceptacyjne}
\label{Chapter713}

%Tutaj piszemy o testach akceptacyjnych -- jak je zrobiliśmy oraz jakie były problemy, wnioski, które się pojawiły. Tutaj zwykle pojawią się testy akceptacyjne przygotowane przez Waszych kierowników, ale także różne opisy, w tym -- jeśli jest to zrobione -- automatycznych testów. 

%Poniżej przedstawiono Manualne Testy Akceptacyjne:

%\matbegin{MAT01}{Nazwa testu pierwszego}
%\matpres
%\matpre{Warunek początkowy 1}
%\matpre{Warunek początkowy 2}
%\matsteps
%\matstep{1}{Polecenie pierwsze}{Odpowiedź systemu}
%\matstep{2}{Polecenie drugie}{Odpowiedź systemu}
%\matstep{3}{Polecenie trzecie}{Odpowiedź systemu}
%\matremark{Jakaś uwaga.}
%
%\matbegin{MAT02}{Nazwa testu drugiego}
%\matpres
%\matpre{Warunek początkowy 1}
%\matpre{Warunek początkowy 2}
%\matsteps
%\matstep{1}{Polecenie pierwsze}{Odpowiedź systemu}
%\matstep{2}{Polecenie drugie}{Odpowiedź systemu}
%\matstep{4}{Polecenie trzecie}{Odpowiedź systemu}
%\matremark{Brak}

\subsection{Inne metody zapewniania jakości}
\label{Chapter714}

%Jeśli testowaliście projekt w jakiś inny sposób, mieliście inne formy weryfikacji (np.~rzucanie komputerem o ściany czy z drugiego miejsca), to tutaj to opisujecie. Prawdopodobnie dobrze będzie opisać tutaj wszelkie testy dla interfejsu, jeśli np.~testowaliście użytkowników i sam system pod tym kątem.

\section{Sposób uruchomienia i~działania systemu}
\label{Chapter72}

%Tutaj można napisać jak przygotować system do działania i jak przeprowadzić konfigurację. Prawdę mówiąc, nie wiem, czy to powinno się tutaj znaleźć, ale być może warto, dlatego na wszelki wypadek to umieszczam.
