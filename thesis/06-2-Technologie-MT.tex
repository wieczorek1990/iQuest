\subsection{JavaScript}
\label{Chapter63b}

Formularze wymagające częstej interakcji z klientem, np. formularz umożliwiający tworzenie nowej ankiety, oraz funkcje związane z walidacją pól uzupełnianych przez klienta zostały napisane w \emph{JavaScript}. Obsługa strony po stronie użytkownika zapobiega frustracji, związanej z częstym przeładowywaniem całej strony, co ogranicza zarówno obciążenie łącza po stronie serwera, jak i po stronie klienta.