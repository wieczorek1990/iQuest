\subsection{\textit{JavaScript}}
\label{Chapter63b}

Formularze wymagające częstej interakcji z klientem, np. formularz umożliwiający tworzenie nowej ankiety, oraz funkcje związane z walidacją pól uzupełnianych przez klienta zostały napisane w \textit{JavaScript}. Obsługa strony po stronie klienta jest dla użytkownika niego znacznie wygodniejsza, gdyż nie wymaga częstego przeładowywania całej strony. Dodatkowo, ogranicza to obciążenie łącza zarówno po jego stronie, jak i po stronie serwera, co jest korzystne dla obu stron.