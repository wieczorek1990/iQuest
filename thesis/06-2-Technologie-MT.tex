\subsection{\textit{JavaScript}}
\label{Chapter63c}

Formularze wymagające częstej interakcji z klientem, np.~formularz umożliwiający tworzenie nowej ankiety oraz~funkcje związane z~walidacją pól uzupełnianych przez~klienta zostały napisane w~\textit{JavaScript}. Obsługa strony po~stronie klienta jest dla~użytkownika znacznie wygodniejsza, gdyż~nie~wymaga częstego przeładowywania całej strony. Dodatkowo, ogranicza to~obciążenie łącza zarówno w~lokalizacji serwera, jak i~klienta, co~jest korzystne dla~obu~stron.