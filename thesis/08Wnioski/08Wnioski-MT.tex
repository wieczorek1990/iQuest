%jako, że ma tu być podział chronologiczny, musicie pisać swoje części w osobnych plikach, a ja dokonam podziału chronologicznego i zmerge'uję to samodzielnie. Sądzę, że jest to znacznie bezpieczniejsze rozwiązanie, niż zabawa w merge'owanie jednego pliku u 4 osób w sposób rozproszony...

\subsection{Inicjalizacja bazy danych}
%label do uzupełnienia po merge'u
Moduł iQuest do działania wymaga rozszerzenie istniejącej bazy danych platformy \emph{Moodle} o dodatkowe tabele, przechowujące niezbędne dane wykorzystywane do spełnienia założonej funkcjonalności.

Do zaimportowania bazy danych przygotowanej przez architekta wykorzystano narzędzie, wbudowane w platformę \emph{Moodle}, \emph{XMLDB}, które gwarantuje bezobsługową instalację modułu w przyszłości. Jak się później okazało narzędzie to posiada błąd, który uniemożliwia zaimportowania kluczy obcych. Wymagało to od programistów ręcznego utworzenia wszystkich kluczy obcych, przewidzianych przez architekta.

\subsection{Instalacja modułu}
%label do uzupełnienia po merge'u
Postanowiono, że wraz z instalacją modułu powinień automatycznie tworzyć się odpowiedni \emph{kurs} związany jedynie z modułem \emph{iQuest}. Zmniejsza to potrzebny czas na przygotowanie platformy do użytku, oraz zapobiega pomyłkom związanych z ręcznym tworzeniem i konfiguracji \emph{kursu}. 

Niestety okazało się, że podejście to uniemożliwia instalację modułu jednocześnie z całą platformą \emph{Moodle}, ponieważ dodatkowe moduły instalowane są przed mechanizmami pozwalającymi na tworzenie \emph{kursu}. Doinstalowanie modułu do zainstalowanej platformy nie powoduje żadnych komplikacji.