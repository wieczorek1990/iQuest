\chapter{Zakończenie}
\label{Chapter9}

\section{Podsumowanie}
\label{Chapter91}

Rozwijanie istniejącego oprogramowania wymaga dużo większego nakładu pracy niż konstrukcja oprogramowania od podstaw. Wiele czasu poświęca się na analizę rozwiązań zastosowanych przez twórców rozwiązania bazowego. W trakcie implementacji pojawiają się problemy, których rozwiązania szuka się długimi godzinami; niektórych nie udaje się w ogóle rozwiązać, co z kolei prowadzi do modyfikacji czyjegoś oprogramowania (ang. \textit{hacking}). Praca nad oprogramowaniem, którego tworzenie zaczęło się przed ponad dziesięcioma laty, wymaga pracy z rozwiązaniami architektonicznymi, które dawno zostały już zarzucone (np. \textit{transaction script} porzucono na rzecz \textit{Model View Controller}). Dodatkową trudnością są zmieniające się lub niewspierane już interfejsy programowania aplikacji. Wymienione problemy dotyczą szczególnie obszernego oprogramowania, właśnie takiego z jakim przyszło nam pracować -- \textit{Moodle} to wg programu \textit{CLOC} ponad 2 miliony linii kodu. Uważamy, że decyzja architekta odnośnie rozwijania systemu \textit{Moodle} była błędna i nieprzemyślana, zwłaszcza w kontekście braku doświadczenia osób należących do zespołu programistów. Potwierdzenie naszych wniosków widzimy chociażby porównując wydajność pracy zespołu z innymi zespołami realizującymi projekty w ramach \textit{Software Development Studio}.

Możliwość współpracy z SDS była niewątpliwie zaletą. Po pierwsze, projekt, który należało wykonać był bardzo złożony. Dzięki dwóm dodatkowym osobom część pracy związaną z projektowaniem systemu, można było przerzucić, a dzięki temu, zająć się jedynie implementacją. Po drugie, wszystkie ewentualnie spory można było szybko rozwiązać, pytając o zdanie architekta, nie tracąc tym samym czasu na niepotrzebne kłótnie, które podejście jest lepsze. Po trzecie, udostępniony system \textit{Redmine} znacząco polepsza organizację pracy. Dzięki ,,śledzeniu zagadnień'' można lepiej organizować czas, co więcej, każdy wie, co ma robić. Nie bez znaczenia jest też możliwość korzystania z repozytorium \textit{SVN}. Wyposażony pokój jest udogodnieniem, które wpływa na lepszą komunikację w zespole. Korzyścią wynikającą z zadanej nam organizacji pracy jest konieczność trzymania się terminów, dzięki której nasz ostatni semestr studiów inżynierskich był dobrze zaplanowany pod względem równomiernego rozkładu pracochłonności.

Na dzień sporządzania niniejszej pracy dyplomowej, można więc śmiało stwierdzić, że wbrew pojawiającym się na drodze przeciwnościom, realizacja projektu \textit{iQuest} zakończone zostało sukcesem. Platforma \textit{Moodle}, w połączeniu z autorskimi wtyczkami składającymi się na system \textit{iQuest} zapewnia pełnię zleconej funkcjonalności. Autorzy niniejszej pracy wyrażają więc nadzieję, że utworzony przez nich system zostanie ciepło przyjęty przez studentów, absolwentów oraz prowadzących.

\section{Propozycja dalszych prac}
\label{Chapter92}

Podczas spotkania z reprezentantami Działu Rozwoju Oprogramowania poruszony został temat etykietowania (ang. tag) badań. Proponowany mechanizm z pewnością usprawniłby proces wyszukiwania badań. W obecnej wersji systemu zachętą dla studentów do wypełniania ankiet są materiały publikowane przez wykładowców uczelni, do których dostęp przyznawany jest użytkownikom, którzy w określonym czasie udzielili odpowiedzi w dowolnym badaniu. W przyszłości mechanizm ten można by rozbudować o wirtualną walutę służącą do ,,wykupowania'' dostępu do publikowanych materiałów. Dodatkowo, warto rozważyć wprowadzenie nowych typów pytań (np. skali jednokrotnego wyboru).