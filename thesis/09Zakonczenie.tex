\chapter{Zakończenie}
\label{Chapter9}

\section{Podsumowanie}
\label{Chapter91}

Realizacja projektu obejmującego utworzenie systemu \textit{iQuest} została zakończona sukcesem. Platforma \textit{Moodle}, w połączeniu z autorskimi wtyczkami składającymi się na system \textit{iQuest} zapewnia pełnię zleconej funkcjonalności, spełniając zarazem wymagania przedstawione w niniejszym dokumencie. \\

Dzięki wdrożeniu systemu \textit{iQuest}, Politechnika Poznańska uzyska dostęp do niezbędnego narzędzia prowadzenia badań wśród swoich absolwentów w rozumieniu ustawy ,,Prawo o Szkolnictwie Wyższym''. Jest to znaczący krok w stronę lepszego poznania potrzeb rynku, zarówno pracodawców, jak i samych studentów, pozwalający usprawnić mechanizmy zapewniania jakości kształcenia funkcjonujące na uczelni. \\

Udział w tak dużym i znaczącym dla Politechniki Poznańskiej projekcie przyczynił się do znaczącego rozwoju jego uczestników. Pozyskali oni wiele cennych doświadczeń i wyciągnęli znaczną liczbę najróżniejszych wniosków. Autorzy niniejszej pracy dyplomowej wyrażają więc nadzieję, że utworzony przez nich system zostanie pozytywnie przyjęty przez studentów, absolwentów oraz prowadzących.

\section{Propozycja dalszych prac}
\label{Chapter92}

Podczas spotkania z reprezentantami Działu Rozwoju Oprogramowania poruszony został temat etykietowania (ang. tag) badań. Proponowany mechanizm z pewnością usprawniłby proces wyszukiwania badań. W obecnej wersji systemu zachętą dla studentów do wypełniania ankiet są materiały publikowane przez wykładowców uczelni, do których dostęp przyznawany jest użytkownikom, którzy w określonym czasie udzielili odpowiedzi w dowolnym badaniu. W przyszłości mechanizm ten można zastąpić możliwością subskrypcji (wykupu) dostępu do publikowanych materiałów z wykorzystaniem wirtualnej waluty (np.~punktów za udział w badaniach).