\chapter{Zapewnianie jakości i konserwacja systemu}
\label{Chapter7}

\section{Testy i weryfikacja jakości oprogramowania}
\label{Chapter71}

\subsection{Wstęp}
\label{Chapter711}

Testy i weryfikacja jakości oprogramowania realizowana była na trzech poziomach: testów jednostkowych (dla logiki) oraz automatycznych i manualnych testów akceptacyjnych. Te ostatnie realizowane były nie tylko w zgodzie z dokumentem \textit{MAT}\cite{Redmine:ProjDocs}, ale też intuicyjnie, poprzez zwykłe korzystanie z systemu.

\subsection{Testy jednostkowe}
\label{Chapter712}

Testy jednostkowe zostały wykonane jako pierwsze i traktowane były z wysokim priorytetem. Realizowane były z użyciem klas PHPUnit, stosowanych powszechnie m.in.~przy testowaniu wtyczek do platformy Moodle. Testy te były kluczowe dla rozwoju logiki systemu iQuest. Uruchamiane są za pomocą przygotowanego skryptu \textit{tests.sh}, uruchamiającego je kolejno. Część testów operuje na systemie w trybie produkcyjnym, część na trybie testowym, obsługującym tzw. ,,atrapy'' (ang. \definicja{mock}), imitujące działanie systemów zewnętrznych. \\

Do stworzenia testów posłużyło środowisko Eclipse z dodatkiem PHP Development Tools. To pozwoliło na znaczące usprawnienie pracy przy realizacji tego zadania, względem stosowania zwykłych edytorów tekstowych, w tym tych poświęconych językowi PHP, jak np. gPHPEdit. \\

Przy realizacji pierwszego wydania, za testy jednostkowe w pełni odpowiadał jeden z członków zespołu programistów. W wydaniu drugim, rolę tę przejął programista realizujący logikę systemu, stosując zamiennie dwie techniki programowania, w tym jedną opartą o metodę TDU (ang. \definicja{Test-Driven Development} - rozwój w oparciu o testy). Pozwoliło to na znaczące zmniejszenie czasochłonności tego zadania.

\begin{figure}[H]
\begin{center}
\includegraphics[width=0.9\textwidth]{figures/lw/tests.pdf} 
\end{center}
\caption{Struktura klas testujących}
\label{fig:tests}
\end{figure}

\subsection{Testy akceptacyjne}
\label{Chapter713}

Testy akceptacyjne rozpatrywane są na dwóch poziomach: automatycznym i manualnym. Różnica polega jedynie na tym, kto (lub co) wykonuje test - komputer z odpowiednim oprogramowaniem, czy człowiek.

\subsubsection{MAT}
\label{Chapter7131}

Poniżej przedstawiono Manualne Testy Akceptacyjne:

\matbegin{MAT01}{Nazwa testu pierwszego}
\matpres
\matpre{Warunek początkowy 1}
\matpre{Warunek początkowy 2}
\matsteps
\matstep{1}{Polecenie pierwsze}{Odpowiedź systemu}
\matstep{2}{Polecenie drugie}{Odpowiedź systemu}
\matstep{3}{Polecenie trzecie}{Odpowiedź systemu}
\matremark{Jakaś uwaga.}

\matbegin{MAT02}{Nazwa testu drugiego}
\matpres
\matpre{Warunek początkowy 1}
\matpre{Warunek początkowy 2}
\matsteps
\matstep{1}{Polecenie pierwsze}{Odpowiedź systemu}
\matstep{2}{Polecenie drugie}{Odpowiedź systemu}
\matstep{4}{Polecenie trzecie}{Odpowiedź systemu}
\matremark{Brak}

\subsubsection{AAT}
\label{Chapter7131}

Automatyczne testy akceptacyjne realizowano w zgodzie z testami manualnymi i operując na tych samych wytycznych. Nagrywanie testów odbywało się za pomocą oprogramowania Selenium IDE, udostępnianego w formie rozszerzenia dla przeglądarki Mozilla Firefox. Pierwotnie, testy były konwertowane do języka Java, celem uruchamiania ich z poziomu języka Java, oferującego sporą swobodę przy projektowaniu warunków początkowych i końcowych dla testów. Problemy, jakie wynikały z takiego działania, opisane zostały w rozdziale \ref{Chapter6}. Na ich podstawie zdecydowano o pozostaniu w obrębie Selenium IDE, które samo w sobie również umożliwia automatyzację w wysokim stopniu. Aby dodatkowo ułatwić zadanie, przygotowany został skrypt ustawiający bazę danych w stan początkowy dla realizacji testów.

\subsection{Inne metody zapewniania jakości}
\label{Chapter714}

Celem zapewnienia jak najwyższej jakości oprogramowania, było ono testowane -- w kontrolowanych warunkach -- na różnorakich maszynach. Co prawda, lokalne serwery developerskie pracowały w oparciu o system Ubuntu 12.04 LTS, z serwerem Apache i systemem zarządzania bazą danych PostgreSQL, jednak maszyny klienckie były już znacznie bardziej różnorodne.

System przetestowano na zbiorze komputerów stacjonarnych klasy PC oraz równoważnych laptopów, z użyciem systemów Windows i Linux. Realizowano je także na laptopach i netbookach klasy PC oraz Macintosh, również pod kontrolą systemów Windows i Linux, oraz - w przypadku tej ostatniej klasy - MacOS. Ponadto, przetestowano trzy główne platformy mobilne (Android, iOS, WP7.x) poprzez dostęp do systemu z poziomu telefonów komórkowych.

Wspomniane wyżej maszyny pracowały pod kontrolą następujących systemów operacyjnych:
\begin{itemize}
\item{Windows XP Professional SP3 32-bit + Internet Explorer 7}
\item{Windows Vista Business SP2 32-bit + Internet Explorer 8}
\item{Windows 7 Professional SP1 64-bit + Mozilla Firefox + Google Chrome}
\item{Windows 8 Professional 64-bit + Internet Explorer 10 + Google Chrome}
\item{Ubuntu 12.04 64-bit + Mozilla Firefox + Google Chrome}
\item{Mac OS X + Apple Safari}
\item{Google Android 2.2, 2.3, 4.0, 4.1}
\item{iOS (iPhone 4S)}
\item{Windows Phone 7.5 (Nokia Lumia 710)}
\end{itemize}

Na podstawie powyższych testów, utworzono dwa raporty: ,,wygląd i działanie systemu iQuest na platformach mobilnych'' oraz ,,wygląd i działanie systemu iQuest w różnych konfiguracjach system-przeglądarka'', udostępnione w ramach systemu zarządzania projektem\cite{Redmine:ProjDocs}. Wynika z nich, że system iQuest posiada bardzo wysoki współczynnik przenośności.