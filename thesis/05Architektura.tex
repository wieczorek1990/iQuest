\chapter{Architektura systemu}
\label{Chapter5}

\section{Wstęp}
\label{Chapter51}

{\color{red}Wprowadzenie, bardzo ogólny opis.}

\section{Opis ogólny architektury -- Marketecture}
\label{Chapter52}

{\color{red}Tutaj rysunek i opis marketecture.}

%\section{Analiza SWOT}
%\label{Chapter53}
%
%My tego nie mieliśmy, ale chyba warto -- tutaj analiza SWOT przyjętego podejścia architektonicznego.
%
\section{Perspektywy architektoniczne}
\label{Chapter54}

\subsection{Perspektywa fizyczna}

{\color{red}Rysunek wraz z opisem.}

\subsection{Perspektywa logiczna}

{\color{red}Rysunek wraz z opisem.}

\subsection{Perspektywa implemetancyjna}

{\color{red}Rysunek wraz z opisem. Można tutaj też umieścić perspektywę kodu.}

\subsection{Perspektywa procesu (równoległości)}

{\color{red}Rysunek wraz z opisem.}

\section{Decyzje projektowe}
\label{Chapter55}

{\color{red}Tutaj piszemy o decyzjach projektowych, związkach pomiędzy nimi oraz innych związanych sprawach.}

\section{Wykorzystane technologie}
\label{Chapter56}

{\color{red}Opis wykorzystywanych technologii (COTS).}

\section{Schemat bazy danych}
\label{Chapter57}

{\color{red}Schemat bazy danych (tu lub w osobnym dodatku).}