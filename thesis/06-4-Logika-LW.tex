\section{Logika (back-end)}
\label{Chapter66}

Jednym z zadań w ramach pracy było zaprogramowanie odpowiedniej logiki biznesowej rozwiązującej zadania stawiane przed zaprojektowanym systemem. Najważniejszym zadaniem z perspektywy back-end'u jest interakcja z bazą danych. Poza tym system posiada: procesor zadań wykonywanych w tle oparty na \emph{cron}; moduł odpowiadający za komunikację z systemem uczelanianym \emph{ePoczta}; moduł logowania zdarzeń. W trakcie implementacji zdecydowano się nie tworzyć osobnego mechanizmu do przechowywania ustawień w bazie danych i skorzystaliśmy z istniejącego już w \emph{Moodle}. Jednym z wymagań pozafunkcjonalnych było wykorzystanie bazy danych \emph{PostgreSQL}. Platforma \emph{Moodle} korzysta z mechanizmu \emph{XMLDB}, co pozwala na ominięcie wielu problemów pojawiających się przy migracjach pomiędzy różnymi systemami baz danych. Niestety kosztem wykorzystania tego mechanizmu jest konieczność pracy z interfejsami programowania aplikacji dostarczanymi przez platformę \emph{Moodle}, m.in. \emph{Data manipulation API} (zarządzanie danymi), \emph{Access API} (zarządzanie dostępem, rolami, prawami).\\