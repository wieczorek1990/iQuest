% Szkielet dla pracy inżynierskiej pisanej w języku polskim.

\documentclass[polish,bachelor,a4paper,oneside]{ppfcmthesis}

\usepackage[utf8]{inputenc}
\usepackage[T1]{fontenc}
\usepackage{usecase}
\usepackage{float}
\usepackage[table]{xcolor}
\usepackage{mat}
\usepackage{verbatim}
\usepackage{pdflscape}
\usepackage{longtable}
\usepackage{multirow}
\usepackage{pdfpages}

\sloppy
%Zamiast wjazdu na prawy margines, będzie rozrzedzał linię (jak w Wordzie).

\hyphenation{a-ktu-a-li-za-cją}
\hyphenation{a-na-li-za}
\hyphenation{a-na-li-zę}
\hyphenation{an-kiet}
\hyphenation{an-kie-to-wa-ne-go}
\hyphenation{an-kie-ty}
\hyphenation{fun-kcjo-nal-ne}
\hyphenation{Jasper-Reports}
\hyphenation{nie-ste-ty}
\hyphenation{o-sob-no}
\hyphenation{po-za-fun-kcjo-nal-ne}
\hyphenation{pra-co-wni-ków}
\hyphenation{pra-wi-dło-wa}
\hyphenation{re-a-li-za-cji}
\hyphenation{redmine}
\hyphenation{roz-sze-rza-ją-cych}
\hyphenation{ser-we-ra}
\hyphenation{speł-nie-nia}
\hyphenation{spra-wia-ły}
\hyphenation{u-czel-nia-nych}
\hyphenation{u-miesz-czo-no}
\hyphenation{u-praw-nie-nia-mi}
\hyphenation{u-two-rze-nie}
\hyphenation{u-wie-rzy-tel-nia-nia}
\hyphenation{u-żyt-ko-wni-ka}
\hyphenation{u-żyt-ko-wni-ków}
\hyphenation{wy-ge-ne-ro-wa-ne}
\hyphenation{za-ga-dnie-nia-mi}
\hyphenation{za-miesz-czo-no}
\hyphenation{ze-sta-wień}
\hyphenation{zmia-ny}
\hyphenation{zmien-nej}
\hyphenation{znaj-du-ją}

% Authors of the thesis here. Separate them with \and
\author{%
   Krzysztof Marian Borowiak \album{94269} \and 
   Maciej Trojan \album{94378} \and 
   Krzysztof Urbaniak \album{94381} \and 
   Łukasz Wieczorek \album{94385}}
\title{iQuest -- system rozszerzonej obsługi ankiet~studenckich}                   % Note how we protect the final title phrase from breaking
\ppsupervisor{dr inż.~Bartosz Walter} % Your supervisor comes here.
\ppyear{2013}                                         % Year of final submission (not graduation!)

\begin{document}

% Front matter starts here
\frontmatter\pagestyle{empty}%
\maketitle\cleardoublepage%

% Blank info page for "karta dyplomowa"
\thispagestyle{empty}\vspace*{\fill}%
\begin{center}Tutaj przychodzi karta pracy dyplomowej;\\oryginał wstawiamy do wersji dla archiwum PP, w pozostałych kopiach wstawiamy ksero.\end{center}%
\vfill\cleardoublepage%

% Table of contents.
\pagenumbering{Roman}\pagestyle{ppfcmthesis}%
\tableofcontents* \cleardoublepage%

% Main content of your thesis starts here.
\mainmatter%
\chapter{Wprowadzenie}
\label{Chapter1}

\section{Opis problemu i koncepcja jego rozwiązania (ToDo)}
\label{Chapter11}

Treść testowa. \textit{TexTit treść testowa.}. Treść testowa.

AKTUALNA WERSJA OPARTA NA: Project Brief

Context
The customer of this project is the Dean of the Faculty of Computing Science at the Poznan University of Technology. The project is going to be developed for the University, which is a middle-sized, Polish higher education institution. 
Problems and Their Impact
Problems:
The University doesn't have statistics on its graduates' careers. It also doesn't have a feedback from the current and former students on their satisfaction of the services provided by the University.
Impact of the problems:
Because of the above problems, the University has a limited ability to measure and improve the quality of provided education services. If the quality is not satisfying, the University's prestige will decrease and fewer students will be willing to study there. This will cause the University to receive less funding from government and other organizations. 
Outline of the Solution
The solution is to conduct surveys among the University students and graduates. In these surveys they will be asked to provide an opinion on the studies and to give some information on their professional life.
There will be a web application developed which will be the platform used to carry out surveys. It will allow reaching various target groups (current students, former students, etc.), verifying respondent's identity, collecting, storing and analyzing answers, and generating reports. The solution will also provide incentives to encourage potential participants to take part in the surveys. 
Additionally, the application will allow to make anonymous surveys so in reports there will be no personal information about respondent. Moreover there will be some extra articles available only for those who complete the survey and kind of CMS to put articles on line. 
Business Constraints
Application has to be done until the end of February/March 2013. It’s a time when bachelor’s degree students need to finish their project before final exam. They are our developers. Budget is not known yet. 
Preliminary Risk Assessment
Priority will be high enough to get appropriate support from client-side. Also deadline seems to be suitable. The budget is still not known. Client did not specify a technology which has to be used in this project. There are some well-known technologies which may contribute our goals. Moreover, we hope that we will find easily a group of developers which can handle one of these technologies.
Acceptance Criteria
The most important quality criteria for the project are usability. Application should be useful for users and provide all functions at the easiest way to don't discourage end-user. System should allow as many users as possible to use it at same time. It is of high importance that the deadline cannot be met.
Proposed Staff
Jerzy Nawrocki – Jerzy.Nawrocki@cs.put.poznan.pl – Executive
Sylwia Kopczyńska – Sylwia.Kopczynska@cs.put.poznan.pl – Senior Supplier
Michał Witczak – mich.witczak@gmail.com  – Project Manager
Błażej Matuszyk – blasoft@live.com – Architect / Quality Assurance
Marcin Domański – marcaj13@gmail.com – Analyst
Additional information
Detailed problem description:
University is currently using its own system to conduct surveys. Unfortunately, it has many disadvantages and students are not willing to use it (e.g. Users have problems with choosing the right tutor of their courses, sometimes they just cannot find them on the list). Moreover, that application doesn’t provide features required by the customer. It doesn’t support short surveys after lectures – surveys intended for graduates. Overall application should be more flexible to assist in carrying out many types of polls and surveys.

\section{Cele projektu}
Zbudowanie systemu umożliwiającego łatwe przeprowadzanie ankiet wśród studentów oraz absolwentów uczelni. System powinien:
\begin{itemize}
\item{współpracować z innymi systemami funkcjonującymi na uczelni (np. Sokrates)}
\item{oferować dużą elastyczność przy definiowaniu ankiet oraz grup respondentów}
\item{oferować rozbudowane możliwości raportowania}
\end{itemize}

%Oto przykład tekstu, do którego istnieje adnotacja na dole strony\footnote{To jest właśnie odnośnik.}. Do bibliografii odnosimy się w taki sposób \cite{Hirsch:HIR05}. Dla oznaczenia wszelkich terminów używany znacznika ,,definicja'': \definicja{Termin z definicji}. Natomiast, jeśli chcemy odnieść się do innego miejsca w dokumencie (które jest oznaczone pewną etykietą): \ref{Chapter12}. Łamanie strony odbywa się poprzez znacznik ,,pagebreak''. Po skrótach z kropką warto używać tyldy (np.~tak). Link podajemy poprzez znacznik ,,url'' (\url{www.google.pl}).

%Do takiego ,,zwykłego'' myślnika używamy podwójnego znaku ,,-'', a pojedynczego do łączenia bezpośrednio dwóch wyrażeń. Potrójny stosujemy, jak chcemy gdzieś pokazać brak informacji.

%Kompilację tego dokumentu najwygodniej zacząć od zainstalowania MikTeXa oraz przygotowana pliku render.bat, którego treść przedstawia się następująco:

%\begin{verbatim}
%@echo off
%
%pdflatex thesis-bachelor-polski.tex 
%bibtex   thesis-bachelor-polski
%pdflatex thesis-bachelor-polski.tex 
%pdflatex thesis-bachelor-polski.tex 
%
%del *.aux *.bak *.log *.blg *.bbl *.toc *.out
%\end{verbatim}
%
%Potrójna kompilacja to nie wymysł szalonego programisty, który uważa, że ,,wtedy lepiej się skompiluje'', ale rzeczywista potrzeba wynikająca z poprawnego powiązania ze sobą wszystkich odwołań w dokumencie. Po uruchomieniu takiego pliku .bat (jeśli wszystko pójdzie dobrze), powinien się utworzyć plik .pdf. Nie poleca się uruchamiania skryptu, gdy mamy otwartą aktualną wersję pliku .pdf. Przy pierwszym uruchomieniu, MikTeX prawdopodobnie będzie prosił o pozwolenie na pobranie wymaganych pakietów. UWAGA! Podczas kompilacji być może trzeba będzie trzy razy potwierdzać zaistnienie jakiegoś błędu związanego ze znacznikiem ,,ppcolophon'' -- jak potwierdzicie to Enterem, to kompilacja pójdzie dalej i wszystko skończy się szczęśliwie. To wynika z jakiejś konstrukcji w szablonie udostępnianym przez uczelnię.
%
%W tym podrozdziale generalnie znajduje się opis problemu, jaki doprowadził do powstania koncepcji Waszego systemu. Umieszcza się tu także ogólny opis tego, co Wasz system powinien robić, jakie ma zastosowanie (fachowo to się nazywa ,,problem i jego implikacje (znaczenie)'' oraz ,,cel biznesowy''). 
%
%\section{Omówienie pracy}
%\label{Chapter12}
%
%Tutaj piszemy o celu samego dokumentu oraz ewentualnych konwencjach, jakie przyjęliśmy podczas opisywania różnych rzeczy. Dla systemu BIS-2 ten fragment wyglądał tak:
%
%\textit{Niniejszy dokument opisuje system System informacji bibliometrycznej (ang.~\definicja{Bibliometric Information System}) zwanego dalej BIS-2 (dla odróżnienia od wersji pilotażowej), który realizuje koncepcję przytoczoną w~punkcie \ref{Chapter11}. Praca ma formę dokumentacji technicznej dla osób, które zamierzają wdrażać i~obsługiwać system, ale także opisuje ideę stojącą za implementacją poszczególnych części projektu z~odniesieniami do literatury. Ponadto, jest to również praca dyplomowa inżynierska, zatem jej odbiorcami są także członkowie komisji egzaminacyjnej.}
%
%Następnie musicie napisać, co zawierają poszczególne rozdziały. U nas wyglądało to tak:
%
%\textit{W rozdziale \ref{Chapter2}. rozszerzono koncepcję projektu o~przedstawienie aktorów oraz obiektów biznesowych, a~także przybliżono scenariusze operacyjne w~postaci przypadków użycia. Specyfikację wymagań oprogramowania przedstawiono w rozdziałach \ref{Chapter3}.~(funkcjonalne) oraz w~\ref{Chapter4}.~(pozafunkcjonalne). W~rozdziale \ref{Chapter5}.~omówiono architekturę systemu na wyższym poziomie abstrakcji. Uzasadnienie wyboru technologii, opis implementacji i~koncepcji znajduje się w~rozdziale \ref{Chapter6}. Informacje dotyczące zapewniania jakości zostały opisane w~rozdziale \ref{Chapter7}. W~rozdziale \ref{Chapter8}.~umieszczono opis zarządzania wersjami i~sposobu pracy nad projektem. Zebrane wnioski i~doświadczenia zawarto w~rozdziale \ref{Chapter9}. W~dodatkach opisano wkład poszczególnych osób i~informacje uzupełniające. Ostatnią część dokumentu stanowi wykaz literatury przybliżający zagadnienia opisane w~pracy.}
\chapter{Opis procesów biznesowych}
\label{Chapter2}

System \textit{iQuest}, będący przedmiotem niniejszej Pracy Dyplomowej, jest nie tylko projektem edukacyjnym, lecz również pełnoprawnym zadaniem biznesowym. Wykonywany dla Dziekana Wydziału Informatyki Politechniki Poznańskiej, traktowany jest dokładnie tak samo, jak w pełni profesjonalne zlecenia, z którymi jego uczestnikom przyjdzie się zmierzyć w przyszłości. Z tego względu, konieczna jest jego analiza w kontekście powiązanych procesów biznesowych.

\section{Aktorzy}
\label{Chapter21}

W systemie zdefiniowani są następujący aktorzy:
\begin{itemize}
\item System -- opisywany system, iQuest.
\item Administrator -- zarządza sprawami technicznymi, związanymi platformą Moodle. Funkcję mogą pełnić osoby mające podstawową wiedzę informatyczną, znający mechanizmy Moodle'a.
\item Administrator Bazy Danych -- zarządza sprawami technicznymi, związanymi z prawami do grup docelowych, ich tworzeniem i utrzymaniem. Funkcję mogą pełnić Pracownicy Uczelni\slash Dziekanatu oraz Administratorzy Systemów.
\item Ankieter -- tworzy ankiety, wskazuje grupy docelowe i rozsyła ankiety. Może też przeglądać raporty. Funkcję mogą pełnić: Prowadzący zajęcia, Pracownik Dziekanatu.
\item Respondent -- odpowiada na otrzymane ankiety. Funkcję mogą pełnić: Absolwenci, Studenci.
\end{itemize}

\section{Obiekty biznesowe}
\label{Chapter22}

W ramach systemu iQuest, zdefiniowane są cztery obiekty biznesowe. Mowa o Badaniu, Ankiecie, Grupie Docelowej i Raporcie.

\subsection{Badanie}
\label{Chapter221}

Jest to Ankieta wraz z wybranymi: grupą docelową i czasem trwania. Badanie determinują następujące atrybuty:

\begin{itemize}
\item Nazwa Badania,
\item Data rozpoczęcia,
\item Data zakończenia,
\item Okresowość,
\item Grupa docelowa,
\item Przypisana Ankieta.
\end{itemize}

\subsection{Ankieta}
\label{Chapter222}

Jest tworzona przez Ankieterów i wysyłana do Respondentów. Raz utworzona Ankieta zostaje zapisana w Katalogu Ankiet. Ankietę charakteryzują następujące atrybuty:

\begin{itemize}
\item Nazwa Ankiety,
\item Wstęp,
\item Podsumowanie,
\item Przypisane Pytania.
\end{itemize}

\subsection{Grupa Docelowa}
\label{Chapter223}

Grupa studentów lub absolwentów, do których skierowana jest ankieta. Atrybuty:

\begin{itemize}
\item Studenci\slash Absolwenci
\end{itemize}

\subsection{Raport}
\label{Chapter224}

Zebrane odpowiedzi z jednego lub z kilku badań. Może zawierać wykresy, zestawienia.


%\subsection{Katalog Ankiet}
%
%Katalog Ankiet zawiera zbiór wszystkich Ankiet dostępnych dla danego Ankietera iQuest. Ankiety mogą być z poziomu Katalogu Ankiet współdzielone, duplikowane, oglądane, edytowane i//lub usuwane, w zależności od aktualnego statusu. Dla przykładu, nowo-utworzoną Ankietę bez odpowiedzi można bez problemu usunąć lub edytować, podczas gdy taka, na którą udzielono już odpowiedzi, dostępna jest jedynie do odczytu, duplikacji i współdzielenia.

%\subsection{Pytanie}
%
%Pytanie jest elementarną jednostką Ankiety. Samo może składać się jedynie z nazwy (w przypadku pytań otwartych), lub nazwy i dostępnych odpowiedzi (dla Pytań zamkniętych). Pytanie w ogólności charakteryzują:

%\begin{itemize}
%\item Treść Pytania,
%\item Rodzaj Pytania,
%\item Dostępne odpowiedzi (dla Pytań zamkniętych).
%\end{itemize}

\pagebreak
\section{Biznesowe przypadki użycia}
\label{Chapter23}

Poniżej przedstawione zostały biznesowe przypadki użycia. Obejmują one dwa główne zagadnienia: zbieranie informacji oraz zarządzanie Grupami Docelowymi.

\subsection{BC01: Zbieranie informacji o Absolwentach}
\label{Chapter231}

\ucsection{BC01: Zbieranie informacji o Absolwentach}{Ankieter, Respondent}
{Ankieter chce ankietować Absolwentów}
{Ankieta, Raport}{\ucactions{
\ucaction{1. Ankieter tworzy Ankietę (UC01)}
\ucaction{2. Ankieter wybiera Absolwentów, do których chce rozesłać Ankietę (UC03)}
\ucaction{3. Ankieter uruchamia Ankietę (UC04)}
\ucaction{4. System powiadamia Respondentów o Ankiecie}
\ucaction{5. Respondent wypełnia Ankietę (UC05)}
\ucaction{6. Ankieter sprawdza podsumowanie Ankiety (UC06)}
}}
%{\ucextensions{
%\ucaction{3.A Opis sytuacji wyjątkowej 1}
%\ucaction{3.A.1 Pierwszy krok sytuacji wyjątkowej 1}
%\ucaction{3.B Opis sytuacji wyjątkowej 2}
%\ucaction{3.B.1 Pierwszy krok sytuacji wyjątkowej 2}
%\ucaction{3.B.2 Drugi krok sytuacji wyjątkowej 2}
%}}
{}

\subsection{BC02: Zbieranie informacji o Studentach}
\label{Chapter232}

\ucsection{BC02: Zbieranie informacji o Studentach}{Ankieter, Respondent}
{Ankieter chce ankietować Studentów}
{Ankieta, Raport}{\ucactions{
\ucaction{1. Ankieter tworzy Ankietę (UC01)}
\ucaction{2. Ankieter wybiera Studentów, do których chce rozesłać Ankietę (UC03)}
\ucaction{3. Ankieter uruchamia Ankietę (UC04)}
\ucaction{4. System powiadamia Respondentów o Ankiecie}
\ucaction{5. Respondent wypełnia Ankietę (UC05)}
\ucaction{6. Ankieter sprawdza podsumowanie Ankiety (UC06)}
}}
{}

\subsection{BC03: Zarządzanie Grupami Docelowymi}
\label{Chapter233}

\ucsection{BC03: Zarządzanie Grupami Docelowymi}{Administrator Bazy Danych}
{Ankieter chce ankietować Studentów}
{Ankieta, Raport}{\ucactions{
\ucaction{1. Ankieter zgłasza potrzebę stworzenia Grupy Docelowej Administratorowi Bazy Danych}
\ucaction{2. Administrator Bazy Danych podaje nazwę Grupy Docelowej, którą zamierza utworzyć}
\ucaction{3. Administrator Bazy Danych dodaje/usuwa członków Grupy Docelowej}
\ucaction{4. Administrator Bazy Danych potwierdza chęć stworzenia Grupy Docelowej}
\ucaction{5. System tworzy Grupę Docelową}
\ucaction{6. Ankieter może korzystać z Grupy Docelowej}
}}
{}
\chapter{Wymagania funkcjonalne}
\label{Chapter3}

\section{Wstęp -- diagram przypadków użycia}

\begin{figure}[th]
\centering\includegraphics[width=15cm]{figures/UseCaseView}
\caption{Diagram przypadków użycia}\label{rys:usecase}
\end{figure}

Na rysunku 3.1 przedstawiono diagram przypadków użycia. W ramach Systemu udostępniane są różne funkcje, możliwe do wykonania przez różnych aktorów. Dla przykładu, Ankieter może tworzyć Badania i analizować ich Statystyki oraz Raporty, podczas gdy Respondent może odpowiadać w Ankietach w ramach skierowanych do niego Badań; Administrator zarządza przydzielaniem ról; Administrator Bazy Danych zajmuje zarządza grupami Respondentów i przydzielaniem praw i zezwoleń.

\section{Ankieter}

Poniżej przedstawiono przypadki użycia dotyczące Ankietera.

\subsection{UC01: Stworzenie Ankiety}

\ucsection{UC01: Stworzenie Ankiety}{Ankieter}
{Ankieter jest zalogowany w Systemie i chce utworzyć Ankietę}
{}{\ucactions{
\ucaction{1. Ankieter wpisuje atrybuty Ankiety: nazwę Ankiety, wstęp, podsumowanie}
\ucaction{2. System prezentuje stronę umożliwiającą dodawanie pytań}
\ucaction{3. Ankieter wybiera typ pytania}
\ucaction{4. Ankieter wpisuje treść pytania}
\ucaction{5. Ankieter podaje możliwe odpowiedzi}
\ucaction{6. System prezentuje podsumowanie ankiety}
\ucaction{7. Ankieter akceptuje ankietę}
\ucaction{8. System zapisuje ankietę w Katalogu Ankiet Ankietera}
}}
{\ucextensions{
\ucaction{4.A Typ pytania: pytanie otwarte}
\ucaction{4.A.1 Ankieter pomija krok 5.}
\ucaction{5.A Ankieter chce dodać kolejne pytanie}
\ucaction{5.A.1 Powrót do kroku 3.}
}}
{}

\subsection{UC02: Edycja Ankiety}

\ucsection{UC02: Edycja Ankiety}{Ankieter}
{
1. Ankieta znajduje się w Systemie i jest dostępna dla Ankietera \\
2. Ankieta nie jest częścią czynnego Badania \\
3. Ankieter jest zalogowany w Systemie i chce zmodyfikować istniejącą Ankietę
}
{}{\ucactions{
\ucaction{1. Ankieter wybiera Ankietę do modyfikacji}
\ucaction{2. System prezentuje wskazaną Ankietę}
\ucaction{3. Ankieter wybiera pytanie do edycji/usunięcia}
\ucaction{4. System prezentuje pytanie z możliwością edycji}
\ucaction{5. Ankieter edytuje/usuwa pytanie}
\ucaction{6. Ankieter potwierdza chęć zapisu zmienionej Ankiety}
\ucaction{7. System zapisuje zmienioną Ankietę}
}}
{\ucextensions{
\ucaction{5.A. Edycja możliwych odpowiedzi do pytań}
\ucaction{5.A.1 Ankieter edytuje możliwe odpowiedzi do pytań}
}}
{}

\subsection{UC03: Wybór Grupy Docelowej}

\ucsection{UC03: Wybór Grupy Docelowej}{Ankieter, Administrator Bazy Danych}
{
1. Ankieta znajduje się w Systemie i jest dostępna dla Ankietera \\
2. Grupa Docelowa znajduje się w Systemie i jest dostępna dla Ankietera \\
3. Ankieter jest zalogowany w Systemie i chce wybrać Grupę Docelową
}
{}{\ucactions{
\ucaction{1. Ankieter wybiera przycisk ,,Utwórz badanie''}
\ucaction{2. System prezentuje listę Grup Docelowych, do których Ankieter ma uprawnienia}
\ucaction{3. Ankieter wybiera Grupy Docelowe dla danej Ankiety}
\ucaction{4. Ankieter wybiera typ powiadamiania Respondentów}
\ucaction{5. Ankieter akceptuje powiązanie Grup Docelowych z Ankietą}
}}
{\ucextensions{
\ucaction{3.A. Zawężenie Grupy Docelowej}
\ucaction{3.A.1 Ankieter wybiera członków Grupy Docelowej, do której ma być skierowana Ankieta.}
\ucaction{3.A.2 Powrót do kroku 4.}
\ucaction{3.B. Grupa Docelowa poszukiwana przez Ankietera nie istnieje}
\ucaction{3.B.1 Ankieter próbuje połączyć kilka Grup Docelowych lub ich fragmentów}
\ucaction{3.B.2 W przypadku niepowodzenia kroku rozszerzenia 3.B.1, bądź wystąpienia takiej konieczności, Ankieter informuje Administratora Bazy Danych, że nie ma praw do wysyłania ankiet do wskazanych osób i/lub powiadamia go (za pomocą poczty elektronicznej) o potrzebie stworzenia Grupy Docelowej o konkretnych atrybutach (BC03)}
\ucaction{3.B.3 W przypadku pominięcia kroku rozszerzenia 3.B.2, powrót do kroku 4., w przeciwnym razie, powrót do kroku 1.}
}}
{}

\subsection{UC04: Uruchomienie Badania}

\ucsection{UC04: Uruchomienie Badania}{Ankieter, Respondent}
{
1. Ankieta znajduje się w Systemie i jest dostępna dla Ankietera \\
2. Ankieta jest powiązana z Grupą Docelową \\
3. Ankieter jest zalogowany w Systemie i chce rozesłać istniejącą Ankietę
}
{Respondenci powiadomieni o Ankiecie}{\ucactions{
\ucaction{1. Ankieter ustawia czas rozpoczęcia i zakończenia Ankiety}
\ucaction{2. Ankieter potwierdza chęć uruchomienia Badania}
\ucaction{3. System rozsyła Ankietę do Respondentów}
\ucaction{4. System powiadamia Respondentów o Ankiecie}
}}
{\ucextensions{
\ucaction{1.A. Ankieter chce prowadzić ankietę wielokrotną}
\ucaction{1.A.1 Ankieter ustawia częstotliwość ankiety}
}}
{}

\subsection{UC06: Sprawdzenie wyników}

\ucsection{UC06: Sprawdzenie wyników}{Ankieter}
{
1. Ankieta znajduje się w Systemie, zawiera odpowiedzi od Grupy Docelowej i jest dostępna dla Ankietera \\
2. Ankieter jest zalogowany w Systemie i chce pozyskać informacje od Studentów/Absolwentów
}
{Wygenerowany Raport}{\ucactions{
\ucaction{1. Ankieter wybiera Ankietę, której wyniki chce poznać}
\ucaction{2. Ankieter wybiera typ Raportu, który chciałby zobaczyć}
\ucaction{3. System generuje i wyświetla Raport}
}}
{}

\section{Respondent}

Poniżej przedstawiono przypadki użycia dotyczące Respondenta.

\subsection{UC05: Udzielenie odpowiedzi}

\ucsection{UC05: Udzielenie odpowiedzi}{Respondent}
{
1. Respondent dostaje powiadomienie o Ankiecie (link bezpośredni do Ankiety) \\
2. Respondent jest zalogowany w Systemie i chce wypełnić Ankietę
}
{}{\ucactions{
\ucaction{1. System prezentuje Ankietę Respondentowi}
\ucaction{2. Respondent udziela odpowiedzi na pytania}
\ucaction{3. System prezentuje podsumowanie odpowiedzi}
\ucaction{4. Respondent zatwierdza wypełnioną Ankietę}
\ucaction{5. System zapisuje odpowiedzi}
}}
{\ucextensions{
\ucaction{1.A. Przedawniona Ankieta}
\ucaction{1.A.1 System informuje, że Ankieta już się zakończyła}
\ucaction{5.A. Brak odpowiedzi na niektóre pytania}
\ucaction{5.A.1 System informuje, że pozostały pytania bez odpowiedzi}
\ucaction{5.A.2 Powrót do kroku 2.}
}}
{}

\section{Administrator Bazy Danych}

Poniżej przedstawiono przypadki użycia dotyczące Administratora Bazy Danych.

\subsection{UC07: Tworzenie Grupy Docelowej}

\ucsection{UC07: Tworzenie Grupy Docelowej}{Administrator Bazy Danych}
{
1. Ankieter chce wysyłać Ankiety do określonych Respondentów w prosty sposób \\
2. Administrator Bazy Danych jest zalogowany w Systemie i chce utworzyć nową Grupę Docelową
}
{Nowa Grupa Docelowa w Systemie}{\ucactions{
\ucaction{1. Administrator Bazy Danych wybiera opcję tworzenia Grup Docelowych}
\ucaction{2. System prezentuje formularz tworzenia Grupy Docelowej}
\ucaction{3. Administrator Bazy Danych wprowadza nazwę tworzonej Grupy Docelowej, wybiera Grupę Nadrzędną oraz Respondentów do dodania do Grupy Docelowej}
\ucaction{4. Administrator Bazy Danych potwierdza chęć stworzenia Grupy Docelowej}
\ucaction{5. System zapisuje nową Grupę Docelową}
}}
{\ucextensions{
\ucaction{4.A Brak Grupy Nadrzędnej}
\ucaction{4.A.1 Administrator Bazy Danych nie uzupełnia Grupy Nadrzędnej}
}}
{}

\subsection{UC08: Edycja Grupy Docelowej}

\ucsection{UC08: Edycja Grupy Docelowej}{Administrator Bazy Danych}
{
1. Grupa Docelowa znajduje się w Systemie \\
2. Administrator Bazy Danych jest zalogowany w Systemie i chce zmodyfikować Grupę Docelową
}
{Zmodyfikowana lista członków Grupy Docelowej}{\ucactions{
\ucaction{1. Administrator Bazy Danych wybiera Grupę Docelową}
\ucaction{2. Administrator Bazy Danych wybiera członka\slash członków Grupy Docelowej do edycji\slash usunięcia}
\ucaction{3. Administrator Bazy Danych dodaje\slash edytuje]\slash usuwa członka\slash członków Grupy Docelowej}
\ucaction{4. Administrator Bazy Danych potwierdza chęć wprowadzenia zmian}
\ucaction{5. System zapisuje zmiany}
}}
{}

\section{Wszyscy Użytkownicy}

Poniżej przedstawiono przypadki użycia dotyczące wszystkich Użytkowników.

\subsection{UC09: Logowanie do Systemu}

\ucsection{UC09: Logowanie do Systemu}{Użytkownik (Ankieter, Administrator, Administrator Bazy Danych, Respondent)}
{Użytkownik posiada konto w Systemie i posiada poprawne dane logowania}
{Użytkownik jest zalogowany w Systemie}{\ucactions{
\ucaction{1. System wyświetla Użytkownikowi formularz logowania}
\ucaction{2. Użytkownik podaje login lub adres e-mail oraz hasło}
\ucaction{3. System uwierzytelnia Użytkownika}
}}
{\ucextensions{
\ucaction{2.A Użytkownik chce się zalogować przy pomocy eKonta}
\ucaction{2.A.1 Użytkownik wybiera opcję logowania przez eKonto}
\ucaction{2.A.2 System przekierowuje Użytkownika na stronę logowania przez eKonto (eLogin)}
\ucaction{2.A.3 Użytkownik wprowadza dane logowania do eKonta}
\ucaction{2.A.4 Powrót do kroku 3.}
}}
{}
\chapter{Wymagania pozafunkcjonalne}
\label{Chapter4}

\section{Wstęp}
\label{Chapter41}

W niniejszym rozdziale zostaną zaprezentowane i krótko opisane charakterystyki oraz wymagania pozafunkcjonalne obowiązujące dla systemu. Ponadto zostanie podjęta próba weryfikacji, które wymagania udało się spełnić i jakie są perspektywy rozwoju

\section{Charakterystyki oprogramowania}
%
%Poniżej są jeszcze stare charakterystyki oprogramowania (wtedy myśleliśmy, że to mądrze brzmi), kategorii wymagań pozafunkcjonalnych. Obecnie to się trochę zmieniło, zatem ta lista będzie znacznie bardziej rozbudowana.
%
%\begin{itemize}
%\item Dokładność (ang. \definicja{accuracy})
%\item Bezpieczeństwo (ang. \definicja{security})
%\item Odporność na błędy (ang. \definicja{fault tolerance})
%\item Odtwarzalność (ang. \definicja{recoverability})
%\item Charakterystyka czasowa (ang. \definicja{time behaviour})
%\item Łatwość analizowania (ang. \definicja{analysability})
%\item Łatwość zmian (ang. \definicja{changeability})
%\item Adaptowalność (ang. \definicja{adaptability})
%\item Instalowalność (ang. \definicja{installability})
%\item Współistnienie (ang. \definicja{co-existence})
%\item Zamienność (ang. \definicja{replaceability})
%\end{itemize}
%
%Tutaj piszemy, które podcharakterystyki były dla nas priorytetowe lub szczególnie ważne, a które mniej. Oczywiście, z uzasadnieniem. Piszemy również, jak zamierzaliśmy (lub to robiliśmy) dbać o to, aby wszystko było spełnione.
%
\section{Wymagania pozafunkcjonalne i ich weryfikacja}
%
%W tablicy \ref{tab:reqs} przedstawiono wymagania pozafunkcjonalne związane z systemem. W kolumnach \textbf{Priorytet} oraz \textbf{Trudność} określono poziomy przy pomocy następującej notacji:
%
%\begin{itemize}
%\item H -- wysoki priorytet lub poziom trudności
%\item M -- średni priorytet lub poziom trudności
%\item L -- niski priorytet lub poziom trudności
%\item N -- wymaganie oczywiste lub bardzo proste do spełnienia
%\end{itemize}
%\begin{table}[h]
%\centering
%\begin{tabular}{ | c | p{7cm} | c | c | }
%\hline
%\textbf{Podcharakterystyka} & \textbf{Wymaganie} & \textbf{Priorytet} & \textbf{Trudność} \\ %\hline
%Nazwa podcharakterystyki & Wymaganie 1 (przykład procentów: 90\%) & H & H \\ \hline
%Nazwa podcharakterystyki & Wymaganie 2 & M & H \\ \hline
%Nazwa podcharakterystyki & Wymaganie 3 & H & L \\ \hline
%Nazwa podcharakterystyki & Wymaganie 4 & N & L \\ \hline
%... & ... & ... & ... \\ \hline
%\end{tabular}
%\caption{Wymagania pozafunkcjonalne}\label{tab:reqs}
%\end{table}
%A tutaj piszemy o wszelkich problemach, wszelkich naszych wnioskach związanych ze spełnianiem wymagań pozafunkcjonalnych -- co się udało, co nie (i dlaczego). Tak, jakbyśmy opisywali nasze doświadczenia i problemy, z jakimi przyszło nam się zmagać i być może rozwiązać. Staramy się odnieść do najważniejszych wymagań (chyba że jest ich mało, wtedy do wszystkich).
\chapter{Architektura systemu}
\label{Chapter5}

\section{Wstęp}
\label{Chapter51}

System iQuest został stworzony w oparciu o model architektury trójwarstwowej, w którym wyróżnione zostały warstwy: danych, logiki biznesowej oraz prezentacji. Dzięki takiemu podejściu, zadania związane z poszczególnymi warstwami można było bez większego problemu rozdzielić między członków zespołu programistów, a w przypadku ewentualnej modyfikacji jednej z warstw nie występuje konieczność wprowadzania zmian w reszcie projektu.

\section{Opis ogólny architektury -- Marketecture}
\label{Chapter52}

\begin{figure}[H]
\centering\includegraphics[width=15cm]{figures/marketecture}
\caption{Diagram "Marketektury"}\label{rys:marketecture}
\end{figure}

System iQuest to aplikacja internetowa w postaci zbioru rozszerzeń platformy e-learningowej Moodle. Całość (Moodle oraz rozszrzenia) zainstalowana jest na serwerze www, zlokalizowanym w sieci Politechniki Poznańskiej, łączącym się z osobnym serwerem baz danych oraz usługami eKonto i eDziekanat. Funkcje raportowania realizowane są w głównej mierze za pośrednictwem zewnętrznego systemu BI, pobierającego dane z iQuest za pośrednictwem usług sieciowych (en.~\definicja{webservices}). Administracja oraz obsługa systemu odbywa się za pośrednictwem przeglądarki internetowej, w ramach połączenia z platformą Moodle lub serwerem raportowania. Respondenci mogą też uzyskać dostęp do systemu przy pomocy urządzeń mobilnych, takich jak tablety czy smartfony.

%\section{Analiza SWOT}
%\label{Chapter53}
%
%My tego nie mieliśmy, ale chyba warto -- tutaj analiza SWOT przyjętego podejścia architektonicznego.
%
\section{Perspektywy architektoniczne}
\label{Chapter53}

\subsection{Perspektywa fizyczna}
\label{Chapter531}

\begin{figure}[H]
\centering\includegraphics[width=15cm]{figures/PhysicalView}
\caption{Diagram perspektywy fizycznej}\label{rys:PerspektywaFizyczna}
\end{figure}

Powyższy schemat prezentuje perspektywę fizyczną projektu. Widać na nim dokładnie opisaną wcześniej budowę systemu iQuest, opartą na rozszerzeniach dla platformy Moodle, w ramach których wykonywana jest cała logika aplikacji. Za jej pośrednictwem dokonuje on połączeń z serwerem baz danych oraz serwerem raportowania. Użytkownik systemu, z wykorzystaniem przeglądarki internetowej, komunikuje się z platformą, lub systemem raportowania, uzyskując w ten sposób dostęp do warstwy prezentacji.

\subsection{Perspektywa logiczna}
\label{Chapter532}

\begin{figure}[H]
\centering\includegraphics[width=15cm]{figures/LogicalView}
\caption{Diagram perspektywy logicznej}\label{rys:PerspektywaLogiczna}
\end{figure}

Przedstawiony powyżej schemat prezentuje perspektywę logiczną systemu. Określa ona zależności między poszczególnymi komponentami ,,wszczepionymi'' do platformy Moodle. Poniżej znajduje się opis wyszczególnionych na rysunku komponentów.

\subsubsection{iQuest Survey Activity Plugin}
\label{Chapter5321}
Funkcjonalnością Activity Pluginów jest udostępnianie możliwości dodania nowych rodzajów Aktywności w ramach platformy Moodle. Pozwala on na dodanie nowego Badania iQuest w ramach kursu iQuest. Komponent ten składa się z dwóch subkomponentów:
\begin{itemize}
\item{Survey Creator,}
\item{Survey Runner.}
\end{itemize}
Pierwszy z nich odpowiada za definiowanie ankiet, podczas gdy drugi za ich przeprowadzanie.

\subsubsection{iQuest Course Format Plugin}
\label{Chapter5322}

Course Format Plugin dla platformy Moodle odpowiada za obsługę interfejsu użytkownika. W przypadku systemu iQuest, zarządza kwestią wyświetlania użytkownikowi tylko tych składowych kursu "iQuest", które są dla niego dostępne. Przykładowo, Respondent uzyska dostęp do listy ankiet, które może wypełnić, podczas gdy ankieter uzyska dostęp do listy zarządzanych przez niego badań.

\subsubsection{eDziekanat Connector oraz ePoczta Connector}
\label{Chapter5323}

Komponenty te odpowiadają za komunikację z usługami eDziekanat i ePoczta, pozwalające m.in. na pozyskanie danych o grupach docelowych (w oparciu o dane Grup Dziekańskich) oraz wysyłanie powiadomień za pośrednictwem poczty studenckiej.

\subsubsection{Background Task Scheduler and Executor}
\label{Chapter5324}

Dzięki temu komponentowi możliwe jest szeregowanie oraz wykonywanie zadań w tle. Jednym z jego zadań jest kolejkowanie i aktywowanie mechanizmów rozsyłania wiadomości e-mail z zaproszeniami do udziału w ankiecie.

\subsubsection{eKonto Authentication Plugin}
\label(Chapter5325}

eKonto Authentication Plugin -- to moduł uwierzytelniania (en.~\definicja{Authentication}), korzystający z systemu eLogin platformy eKonto do logowania się do platformy Moodle, obsługującej system iQuest. Korzystanie z tego systemu pozwala nie tylko na jednoznaczną weryfikację tożsamości użytkownika łączącego się z systemem, ale jest zarazem wygodne - dzięki jego zastosowaniu, nie ma potrzeby posiadania osobnego konta w systemie iQuest.

\subsection{Perspektywa implemetancyjna}
\label{Chapter533}

\begin{figure}[H]
\centering\includegraphics[width=15cm]{figures/Layers}
\caption{Diagram perspektywy implementacyjnej}\label{rys:PerspektywaImplementacyjna}
\end{figure}

Diagram perspektywy implementacyjnej pozwala na analizę przeplatania się elementów systemu iQuest w ramach poszczególnych warstw zastosowanego modelu trójwarstwowego. Funkcje zaprezentowanych na nim modułów zostały wyjaśnione już wcześniej w niniejszym dokumencie.

\subsection{Perspektywa procesu (równoległości)}
\label{Chapter534}

{\color{red}Rysunek wraz z opisem.}

\section{Decyzje projektowe}
\label{Chapter54}

{\color{red}Tutaj będzie o decyzjach projektowych, związkach pomiędzy nimi oraz innych związanych sprawach. Jest tego dość sporo i wymaga specyficznego formatowania.}

\section{Wykorzystane technologie}
\label{Chapter55}

{\color{red}Opis wykorzystywanych technologii (COTS). Sprawdzić czy ten rozdział to nie dubel!}

\section{Schemat bazy danych}
\label{Chapter56}

Schemat bazy danych, ze względu na objętość, zamieszczono w dodatku C.
\chapter{Opis implementacji}
\label{Chapter6}

\section{Wstęp}
\label{Chapter61}

{\color{red}Wersja bez redakcji.}

%Wprowadzenie. Struktura tego rozdziału nie jest z góry określona, gdyż mocno zależy to od specyfiki projektu. Generalnie w poszczególnych podrozdziałach każdy powinien opisać swoją część z takiego technicznego punktu widzenia. Piszecie, jak zrealizowaliście poszczególne wymagania, jak to wygląda ,,pod maską'', oczywiście też trzeba przyjąć jakiś poziom szczegółowości. W bardzo szczególnych przypadkach chyba może się zdarzyć, że trzeba będzie załączyć fragment jakiegoś kodu źródłowego czy konfiguracji -- generalnie ma to być opisane w taki sposób, że jako osoba nieznająca systemu siadam i wiem, jak i co zrobiliście. Oczywiście, to moje dywagacje, być może osoby związane z uczelnią zlinczują mnie za ten fragment.
%
\section{Użyte technologie}
\label{Chapter62}

\subsection{Moodle}
\emph{Moodle} (roz. \textit{Modular Object-Oriented Dynamic Learning Environment}) -- stanowi podstawę systemu \textit{iQuest}. Wyboru dokonano ze względu na kilka czynników:
\begin{itemize}
\item{Propozycję Architekta, wynikającą z faktu, iż Moodle posiada już implementację wielu wymaganych w \textit{iQuest} mechanizmów.}
\item{Oczekiwania Klienta, wynikające z popularności platformy Moodle wśród systemów uczelnianych.}
\end{itemize}

\subsection{PHP}
\emph{PHP} -- platforma Moodle opiera się właśnie na tym języku programowania. Z tego względu, jest to też technologia zastosowana w większości rozszerzeń utworzonych przez zespół /textit{iQuest}, korzystających z wielu interfejsów programowania aplikacji tej platformy.

\subsection{PHPUnit}
Ze względu na fakt, iż programiści \textit{Moodle'a} wykonują testy jednostkowe kodu wykorzystując do tego celu \emph{PHPUnit}, zdecydowano o wzorowaniu się na tym działaniu. \textit{Moodle} udostępnia dwie klasy do testowania -- \textit{basic\_testcase} i \textit{advanced\_testcase}, przy czym ta druga służy do testów, które wchodzą w interakcję z bazą danych.

\subsection{Selenium}
\emph{Selenium} -- szybko rozwijające się narzędzie do testów akceptacyjnych. Był to naturalny wybór zwłaszcza, że zostało ono przybliżone zespołowi \textit{iQuest} na zajęciach z Inżynierii Oprogramowania w trakcie toku studiów.

\subsection{PostgreSQL}
System zarządzania bazą danych \emph{PostgreSQL} został wybrany ze względu na wymagania pozafunkcjonalne.

\subsection{Eclipse IDE}
Wybór \emph{Eclipse IDE} jako stosowanego dla projektu \textit{iQuest} zintegrowanego środowiska programistycznego wynika z faktu, iż oprogramowanie to jest dostępne za darmo. Dodatkową zaletą \textit{Eclipse} jest modularność tego rozwiązania, dzięki czemu dostępny jest w nim dodatek \emph{PHP Development Tools}, znacząco upraszczający pracę z technologią PHP.

\subsection{SVN}
\emph{Subversion} został wybrany jako podstawowy system zarządzania treścią ze względu na wymagania pozafunkcjonalne. Dodatkową zaletą jego użycia jest fakt, że zespół eksploatacji, który docelowo przejmie zarządzanie artefaktami związanymi z projektem, wykorzystuje właśnie \textit{SVN}.

\subsection{Redmine}
Systemu zarządzania projektami \emph{Redmine} wykorzystywany był od samego początku istnienia projektu. Jest to narzędzie bardzo przydatne w wymianie informacji pomiędzy członkami zespołu, integrujące się m.in. z repozytorium kodu. Technologia ta została narzucona, ze względu na sposób organizacji pracy w \textit{Software Development Studio} na Politechnice Poznańskiej.

\subsection{JasperReports}
Ze względu na wymagania pozafunkcjonalne, zdecydowano się skorzystać z mechanizmów raportowania oferowanych przez \emph{JasperReports}.

%MT
\subsection{JavaScript}
Formularze wymagające częstej interakcji z klientem, np. formularz umożliwiający tworzenie nowej ankiety, oraz funkcje związane z walidacją pól uzupełnianych przez klienta zostały napisane w \emph{JavaScript}. Obsługa strony po stronie użytkownika zapobiega frustracji, związanej z częstym przeładowywaniem całej strony, co ogranicza zarówno obciążenie łącza po stronie serwera, jak i po stronie klienta.

\section{Ogólna struktura projektu}
\label{Chapter63}

{\color{red}Sekcja druga.}

\section{Interfejs}
\label{Chapter64}

\subsection{Wprowadzenie}
Jedną z części pracy było zaprojektowanie graficznego interfejsu użytkownika. Głównym problemem jaki się pojawił, był wybór odpowiedniego narzędzia. Celem jaki postawiono, była maksymalna zgodność projektowanych elementów z różnymi wersjami \emph{Moodle} -- zarówno wcześniejszymi, jak i późniejszymi. Zdecydowano, aby starać się korzystać z gotowych interfejsów programowania aplikacji \emph{(API)} dostarczonych przez \emph{Moodle}, tj. \emph{Page API}, \emph{Form API}, oraz \emph{Access API}. Wszystkie interfejsy są napisane przy użyciu języka PHP -- są wykonywane po stronie serwera. Konieczne okazało się też wykonanie niektórych skryptów po stronie klienta. Dlatego w projekcie wykorzystano również język skryptowy \emph{Java Script}.

\section{Logika (back-end)}
\label{Chapter65}

Jednym z zadań w ramach pracy było zaprogramowanie odpowiedniej logiki biznesowej rozwiązującej zadania stawiane przed zaprojektowanym systemem. Najważniejszym zadaniem z perspektywy back-end'u jest interakcja z bazą danych. Poza tym system posiada: procesor zadań wykonywanych w tle oparty na \emph{cron}; moduł odpowiadający za komunikację z systemem uczelanianym \emph{ePoczta}; moduł logowania zdarzeń. W trakcie implementacji zdecydowano się nie tworzyć osobnego mechanizmu do przechowywania ustawień w bazie danych i skorzystaliśmy z istniejącego już w \emph{Moodle}. Jednym z wymagań pozafunkcjonalnych było wykorzystanie bazy danych \emph{PostgreSQL}. Platforma \emph{Moodle} korzysta z mechanizmu \emph{XMLDB}, co pozwala na ominięcie wielu problemów pojawiających się przy migracjach pomiędzy różnymi systemami baz danych. Niestety kosztem wykorzystania tego mechanizmu jest konieczność pracy z interfejsami programowania aplikacji dostarczanymi przez platformę \emph{Moodle}, m.in. \emph{Data manipulation API} (zarządzanie danymi), \emph{Access API} (zarządzanie dostępem, rolami, prawami).\\

{\color{red} Część do rozszczepienia w inne obszary}
Poniżej przedstawiony został schemat bazy danych i logikę biznesową z perspektywy implementacyjnej.

\newpage
\begin{landscape}

\begin{figure}[!th]
\centering\includegraphics[width=1.25\textheight]{figures/iQuest_Database.png}
\caption{Backend -- Schemat bazy danych}\label{rys:iquest-db}
\end{figure}

\begin{figure}[!th]
\centering\includegraphics[width=1.25\textheight]{figures/Survey_Creator_Survey_Runner.png}
\caption{Backend -- moduły Survey Creator i Survey Runner}\label{rys:iquest-backend}
\end{figure}

\begin{figure}[!th]
\centering\includegraphics[width=1.25\textheight]{figures/ePoczta_Connector_Background_Task_Scheduler_and_Executor.png}
\caption{Backend -- moduły ePoczta Connector i Background Task Scheduler and Executor}\label{rys:iquest-backend2}
\end{figure}

\end{landscape}

\section{Powiązanie back-endu z interfejsem}
\label{Chapter66}

{\color{red}Dalsze opisy.}
\chapter{Zapewnianie jakości i konserwacja systemu}
\label{Chapter7}

\section{Testy i weryfikacja jakości oprogramowania}
\label{Chapter71}

\subsection{Wstęp}
\label{Chapter711}

Testy i weryfikacja jakości oprogramowania realizowana była na trzech poziomach: testów jednostkowych (dla logiki) oraz automatycznych i manualnych testów akceptacyjnych. Te ostatnie realizowane były nie tylko w zgodzie z dokumentem \textit{MAT}\cite{Redmine:ProjDocs}, ale też intuicyjnie, poprzez zwykłe korzystanie z systemu.

\subsection{Testy jednostkowe}
\label{Chapter712}

Testy jednostkowe zostały wykonane jako pierwsze i traktowane były z wysokim priorytetem. Realizowane były z użyciem klas PHPUnit, stosowanych powszechnie m.in.~przy testowaniu wtyczek do platformy Moodle. Testy te były kluczowe dla rozwoju logiki systemu iQuest. Przygotowaliśmy konfigurowalne skrypty automatyzujące proces testowania w języku BASH. Część testów operuje na systemie w trybie produkcyjnym, część na trybie testowym, obsługującym tzw. ,,atrapy'' (ang. \definicja{mock}), imitujące działanie systemów zewnętrznych poprzez zwracanie przykładowych danych.\\

Przy realizacji pierwszego wydania, za testy jednostkowe w pełni odpowiadał jeden z członków zespołu programistów. W wydaniu drugim, rolę tę przejął programista realizujący logikę systemu. Z początku próbowaliśmy działać w oparciu o TDU (ang. \definicja{Test-Driven Development} - rozwój w oparciu o testy), jednakże szybko porzuciliśmy tą metodykę, ze względu na brak doświadczenia programistów.

Na początku projektu przygotowaliśmy program w języku Java uruchamiający zestaw testów akceptacyjnych. W drugim wydaniu korzystaliśmy już wyłącznie z Selenium IDE, ze względu możliwość szybszego rozpoznania problemów z poziomu przeglądarki, w przeciwieństwie do terminala.

\begin{figure}[H]
\begin{center}
\includegraphics[width=0.9\textwidth]{figures/lw/tests.pdf} 
\end{center}
\caption{Struktura klas testujących}
\label{fig:tests}
\end{figure}

Na diagramie widzimy trzy klasy służące do testowania, z których dziedziczą wszystkie inne, są to:
\begin{itemize}
\item \emph{basic\_testcase} -- podstawowe testy jednostkowe
\item \emph{advanced\_testcase} -- testy z użyciem bazy danych
\item \emph{iquest\_testcase} -- rozszerzenie \emph{advanced\_testcase} na potrzeby testowania klas dziedziczących z \emph{record}
\end{itemize}

\subsection{Testy akceptacyjne}
\label{Chapter713}

Testy akceptacyjne rozpatrywane są na dwóch poziomach: automatycznym i manualnym. Różnica polega jedynie na tym, kto (lub co) wykonuje test - komputer z odpowiednim oprogramowaniem, czy człowiek.

\subsubsection{MAT}
\label{Chapter7131}
{\color{red}Testy akceptacyjne wymagają poprawienia -- dane od zespołu zarządzającego były nieaktualne!!!}
Poniżej przedstawiono Manualne Testy Akceptacyjne:

\matbegin{TC1}{Logowanie do systemu przez eKonto}
\matpres
\matpre{Użytkownik jest niezalogowany}
\matpre{Użytkownik posiada eKonto}
\matpre{Połączenie z Internetem}
\matsteps
\matstep{1}{Użytkownik wpisuje adres systemu}{Strona logowania do system iQuest}
\matstep{2}{Użytkownik naciska przycisk ,,Zaloguj przez eKonto''}{Strona logowania eLogin}
\matstep{3}{Użytkownik wpisuje dane logowania}{}
\matstep{4}{Użytkownik naciska przycisk ,,Zaloguj''}{Przekierowanie na stronę systemu iQuest, wyświetlenie strony głównej z zalogowanym użytkownikiem.}
\matremark{}

\matbegin{TC2}{Stworzenie Ankiety}
\matpres
\matpre{Zalogowany użytkownik z prawem do tworzenia ankiet}
\matsteps
\matstep{1}{Użytkownik wybiera przycisk ,,Stwórz ankietę''}{Strona umożliwiająca tworzenie ankiet}
\matstep{2}{Użytkownik podaje nazwę ankiety}{}
\matstep{3}{Użytkownik podaje wstęp  i podsumowanie ankiety}{}
\matstep{4}{Użytkownik dodaje pytanie jednokrotnego wyboru}{Pojawia się pole na wpisanie treści pytania}
\matstep{5}{Użytkownik wpisuje treść pytania}{}
\matstep{6}{Użytkownik naciska przycisk „Dodaj odpowiedź'' dwukrotnie}{Pojawiają się dwa pola do wpisania możliwych odpowiedzi}
\matstep{7}{Użytkownik podaje treści możliwych odpowiedzi}{}
\matstep{8}{Użytkownik dodaje stronę wciskając przycisk „Dodaj stronę''}{Wyświetla się nowa strona na dodawanie pytań}
\matstep{9}{Użytkownik dodaje pytanie otwarte}{Wyświetla się pole na wpisanie treści pytania}
\matstep{10}{Użytkownik wpisuje treść pytania}{}
\matstep{11}{Użytkownik wybiera przycisk „Zapisz zmiany''}{Komunikat o pomyślnym stworzeniu ankiety}
\matremark{}

\begin{table}[H]
\centering
\begin{tabular}{ | >{\bfseries}c | p{5cm} | } \hline
Krok & \textbf{Dane} \\ \hline
2 & ,,Ankieta testowa'' \\ \hline
3 & ,,Wstęp'' oraz ,,Podsumowanie'' \\ \hline
5 & ,,Pytania jednokrotnego wyboru działają?'' \\ \hline
7 & ,,tak'' oraz ,,nie'' \\ \hline
10 & ,,Pytania otwarte działają?'' \\ \hline
\end{tabular}
\caption{Poprawne dane dla scenariusza TC2}\label{tab:TC2-correct}
\end{table}

\matbegin{TC2.2}{Stworzenie Ankiety - brak pytań}
\matpres
\matpre{Zalogowany użytkownik z prawem do tworzenia ankiet}
\matsteps
\matstep{1}{Użytkownik wybiera przycisk „Stwórz ankietę''}{Strona umożliwiająca tworzenie ankiet}
\matstep{2}{Użytkownik podaje nazwę ankiety}{}
\matstep{3}{Użytkownik podaje wstęp  i podsumowanie ankiety}{}
\matstep{4}{Użytkownik wybiera przycisk „Zapisz zmiany''}{Komunikat o braku pytań w ankiecie}
\matremark{}

\matbegin{TC3}{Edycja ankiety}
\matpres
\matpre{Zalogowany użytkownik z prawem do tworzenia ankiet}
\matpre{Użytkownik posiada prawo do edycji ankiety ,,Ankiety testowej''}
\matsteps
\matstep{1}{Użytkownik wybiera przycisk „Katalog Ankiet''}{Strona z listą ankiet do których prawo ma użytkownik.}
\matstep{2}{Użytkownik wybiera przycisk „edytuj'' przy „Ankiecie Testowej''}{Strona umożliwiająca edycję „Ankiety Testowej''}
\matstep{3}{Użytkownik wciska przycisk „usuń'' przy pytaniu drugim}{Pytanie drugie znika}
\matstep{4}{Użytkownik naciska przycisk „Dodaj odpowiedź'' przy pytaniu pierwszym}{Pojawia się pole do wpisania możliwej odpowiedzi}
\matstep{5}{Użytkownik wpisuje możliwą odpowiedź}{}
\matstep{6}{Użytkownik wybiera przycisk „Zapisz zmiany''}{Strona wyświetla komunikat potwierdzający zapisanie zmian w ankiecie.}
\matremark{}

\begin{table}[H]
\centering
\begin{tabular}{ | >{\bfseries}c | p{5cm} | } \hline
Krok & \textbf{Dane} \\ \hline
5 & ,,nie wiem'' \\ \hline
\end{tabular}
\caption{Poprawne dane dla scenariusza TC3}\label{tab:TC3-correct}
\end{table}

\matbegin{TC3}{Edycja ankiety - usunięcie wszystkich pytań}
\matpres
\matpre{Zalogowany użytkownik z prawem do tworzenia ankiet}
\matpre{Użytkownik posiada prawo do edycji ankiety ,,Ankiety testowej''}
\matsteps
\matstep{1}{Użytkownik wybiera przycisk „Katalog Ankiet''}{Strona z listą ankiet do których prawo ma użytkownik.}
\matstep{2}{Użytkownik wybiera przycisk „edytuj'' przy „Ankiecie Testowej''}{Strona umożliwiająca edycję „Ankiety Testowej''}
\matstep{3}{Użytkownik wciska przycisk „usuń'' przy każdym z pytań}{Pytania znikają}
\matstep{4}{Użytkownik wybiera przycisk „Zapisz zmiany''}{Strona wyświetla komunikat, że ankieta nie posiada pytań i prosi o ich dodanie.}
\matremark{}

\matbegin{TC4}{Wybranie grupy docelowej}
\matpres
\matpre{Zalogowany użytkownik z prawami do tworzenia ankiet}
\matpre{Użytkownik posiada prawa do ankiety „Ankieta Testowa''}
\matpre{Użytkownik posiada prawo do ankietowania grupy docelowej „test''}
\matsteps
\matstep{1}{Użytkownik wybiera przycisk ,,Włącz tryb edycji''}{Interfejs edycji Moodle}
\matstep{2}{Użytkownik wybiera przycisk „edytuj'' przy „Ankiecie Testowej''}{Strona umożliwiająca edycję „Ankiety Testowej''}
\matstep{3}{Użytkownik wybiera grupę docelową „test''}{}
\matstep{4}{Użytkownik wybiera przycisk „Zapisz zmiany''}{Strona wyświetla komunikat, że ankieta została zaktualizowana pomyślnie.}
\matstep{5}{Respondent otrzymuje email z powiadomieniem o ankiecie}{}
\matremark{}

\matbegin{TC5}{Udzielanie odpowiedzi}
\matpres
\matpre{Zalogowany użytkownik}
\matpre{Użytkownik znajduje się w grupie docelowej ankiety ,,testowa''}
\matpre{Użytkownik nie odpowiadał udzielał odpowiedzi na ankietę ,,testowa''}
\matsteps
\matstep{1}{System prezentuje ankiety na które użytkownik jeszcze nie odpowiedział}{}
\matstep{2}{Użytkownik wybiera ankietę ,,testowa''}{System prezentuje ankietę}
\matstep{3}{Użytkownik zaznacza odpowiedź na pytanie 1 jako ,,tak''}{}
\matstep{4}{Użytkownik podaje odpowiedź na pytanie drugie jako ,,tak''}{}
\matstep{5}{Użytkownik potwierdza wypełnienie ankiety przyciskiem ,,Wyślij''}{System prezentuje ankiety na które użytkownik jeszcze nie odpowiedział oraz komunikat o pomyślnym przesłaniu odpowiedzi}
\matstep{6}{Respondent otrzymuje email z powiadomieniem o ankiecie}{}
\matremark{}

\matbegin{TC6}{Sprawdzenie wyników}
\matpres
\matpre{Zalogowany użytkownik z prawem do oglądania wyników ankiety ,,Testowa''}
\matsteps
\matstep{1}{Użytkownik wybiera badanie, którego podstawowe wyniki chce sprawdzić}{System prezentuje podsumowanie ankiety}
\matremark{Dotyczy podstawowych wyników. Wyniki zaawansowane obsługuje zewnętrzny serwer BI}

\matbegin{TC7}{Dodawanie grupy docelowej}
\matpres
\matpre{Zalogowany administrator z prawami do tworzenia grup docelowych}
\matpre{Brak w systemie grupy docelowej ,,Grupa 1''}
\matsteps
\matstep{1}{Administrator wybiera przycisk ,,Zarządzaj grupami docelowymi''}{System prezentuje dostępne grupy docelowe w systemie}
\matstep{2}{Administrator wybiera przycisk ,,dodaj''}{System prezentuje interfejs dodawania grupy docelowej}
\matstep{3}{Administrator podaję nazwę nowej grupy „Grupa 1” i wskazuje jej członków}{}
\matstep{4}{Administrator wybiera grupę nadrzędną dla nowej grupy docelowej}{System automatycznie zapisują zmiany}
\matremark{Test przygotowany na podstawie makiety systemu iQuest}

\matbegin{TC8}{Edycja grupy docelowej}
\matpres
\matpre{Zalogowany administrator z prawami do tworzenia grup docelowych}
\matpre{Grupa docelowa ,,Grupa 1'' istnieje w systemie}
\matsteps
\matstep{1}{Administrator wybiera przycisk ,,Zarządzaj grupami docelowymi''}{System prezentuje dostępne grupy docelowe w systemie}
\matstep{2}{Administrator wybiera przycisk ,,zmień nazwę''}{System prezentuje interfejs zmiany nazwy grupy docelowej}
\matstep{3}{Administrator pozostawia nazwę niezmienioną i zatwierdza zmiany}{System prezentuje dostępne grupy docelowe w systemie}
\matstep{4}{Administrator wybiera grupę nadrzędną dla grupy docelowej i dodaje nowego członka do grupy}{System automatycznie zapisują zmiany}
\matremark{Test przygotowany na podstawie makiety systemu iQuest}

\matbegin{TC9}{Edycja grupy docelowej}
\matpres
\matpre{Użytkownik jest niezalogowany}
\matpre{Istnieje konto użytkownika w systemie}
\matsteps
\matstep{1}{Użytkownik wpisuje adres systemu}{Strona główna moodla z kursem iQuest}
\matstep{2}{Użytkownik naciska przycisk ,,Zaloguj się''}{Strona logowania}
\matstep{3}{Użytkownik wpisuje dane logowania}{}
\matstep{4}{Użytkownik naciska przycisk ,,Zaloguj się''}{Przekierowanie na główną stronę moodla z kursem iQuest. Wyświetla się napis: ,,Jesteś zalogowany(a) jako...''}
\matremark{}

{\color{red}Niedodane jeszcze TC: Logowanie bez eKonta, }

Dodatkowo, poniżej znajduje się wykaz mapujący przypadki testowe do przypadków użycia:
\begin{itemize}
\item{TC1.X – UC09 Logowanie do systemu.}
\item{TC2.X – UC01 Stworzenie ankiety}
\item{TC3.X – UC02 Edycja ankiety}
\item{TC4.X – UC03 UC04 Wybranie grupy docelowej, uruchomienie ankiety}
\item{TC5.X – UC05 Udzielenie odpowiedzi}
\item{TC6.X – UC06 Sprawdzenie wyników}
\item{TC7.X – UC07 Tworzenie grupy docelowej}
\item{TC8.X – UC08 Edycja grupy docelowej}
\item{TC9.X – UC09 Logowanie do systemu}
\end{itemize}

\subsubsection{AAT}
\label{Chapter7131}

Automatyczne testy akceptacyjne realizowano w zgodzie z testami manualnymi i operując na tych samych wytycznych. Nagrywanie testów odbywało się za pomocą oprogramowania Selenium IDE, udostępnianego w formie rozszerzenia dla przeglądarki Mozilla Firefox. Pierwotnie, testy były konwertowane do języka Java, celem uruchamiania ich z poziomu języka Java, oferującego sporą swobodę przy projektowaniu warunków początkowych i końcowych dla testów. Problemy, jakie wynikały z takiego działania, opisane zostały w rozdziale \ref{Chapter6}. Na ich podstawie zdecydowano o pozostaniu w obrębie Selenium IDE, które samo w sobie również umożliwia automatyzację w wysokim stopniu. Aby dodatkowo ułatwić zadanie, przygotowany został skrypt ustawiający bazę danych w stan początkowy dla realizacji testów.

\subsection{Inne metody zapewniania jakości}
\label{Chapter714}

Celem zapewnienia jak najwyższej jakości oprogramowania, było ono testowane -- w kontrolowanych warunkach -- na różnorakich maszynach. Co prawda, lokalne serwery developerskie pracowały w oparciu o system Ubuntu 12.04 LTS, z serwerem Apache i systemem zarządzania bazą danych PostgreSQL, jednak maszyny klienckie były już znacznie bardziej różnorodne.

System przetestowano na zbiorze komputerów stacjonarnych klasy PC oraz równoważnych laptopów, z użyciem systemów Windows i Linux. Realizowano je także na laptopach i netbookach klasy PC oraz Macintosh, również pod kontrolą systemów Windows i Linux, oraz - w przypadku tej ostatniej klasy - MacOS. Ponadto, przetestowano trzy główne platformy mobilne (Android, iOS, WP7.x) poprzez dostęp do systemu z poziomu telefonów komórkowych.

Wspomniane wyżej maszyny pracowały pod kontrolą następujących systemów operacyjnych:
\begin{itemize}
\item{Windows XP Professional SP3 32-bit + Internet Explorer 7}
\item{Windows Vista Business SP2 32-bit + Internet Explorer 8}
\item{Windows 7 Professional SP1 64-bit + Mozilla Firefox + Google Chrome}
\item{Windows 8 Professional 64-bit + Internet Explorer 10 + Google Chrome}
\item{Ubuntu 12.04 64-bit + Mozilla Firefox + Google Chrome}
\item{Mac OS X + Apple Safari}
\item{Google Android 2.2, 2.3, 4.0, 4.1}
\item{iOS (iPhone 4S)}
\item{Windows Phone 7.5 (Nokia Lumia 710)}
\end{itemize}

Na podstawie powyższych testów, utworzono dwa raporty: ,,wygląd i działanie systemu iQuest na platformach mobilnych'' oraz ,,wygląd i działanie systemu iQuest w różnych konfiguracjach system-przeglądarka'', udostępnione w ramach systemu zarządzania projektem\cite{Redmine:ProjDocs}. Wynika z nich, że system iQuest posiada bardzo wysoki współczynnik przenośności.
\chapter{Zebrane doświadczenia}
\label{Chapter8}

\chapter{Zakończenie}
\label{Chapter9}

\section{Podsumowanie}
\label{Chapter91}

Realizacja projektu obejmującego utworzenie systemu \textit{iQuest} została zakończona sukcesem. Platforma \textit{Moodle}, w połączeniu z autorskimi wtyczkami składającymi się na system \textit{iQuest} zapewnia pełnię zleconej funkcjonalności, spełniając zarazem wymagania przedstawione w niniejszym dokumencie. \\

Dzięki wdrożeniu systemu \textit{iQuest}, Politechnika Poznańska uzyska dostęp do niezbędnego narzędzia prowadzenia badań wśród swoich absolwentów w rozumieniu ustawy ,,Prawo o Szkolnictwie Wyższym''. Jest to znaczący krok w stronę lepszego poznania potrzeb rynku, zarówno pracodawców, jak i samych studentów, pozwalający usprawnić mechanizmy zapewniania jakości kształcenia funkcjonujące na uczelni. \\

Udział w tak dużym i znaczącym dla Politechniki Poznańskiej projekcie przyczynił się do znaczącego rozwoju jego uczestników. Pozyskali oni wiele cennych doświadczeń i wyciągnęli znaczną liczbę najróżniejszych wniosków. Autorzy niniejszej pracy dyplomowej wyrażają więc nadzieję, że utworzony przez nich system zostanie pozytywnie przyjęty przez studentów, absolwentów oraz prowadzących.

\section{Propozycja dalszych prac}
\label{Chapter92}

Podczas spotkania z reprezentantami Działu Rozwoju Oprogramowania poruszony został temat etykietowania (ang. tag) badań. Proponowany mechanizm z pewnością usprawniłby proces wyszukiwania badań. W obecnej wersji systemu zachętą dla studentów do wypełniania ankiet są materiały publikowane przez wykładowców uczelni, do których dostęp przyznawany jest użytkownikom, którzy w określonym czasie udzielili odpowiedzi w dowolnym badaniu. W przyszłości mechanizm ten można zastąpić możliwością subskrypcji (wykupu) dostępu do publikowanych materiałów z wykorzystaniem wirtualnej waluty (np.~punktów za udział w badaniach).

% All appendices and extra material, if you have any.
\cleardoublepage\appendix%
\chapter{Informacje uzupełniające}
\label{Chapter10}

\section{Wkład poszczególnych osób do przedsięwzięcia}
\label{Chapter101}

Skład zespołu pracującego nad projektem został przedstawiony w tablicy \ref{tab:roster}.

\begin{table}[H]
\centering
\begin{tabular}{ | c | c | }
\hline
\textbf{Stanowisko} & \textbf{Osoba} \\ \hline
Założyciel projektu, klient & prof. Jerzy Nawrocki \\ \hline
Główny użytkownik & prof. Jerzy Nawrocki \\ \hline
Główny dostawca & Tomasz Sawicki \\ \hline
Dostawca od strony DRO & Tomasz Sawicki \\ \hline
Starszy konsultant & Sylwia Kopczyńska \\ \hline
Konsultant & Sylwia Kopczyńska \\ \hline
Kierownik projektu & inż.~Marcin Domański \\ \hline
Analityk/Architekt & inż.~Błażej Matuszczyk \\ \hline
Programiści & Krzysztof Marian Borowiak \\ 
 & Maciej Trojan \\ 
 & Krzysztof Urbaniak \\ 
 & Łukasz Wieczorek \\
\hline
\end{tabular}
\caption{Osoby związane z przedsięwzięciem}\label{tab:roster}
\end{table}

\noindent
Odpowiedzialność za utworzenie treści niniejszej pracy dyplomowej została przedstawiona poniżej:

\begin{description}
\item Krzysztof Marian Borowiak

\begin{itemize}
\item Edycja i dostosowanie szablonu pracy w środowisku \LaTeX
\item Redakcja całej pracy, włącznie z częściami pozostałych autorów
\item Pozyskanie, przetworzenie i zamieszczenie materiałów zewnętrznych
\item Pozyskanie, przetworzenie i zamieszczenie materiałów pochodzących od zespołu zarządzającego
\item Rozdział 1 -- Wprowadzenie
\item Rozdział 7 -- Zapewnianie jakości i konserwacja systemu
\item Rozdział 6.2.11 -- Testy jednostkowe i akceptacyjne
\item Rozdział 8 -- Wnioski - część własna
\item Rozdział 9 -- Zakończenie
\item Dodatki
\end{itemize}
\noindent

\item Maciej Trojan

\begin{itemize}
\item Rozdział 6.2.3; 6.2.4 -- Napotkane problemy i ich rozwiązania -- Inicjalizacja bazy danych; Inicjalizacja modułu
\item Rozdział 6.3.11 -- Użyte technologie -- JavaScript
\item Rozdział 8 -- Wnioski -- część własna
\end{itemize}
\noindent

\item Krzysztof Urbaniak

\begin{itemize}
\item Rozdział 6.2.5-10 -- Napotkane problemy i ich rozwiązania - Formularze; Role; Formater kursu; Tworzenie badania; Tworzenie ankiety; Hierarchia CSS
\item Rozdział 6.5 -- Interfejs
\item Rozdział 6.7 -- Powiązanie logiki z interfejsem
\item Rozdział 8 -- Wnioski - część własna
\end{itemize}
\noindent

\item Łukasz Wieczorek

\begin{itemize}
\item Rozdział 6.2.12 -- Mapowanie obiektowo-relacyjne
\item Rozdział 6.3.1-10 -- Użyte technologie - Moodle, PHP, PHPUnit, Selenium, PostgreSQL,Eclipse IDE, SVN, Redmine, JasperReports, JetBrains PhpStorm
\item Rozdział 6.6 -- Logika (back-end)
\item Rozdział 7.1.2 -- Testy jednostkowe
\item Rozdział 8 -- Wnioski - część własna
\end{itemize}
\noindent

\item Zespół zarządzający projektem (Marcin Domański, Błażej Matuszyk)

\begin{itemize}
\item Materiały wskazane w bibliografii\cite{Redmine:ProjDocs}, zastosowane jako baza dla rozdziałów 1-5
\item Część grafik (z odpowiednim odniesieniem w etykiecie)
\end{itemize}
\noindent

\end{description}
\noindent

Odpowiedzialność za część implementacyjną systemu została przedstawiona poniżej:

\begin{description}
\item Krzysztof Marian Borowiak

\begin{itemize}
\item Testy jednostkowe i akceptacyjne
\item Dokumentacja dla Użytkownika Końcowego (Administratora, Użytkownika)
\item Dokumentacja techniczna (raporty dot. funkcjonowania na platformach mobilnych oraz w różnych środowiskach)
\end{itemize}
\noindent

\item Maciej Trojan

\begin{itemize}
\item Interfejs użytkownika
\item Utworzenie Bazy Danych
\end{itemize}
\noindent

\item Krzysztof Urbaniak

\begin{itemize}
\item Interfejs użytkownika
\item Powiązanie interfejsu z logiką
\end{itemize}
\noindent

\item Łukasz Wieczorek

\begin{itemize}
\item Logika
\item Testy jednostkowe
\item System raportowania
\end{itemize}
\noindent

\end{description}

%\noindent
Serdeczne podziękowania należą się Promotorowi niniejszej pracy dyplomowej, dr inż. Bartoszowi Walterowi, który  wytrwale wspierał autorów przy realizacji ich zadań. Podobnie, podziękowania należą się prowadzącemu przedmiot ,,Pracownia inżynierska'', dr inż. Grzegorzowi Pawlakowi, aktywnie motywującego zespół do wytężonej pracy. \\

Dodatkowe podziękowania należą się zespołowi zarządzającemu, przygotowane przez który materiały, choć wymagające znaczącej redakcji, pomogły w utworzeniu niniejszej pracy dyplomowej. Autorzy niniejszej pracy dyplomowej chcieliby także podziękować opiekunom \textit{SDS}, w tym w szczególności mgr inż. Sylwii Kopczyńskiej, za niewyczerpaną wiarę w ich (w rozumieniu: zespołu programistów) możliwości.

\section{Wykaz użytych narzędzi i technologii}
\label{ChapterA4}

Numery w nawiasach w poniższej liście oznaczają numer wersji.

\begin{itemize}
\item Apache (2.2.22)
\item BASH (4.2.37)
\item Check Point's Linux SNX (800007027)
\item Chrome (24.0)
\item CLOC (1.56)
\item CRON (3.0)
\item Eclipse IDE (3.7.2)
\item FastStone Capture (5.3)
\item Git (1:1.7.10.4)
\item JasperReports Studio (1.3.2)
\item JasperReports Server (5.0.1)
\item Java (1.6.0\_38)
\item JavaScript
\item JetBrains PhpStorm (5.0.4)
\item Kazam Screencaster (1.0.6)
\item Meld (1.6.0)
\item Moodle (2.3.1)
\item Mozilla Firefox (18.0.1)
\item MySQL (14.14)
\item PHP (> 5.3)
\item PHPUnit (3.6.10)
\item psql (9.1.7)
\item PostgreSQL (9.1)
\item recordMyDesktop (0.3.8.1)
\item Redmine
\item Selenium IDE (1.10.0)
\item SSH (1:6.0)
\item SVN (1.7.5)
\item TexLive (20120611)
\item TexMaker (3.4)
\item VIM (7.3)
\item Zend PHP Developer Tools for Eclipse IDE (3.0.2)
\end{itemize}

\section{Zawartość płyty CD}

Do dokumentu załączono płytę CD o następującej zawartości:

\begin{itemize}
\item Dokumentacja systemu iQuest
\item Niniejszy dokument w formacie PDF
\item Pliki źródłowe systemu iQuest
\item Pliki źródłowe wykorzystywanej wersji Moodle
\end{itemize}
\chapter{Wygląd aplikacji}
\label{Chapterb1}

\begin{figure}[H]
\centering\includegraphics[width=0.9\textwidth]{figures/kb/W2-logowanie}
\caption{Logowanie do systemu \textit{iQuest} (wyk. Krzysztof Marian Borowiak)}\label{rys:Logowanie}
\end{figure}

\begin{figure}[H]
\centering\includegraphics[width=0.9\textwidth]{figures/kb/W2-stronaglowna}
\caption{Strona główna systemu \textit{iQuest} -- widok ankietera (wyk. Krzysztof Marian Borowiak)}\label{rys:StronaGlowna}
\end{figure}

\begin{figure}[H]
\centering\includegraphics[width=0.9\textwidth]{figures/kb/W2-dodawaniebadania}
\caption{Dodawanie badania w systemie \textit{iQuest} (wyk. Krzysztof Marian Borowiak)}\label{rys:DodawanieBadania}
\end{figure}

\begin{figure}[H]
\centering\includegraphics[width=0.9\textwidth]{figures/kb/W2-tworzenieankiety}
\caption{Tworzenie ankiety w systemie \textit{iQuest} -- wybór typu pytania (wyk. Krzysztof Marian Borowiak)}\label{rys:TworzenieAnkiety}
\end{figure}

\begin{figure}[H]
\centering\includegraphics[width=0.9\textwidth]{figures/kb/W2-tworzenieankiety2}
\caption{Tworzenie ankiety w systemie \textit{iQuest} -- modyfikacja pytania (wyk. Krzysztof Marian Borowiak)}\label{rys:TworzenieAnkiety2}
\end{figure}

\begin{figure}[H]
\centering\includegraphics[width=0.9\textwidth]{figures/kb/W2-wyborbadania}
\caption{Lista badań dostępna dla respondenta (wyk. Krzysztof Marian Borowiak)}\label{rys:WyborBadania}
\end{figure}

\begin{figure}[H]
\centering\includegraphics[width=0.9\textwidth]{figures/kb/W2-odpowiadanienanakiete}
\caption{Interfejs ankiety dla respondenta (wyk. Krzysztof Marian Borowiak)}\label{rys:OdpowiadanieNaAnkiete}
\end{figure}
%\chapter{Schemat bazy danych}
\label{Chapterc1}

\begin{landscape}
\begin{figure}[th]
\centering\includegraphics[height=\textheight, width=1.5\textwidth]{figures/iQuest_Database}
\caption{Diagram Bazy Danych}\label{rys:iQuest_DataBase}
\end{figure}
\end{landscape}

% Bibliography (books, articles) starts here.
\bibliographystyle{plain}{\raggedright\sloppy\small\bibliography{bibliography}}

% Colophon is a place where you should let others know about copyrights etc.
\ppcolophon

\end{document}