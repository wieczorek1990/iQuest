% Szkielet dla pracy pisanej w języku angielskim.

\documentclass[english,a4paper,oneside]{ppfcmthesis}

\usepackage[utf8]{inputenc}
\usepackage[OT4]{fontenc}

\author{Ignacy Iksiński}                              % Your name comes here
\title{A Few Words About the Nature~of~Things}        % Note how we protect the final title phrase from breaking
\ppsupervisor{prof.~dr hab.~inż.~Alojzy Wołodyjowski} % Your supervisor comes here.
\ppyear{2006}                                         % Year of final submission (not graduation!)

\begin{document}

% Front matter starts here
\frontmatter\pagestyle{empty}%
\maketitle\cleardoublepage%

% Blank info page for "karta dyplomowa"
\thispagestyle{empty}\vspace*{\fill}%
\begin{center}Tutaj przychodzi karta pracy dyplomowej;\\oryginał wstawiamy do wersji dla archiwum PP, w pozostałych kopiach wstawiamy ksero.\end{center}%
\vfill\cleardoublepage%

% Table of contents.
\pagenumbering{Roman}\pagestyle{ppfcmthesis}%
\tableofcontents* \cleardoublepage%

% Main content of your thesis starts here.
\mainmatter%

\chapter{Lorem Ipsum}

\section{Lorem}

Lorem ipsum dolor sit amet, consectetuer adipiscing elit. Nunc fringilla facilisis massa. Donec
ultrices dignissim quam. Phasellus vel erat quis diam tempor porttitor. Maecenas purus ligula,
varius id, suscipit in, congue quis, erat. Vivamus erat purus, laoreet rutrum, aliquet eleifend,
porta a, erat. Etiam id neque id nunc porttitor nonummy. Mauris pulvinar mi quis elit. Sed placerat.
Cras dictum neque vel odio. Sed a augue.

Aenean eu ligula. Nulla sit amet metus et risus vehicula elementum. Vivamus a eros. Etiam lorem
odio, iaculis ac, vulputate vitae, cursus a, tortor. Sed sagittis volutpat libero. Integer ornare,
dui nec vulputate auctor, ante mi lacinia ipsum, nec mollis ipsum libero at arcu. Suspendisse ut
ligula. Nunc metus. Proin mollis pellentesque ante. Donec dictum. Sed purus ante, placerat eget,
vestibulum in, lobortis ut, pede.

Sed blandit. Fusce blandit turpis id eros auctor dictum. Ut porttitor mollis magna. Nam iaculis sem
vel velit. Sed sollicitudin libero non quam. In hac habitasse platea dictumst. Praesent tristique,
justo vitae mollis fringilla, nisl ligula feugiat lacus, ac vestibulum mi turpis in nisi. Aenean
tristique fermentum erat. Integer consectetuer erat quis ante. Morbi gravida hendrerit nulla. Etiam
magna.

\subsection{Ipsum}

Curabitur adipiscing. Nam justo quam, dapibus eu, dignissim eu, hendrerit consequat, odio. Praesent
mi ligula, vulputate et, ullamcorper ut, pellentesque ut, massa. Suspendisse potenti. Morbi interdum
nisi vel justo. Aenean in tortor id mi dapibus porttitor. Nunc convallis. Nullam vitae felis.
Maecenas fringilla, lorem quis aliquam laoreet, elit justo tincidunt nisi, ac sollicitudin sapien
nulla non nisi. Etiam et mauris. Nunc scelerisque rhoncus leo.

Morbi posuere nisi id libero. Etiam tincidunt facilisis sem. Suspendisse vulputate, lectus quis
volutpat consectetuer, urna massa vestibulum tellus, nec vehicula dolor orci id odio. Nam faucibus
sem eu nisl ullamcorper lobortis. Sed quis ipsum. Morbi volutpat pellentesque erat. Quisque ipsum
velit, aliquet eu, imperdiet ut, commodo sed, est. Proin euismod porta enim. Mauris sit amet purus.
Nunc ac turpis. Nunc tellus magna, sagittis in, pellentesque vitae, semper et, sem. Vivamus et odio.
Suspendisse tincidunt mattis leo. Morbi pharetra. Quisque fermentum, lectus eget feugiat iaculis,
enim neque accumsan felis, id hendrerit enim erat et ante. Nunc blandit commodo est. Fusce venenatis
urna quis lorem. Aenean nisi. Vivamus ornare velit in magna imperdiet dapibus.

\subsection{Aliquam Venenatis}

Cras non enim eget mi aliquam venenatis. Duis posuere imperdiet quam. Maecenas egestas. In tempus
tortor quis pede. Aenean eget mauris sit amet ipsum scelerisque porttitor. Vestibulum nisl.
Vestibulum id velit eget tellus ultrices laoreet. Sed pellentesque, odio eget adipiscing commodo,
turpis lorem placerat lacus, et interdum quam mi quis odio. Nam et magna. Pellentesque vitae nisl.
In erat est, iaculis nec, sodales imperdiet, dictum at, eros. Nullam eget leo. Praesent nunc. Morbi
id erat. Pellentesque vel felis vitae est consectetuer interdum. Fusce venenatis. Nam rhoncus tortor
et sem. Sed ut sem aliquet urna congue suscipit. Nulla eget risus in purus molestie feugiat.
Phasellus non libero vel odio vehicula placerat.

Aenean velit. Vestibulum dui sem, scelerisque non, nonummy quis, accumsan eget, ipsum. Quisque nec
est et ipsum bibendum venenatis. Donec eget pede. Sed vitae ante et metus pulvinar fermentum. Nullam
scelerisque laoreet dui. Aliquam consectetuer. Quisque molestie eros a felis. Maecenas nec nisi.
Praesent mollis nisl sit amet pede. Phasellus iaculis elementum ante.

Fusce lacinia magna ut lacus. In aliquet diam eget augue. Curabitur blandit nisi sit amet nulla.
Donec nunc. Nullam pellentesque eleifend nisi. Proin ac elit. Fusce a neque. Nunc felis. Aenean
sodales. Etiam a risus non elit commodo rhoncus. Ut mi. Cras gravida neque nec diam. Proin consequat
vestibulum nunc. Aliquam mattis justo et purus. Sed feugiat.

Fusce fermentum. Nulla orci lectus, pellentesque sit amet, egestas sed, ornare a, erat. Vivamus ante
odio, auctor ac, elementum eget, feugiat sit amet, enim. Aliquam venenatis semper erat. Pellentesque
habitant morbi tristique senectus et netus et malesuada fames ac turpis egestas. Aliquam non tellus.
Mauris ornare massa iaculis magna. Morbi eu enim. Mauris ultrices. Vivamus dapibus aliquet nulla.
Quisque consectetuer.

\section{Pellentesque Habitant}

Pellentesque est nisl, scelerisque in, dictum vel, dapibus nec, lacus. Donec eu quam sed ligula
luctus volutpat. Donec ornare ornare purus. Mauris ligula eros, pulvinar nec, viverra in, tristique
quis, nunc. Pellentesque at diam. Integer tempus risus vitae lorem. Vestibulum sed lorem id elit
convallis rutrum. Quisque lobortis massa. Pellentesque habitant morbi tristique senectus et netus et
malesuada fames ac turpis egestas. Maecenas purus. Etiam hendrerit malesuada libero. Sed non mauris.

Pellentesque habitant morbi tristique senectus et netus et malesuada fames ac turpis egestas. Sed
ligula. Quisque nec augue. Suspendisse id lacus. Mauris a turpis. Curabitur interdum nulla tempus
justo. Sed nisl mauris, volutpat eleifend, suscipit in, ornare vitae, odio. Ut tempor, pede sed
volutpat ornare, mauris elit laoreet turpis, non posuere lorem nunc eu risus. Curabitur vulputate,
risus et auctor dapibus, ipsum nisl gravida nulla, vitae venenatis nibh libero id tellus. In
accumsan aliquet augue.

In consectetuer, magna in ornare eleifend, neque leo tempor nibh, quis viverra neque sapien
malesuada orci. Quisque eget magna. Integer tempus. Sed malesuada, orci a cursus posuere, justo
magna laoreet mauris, at pulvinar sapien velit ut nisi. Sed sed augue quis libero hendrerit feugiat.
Nam porta. In et massa. Curabitur blandit purus vitae risus luctus tristique. Sed ipsum. Morbi
tincidunt, dolor ut aliquam luctus, magna libero sagittis risus, eget dignissim risus lacus quis
magna. Donec tincidunt varius elit. Ut in velit. Curabitur dui velit, vestibulum in, tristique
nonummy, venenatis et, eros. Maecenas mi leo, placerat vel, congue vitae, semper ut, nunc. Lorem
ipsum dolor sit amet, consectetuer adipiscing elit. Mauris orci elit, hendrerit vestibulum,
vulputate ut, venenatis et, mauris. Curabitur vitae tortor vitae elit vulputate congue. Mauris felis
dui, lobortis vel, consectetuer id, cursus at, sapien.

Integer ut turpis. Duis odio ligula, lobortis ut, egestas a, porttitor id, ligula. Nunc vulputate,
ipsum quis varius aliquet, est eros sagittis urna, interdum imperdiet lorem ipsum tincidunt nibh.
Vivamus dictum porttitor diam. Lorem ipsum dolor sit amet, consectetuer adipiscing elit. Praesent at
nulla. Praesent tincidunt ipsum a velit. Fusce ut ipsum sed purus scelerisque nonummy. Cras
imperdiet velit eu nibh. Ut ultrices mauris et risus.




% All appendices and extra material, if you have any.
\cleardoublepage\appendix%
%\input{0a-appendix.tex}

% Bibliography (books, articles) starts here.
\bibliographystyle{alpha}{\raggedright\sloppy\small\bibliography{bibliography}}

% Colophon is a place where you should let others know about copyrights etc.
\ppcolophon

\end{document}
