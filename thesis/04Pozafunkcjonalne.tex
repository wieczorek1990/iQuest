\chapter{Wymagania pozafunkcjonalne}
\label{Chapter4}

\section{Wstęp}
\label{Chapter41}

W niniejszym rozdziale zostaną zaprezentowane i krótko opisane charakterystyki oraz wymagania pozafunkcjonalne obowiązujące dla systemu. Dodatkowo, podjęta zostaje tu próba weryfikacji, które wymagania udało się spełnić i jakie są perspektywy dalszego rozwoju projektu.

\section{Charakterystyki oprogramowania}
\label{Chapter42}

Projektując system, pod uwagę brane były następujące charakterystyki:

\begin{itemize}
\item{Funkcjonalna poprawność}
\item{Zużycie zasobów}
\item{Współistnienie}
\item{Interoperacyjność}
\item{Łatwość nauczenia się}
\item{Ochrona użytkownika przed błędami}
\item{Estetyka interfejsu Użytkownika}
\item{Dostępność personalna}
\item{Bezpieczeństwo (wolność od ryzyka)}
\item{Kompletność kontekstowa}
\item{Odporność na wady}
\item{Odtwarzalność}
\item{Integralność}
\item{Niezaprzeczalność}
\item{Łatwość instalacji}
\item{Łatwość adaptacji}
\item{Łatwość zamiany}
\item{Łatwość zmiany}
\item{Łatwość testowania}
\item{Poufność}
\item{Identyfikowalność}
\item{Autentyczność}
\item{Analizowalność}
\end{itemize}

Wymienione powyżej charakterystyki były rozpatrywane w fazach projektowania i implementacji z zastosowaniem różnych priorytetów. Dla przykładu, najważniejsza okazała się %DOKOŃCZYĆ%

cji.
Najważniejsza okazała się dokładność, gdyż system ma wspomagać ocenę jakości pracy naukowej, a zatem powinien udostępniać rzetelne informacje. Dużą uwagę zwrócono na fazę oczyszczania
danych, choć w wyniku napotkanych trudności, nie uzyskano tak dużej dokładności jaką zakładano.
Na bezpieczeństwo położono duży nacisk ze względu na wrażliwość danych i dostepu do nich –
zastosowano silny mechanizm uprawnień kontrolowany przez administratora. Odporność na błędy
154.3. Wymagania pozafunkcjonalne i ich weryfikacja 16
została ogólnie zapewniona przez zastosowanie obsługi wyjątków, częste testowanie systemu oraz
sprawdzanie poprawności danych.
Modułowość sprawiła, iż łatwe staje się przeprowadzanie zmian i poprawienie działania jednej
z części systemu. Poprzez dobre i wyczerpujące komentowanie kodu, zwłaszcza w miejscach o
większym stopniu złożoności, osiągnięto łatwiejszą możliwość analizy i wprowadzanie potrzebnych
korekt.
BIS-2 korzysta ze sprawdzonych i uznanych narzędzi omówionych w rozdziale 6. To sprawia,
iż zachowana została adaptowalność i przenośność systemu, chociażby dzięki wybrania języka programowania Java. Także współdziałanie z innymi aplikacjami nie powinno powodować żadnych
komplikacji. System zarządzania bazą danych PostgreSQL zapewnia narzędzia związane z odtwarzaniem danych po awarii, co można wykorzystać w systemie. Charakterystyka czasowa nie
była najważniejszym wymaganiem dla BIS-2, jednak dołożono wszelkich starań, aby wykonywanie
operacji przez aplikację było jak najszybsze. Najbardziej czasochłonnymi procesami okazały się:
pozyskiwanie danych o publikacjach z zewnętrznych serwisów oraz algorytm oczyszczania danych,
a zatem etapy w większości sytuacji niebezpośrednio odczuwalne przez użytkownika.

Każde z wymagań w ramach powyższych charakterystyk oprogramowania zostało określone za pomocą etykiety określającej jej ważność oraz trudność implementacji w formacie \textit{[ważność]/[trudność_implementacji]}. Wartości te wyrażone są w następującej skali:
\begin{itemize}
\item{H (ang.~\definicja{high}) -- wysoka;
\item{M (ang.~\definicja{medium}) -- średnia;
\item{L (ang.~\definicja{low}) -- niska.
\end{itemize}

Funkcjonalna poprawność
R1: Wszystkie wartości mają być prezentowane z dokładnością do 2 miejsc do przecinku. H/L
Charakterystyka czasowa
R2: Wyświetlanie ankiety powinno odbywać się z maksymalną prędkością nie większą iż 4 sekundy. H/L
R3: Generowanie raportów powinno odbywać się ze średnią prędkością nie większą iż 1 godzina. M/M
R54: Należy zdefiniować klasy operacji w zależności od czasu ich trwania (klasy: bez komunikatu potwierdzającego wykonanie, z potwierdzeniem wykonania, wykonywane na serwerze w tle) M/M
Zużycie zasobów
R4: System nie powinien wykorzystywać więcej serwerów niż jeden. M/L
R5: System powinien działać poprawnie przy nawet 10000 użytkowników zalogowanych jednocześnie. H/M
R55: Należy kolejkować zadania do wykonania przez system, które nie muszą zostać wykonane natychmiastowo. M/M
R57: Możliwość pracy dla ankietera i respondenta na średniej klasy laptopie (CPU 2GHz, RAM 4GB). H/L
Współistnienie
R6: Jedynym kryterium działania systemu ankiet (bez raportów) jest działająca przeglądarka. H/L
R7: System generowania raportów może być osobną aplikacją, działać na specjalnym potrzebnym do tego sprzęcie i oprogramowaniu. L/L
R58: Wszystkie potrzebne dane z zewnętrznych systemów przechowywać lokalnie i synchronizować . M/H
R59: Kolejkować e-maile w systemie, podczas gdy nie da się ich wysłać (awaria ePoczty). H/M
Interoperacyjność
R8: System ma wymieniać potrzebne dane z systemami uczelnianymi. eKonto, eDziekanat, ePoczta. Dane mają być aktualne. H/M
Łatwość nauczenia się
R9: Interfejs użytkownika (dla ankietowanych) powinien być całkowicie intuicyjny. H/M
R10: Interfejs użytkownika (generowanie raportów, tworzenie ankiet) może wymagać drobnego szkolenia. M/L
Ochrona użytkownika przed błędami
R11: Dodatkowe potwierdzenie chęci wykonania operacji nieodwracalnych (nawet dla administratora), albo możliwość przywrócenia usuniętych danych przez jakiś czas. H/M
R12: Dla dużych ankiet, zatwierdzenie odesłania jej przez ankietowanego. M/L
R30: Potwierdzenie przed rozesłaniem ankiet. M/L
R56: Lista operacji wykonywanych w tle. M/L
Estetyka Interfjesu Użytkownika
R13: Ma zachecać a nie zniechęcać, przyjazne, czytelne. H/M
Dostępność personalna
R14: Przewidzieć na poziomie architektury, możliwośc rozbudowy np. dla osób niedowidzących. L/M
Bezpieczeństwo (wolność od ryzyka)
R15: Dane trzymane w systemie muszą być tymi, uzyskanymi w ankietach H/L
R31: Projekt systemu należy poddać analizie z wykorzystaniem metody ATAM. M/M
Kompletność kontekstowa
R16: System ma działać dla: IE 7.0+, Firefox 15, Opera 12 H/M
R62: Przygotować raport jak system zachowuje się na platformach mobilnych. L/L
Dostępność techniczna
R17: System może mieć przerwę serwisową, ale wtedy specjalny ekran do kiedy będzie trwać. H/L
Odporność na wady
R18: Gdy nastąpi awaria innych systemów np. eKonta, należy poinformować użytkownika o błędzie i uniemożliwić mu dalsze działanie w systemie. H/L
Odtwarzalność
R19: Odtwarzanie całego systemu 3h. M/L
R20: Kopia zapasowa bazy danych, raz na dobę. H/L
R64: Dostępność instrukcji odtwarzania. H/L
R65: Wykonywanie operacji w transakcjach tam gdzie to możliwe. H/L
Integralność
R34: Baza danych powinna być chroniona przed nieuprawnionym dostępem [modyfikacją / usunięciem] w następujący sposób: username/pass. H/L
R35: System powinien być odporny na następujące próby nielegalnego dostępu: dostęp fizyczny do serwera. M/L
R36: Należy chronić, szyfrować dane przesyłane z i do systemu. M/L
Niezaprzeczalność
R22: System musi posiadać logi (zalogowanie w systemie, stworzenie/edycja/usunięcie/wysłanie/wypełnienie ankiety), aby móc udokumentować skąd pochodzą dane. H/L
Łatwość instalacji
R24: Należy utworzyć raport z łatwości adaptacji. Gdzie znajdują się adaptery. M/L
R41: System umożliwiać łatwą aktualizacje. Zakładamy, że wersja Moodle'a pozostaje bez zmian. H/H
Łatwość adaptacji
R42: System powinien współpracować z systemem BI (raportowanie). H/H
Łatwość zamiany
R43: System powinien umożliwiać wczytanie wszystkich danych z poprzedniej wersji. H/M
R44: Procedura wymiany oprogramowania powinna trwać nie dłużej niż 2 dni i odbywać się w następujący sposób: zgodność z instrukcją. M/L
R61: Brać pod uwagę podczas projektowania systemu, możliwość wielojęzyczności. M/M
Łatwość zmiany
R51: System powinien być przygotowany na wprowadzenie następujących zmian: nowe typy raportów, nowe typy pytań, modyfikacje interfejsów systemów zewnętrznych. M/M
Łatwość testowania
R52: „Atrapy” systemów zewnętrznych (w tym eKonto). M/M
R53: Należy umożliwić przełączanie systemu między trybem testowym i produkcyjnym. M/L
Interoperacyjność
R26: System ma pobierać dane z systemu eDziekanat w następujący sposób: SOAP. H/M
R27: System ma przesyłać dane o wiadomościach do wysłania do systemu ePoczta w następujący sposób: SOAP. H/M
R28: System ma pobierać dane do autoryzacji z systemu eKonto w następujący sposób: SOAP. H/M
R29: System ma przesyłać wyniki ankiet do systemu raportowania. (dodefiniować później!) H/M
Poufność
R32: Uwierzytelnianie i autoryzacja dla każdej operacji. H/L
Identyfikowalność
R38: System ma umożliwiać identyfikowanie podmiotów (osobno administratorów, ankieterów, respondentów) podejmujących działania: tworzenie ankiet, odpowiadanie w 
ankietach. H/M
Autentyczność
R39: Gdy student staje się absolwentem, należy umożliwić mu dalsze logowanie się do systemu bez użycia eKonta. H/M
Analizowalność
R45: Komentarze w kodzie źródłowym powinny być w języku angielskim. M/L
R46: Kod źródłowy sytemu powinien być utworzony zgodnie ze standardami Moodle'a. M/L
R47: System powinien zawierać testy jednostkowe. M/M
R48: System powinien rejestrować stack trace i rodzaj błędu (fatal, warning). H/L
R49: Wraz z kodem źródłowym systemu należy dostarczyć dokumentację: jak opisano w PID. H/M
R50: Logować niewłaściwe wywołania metod. H/L

%
%Poniżej są jeszcze stare charakterystyki oprogramowania (wtedy myśleliśmy, że to mądrze brzmi), kategorii wymagań pozafunkcjonalnych. Obecnie to się trochę zmieniło, zatem ta lista będzie znacznie bardziej rozbudowana.
%
%\begin{itemize}
%\item Dokładność (ang. \definicja{accuracy})
%\item Bezpieczeństwo (ang. \definicja{security})
%\item Odporność na błędy (ang. \definicja{fault tolerance})
%\item Odtwarzalność (ang. \definicja{recoverability})
%\item Charakterystyka czasowa (ang. \definicja{time behaviour})
%\item Łatwość analizowania (ang. \definicja{analysability})
%\item Łatwość zmian (ang. \definicja{changeability})
%\item Adaptowalność (ang. \definicja{adaptability})
%\item Instalowalność (ang. \definicja{installability})
%\item Współistnienie (ang. \definicja{co-existence})
%\item Zamienność (ang. \definicja{replaceability})
%\end{itemize}
%
%Tutaj piszemy, które podcharakterystyki były dla nas priorytetowe lub szczególnie ważne, a które mniej. Oczywiście, z uzasadnieniem. Piszemy również, jak zamierzaliśmy (lub to robiliśmy) dbać o to, aby wszystko było spełnione.
%
\section{Wymagania pozafunkcjonalne i ich weryfikacja}
\label{Chapter43}
%
%W tablicy \ref{tab:reqs} przedstawiono wymagania pozafunkcjonalne związane z systemem. W kolumnach \textbf{Priorytet} oraz \textbf{Trudność} określono poziomy przy pomocy następującej notacji:
%
%\begin{itemize}
%\item H -- wysoki priorytet lub poziom trudności
%\item M -- średni priorytet lub poziom trudności
%\item L -- niski priorytet lub poziom trudności
%\item N -- wymaganie oczywiste lub bardzo proste do spełnienia
%\end{itemize}
%\begin{table}[h]
%\centering
%\begin{tabular}{ | c | p{7cm} | c | c | }
%\hline
%\textbf{Podcharakterystyka} & \textbf{Wymaganie} & \textbf{Priorytet} & \textbf{Trudność} \\ %\hline
%Nazwa podcharakterystyki & Wymaganie 1 (przykład procentów: 90\%) & H & H \\ \hline
%Nazwa podcharakterystyki & Wymaganie 2 & M & H \\ \hline
%Nazwa podcharakterystyki & Wymaganie 3 & H & L \\ \hline
%Nazwa podcharakterystyki & Wymaganie 4 & N & L \\ \hline
%... & ... & ... & ... \\ \hline
%\end{tabular}
%\caption{Wymagania pozafunkcjonalne}\label{tab:reqs}
%\end{table}
%A tutaj piszemy o wszelkich problemach, wszelkich naszych wnioskach związanych ze spełnianiem wymagań pozafunkcjonalnych -- co się udało, co nie (i dlaczego). Tak, jakbyśmy opisywali nasze doświadczenia i problemy, z jakimi przyszło nam się zmagać i być może rozwiązać. Staramy się odnieść do najważniejszych wymagań (chyba że jest ich mało, wtedy do wszystkich).