\chapter{Wymagania pozafunkcjonalne}
\label{Chapter4}

\section{Wstęp}
\label{Chapter41}

W niniejszym rozdziale zostaną zaprezentowane i krótko opisane wymagania pozafunkcjonalne dotyczące systemu \textit{iQuest}. Dodatkowo, zostały one przydzielone do wybranych przez zespół zarządzający charakterystyk oprogramowania.

\section{Opis wymagań}
\label{Chapter42}

Wymagania pozafunkcjonalne przypisane do wyżej opisanych charakterystyk oprogramowania przedstawione zostały w tabeli \ref{tab:reqs}. Dodatkowo, określono za pomocą etykiety \emph{ważność (W)} oraz \emph{trudność implementacji (TI)} dla każdego z wymagań. Wartości w ramach tego oznaczenia wyrażone są w następującej skali:

\begin{itemize}
\item{H (ang.~\definicja{high}) -- wysoka}
\item{M (ang.~\definicja{medium}) -- średnia}
\item{L (ang.~\definicja{low}) -- niska}
\end{itemize}

Analizy, zakończonej etykietowaniem, dokonał zespół zarządzający, rozpatrując to wedle wytycznych przedstawionych w trakcie zajęć akademickich na 1. roku studiów II stopnia.

\newpage
\begin{center}
\begin{longtable}{ | p{4cm} | p{9cm} | c | c | }
\hline
\textbf{Charakterystyka} & \textbf{Wymaganie} & \textbf{W} & \textbf{TI} \\ \hline
%
Analizowalność & Komentarze w kodzie źródłowym powinny być w języku angielskim. & M & L \\ \hline
Analizowalność & Kod źródłowy sytemu powinien być utworzony zgodnie ze standardami \textit{Moodle}. & M & L \\ \hline
Analizowalność & System powinien zawierać testy jednostkowe. & M & M \\ \hline
Analizowalność & System powinien rejestrować stack trace i rodzaj błędu (fatal, warning). & H & L \\ \hline
Analizowalność & Wraz z kodem źródłowym systemu należy dostarczyć dokumentację. & H & M \\ \hline
Analizowalność & System powinien logować niewłaściwe wywołania metod. & H & L \\ \hline
%
Autentyczność & Gdy student staje się absolwentem, należy umożliwić mu dalsze logowanie się do systemu bez użycia \textit{eKonta}. & H & M \\ \hline
%
Bezpieczeństwo (wolność od ryzyka) & Dane (opinie respondentów) przechowywane w systemie muszą być uzyskiwane poprzez wbudowane mechanizmy ankietowania. & H & L \\ \hline
Bezpieczeństwo (wolność od ryzyka) & Projekt systemu należy poddać analizie z wykorzystaniem metody ATAM. & M & M \\ \hline
%
Charakterystyka czasowa & Wyświetlenie ankiety powinno trwać nie dłużej niż 4 sekundy. & H & L \\ \hline
Charakterystyka czasowa & Generowanie raportów powinno odbywać się ze średnio nie dłużej niż 1 godzinę. & M & M \\ \hline
Charakterystyka czasowa & Należy zdefiniować klasy operacji, w zależności od czasu ich trwania. Klasy:
\begin{itemize}
\item{bez komunikatu potwierdzającego wykonanie}
\item{z potwierdzeniem wykonania}
\item{wykonywane na serwerze w tle}
\end{itemize} & M & M \\ \hline
%
Dostępność personalna & Przewidzieć, na poziomie architektury, możliwość rozbudowy np. o interfejs dla osób niedowidzących. & L & M \\ \hline
%
Dostępność techniczna & System może mieć przerwę serwisową, lecz musi wówczas prezentować specjalny ekran informujący o czasie jej trwania. & H & L \\ \hline
%
Estetyka Interfejsu Użytkownika & Środowisko ma być przyjazne i czytelne dla użytkownika końcowego. & H & M \\ \hline
%
Funkcjonalna poprawność & Wszystkie wartości mają być prezentowane z dokładnością do 2~miejsc po przecinku. & H & L \\ \hline
%
Identyfikowalność & System ma umożliwiać identyfikowanie podmiotów (osobno: administratorów, ankieterów, respondentów), podejmujących konkretne działania: tworzenie ankiet, odpowiadanie w 
ankietach, itp. & H & M \\ \hline
%
Integralność & Baza danych powinna być chroniona przed nieuprawnionym dostępem [modyfikacją\slash usunięciem] w następujący sposób: logowanie za pomocą loginu i hasła. & H & L \\ \hline
Integralność & System powinien być odporny na następujące próby nielegalnego dostępu: nieuprawniony dostęp fizyczny do serwera. & M & L \\ \hline
Integralność & Należy chronić\slash szyfrować dane przesyłane z i do systemu. & M & L \\ \hline
%
Interoperacyjność & System ma wymieniać potrzebne dane z systemami uczelnianymi: \textit{eKonto}, \textit{eDziekanat}, \textit{ePoczta}. Dane mają być aktualne. & H & M \\ \hline
Interoperacyjność & System ma pobierać dane z systemu \textit{eDziekanat} w następujący sposób: \textit{SOAP}. & H & M \\ \hline
Interoperacyjność & System ma przesyłać dane o wiadomościach do wysłania do systemu \textit{ePoczta} w następujący sposób: \textit{SOAP}. & H & M \\ \hline
Interoperacyjność & System ma pobierać dane do autoryzacji z systemu \textit{eKonto} w następujący sposób: \textit{SOAP}. & H & M \\ \hline
Interoperacyjność & System ma przesyłać wyniki ankiet do systemu \textit{BI}\footnote{ang. \definicja{Business Intelligence}}. & H & M \\ \hline
%
Kompletność kontekstowa & System ma działać w przeglądarkach: \textit{IE 7.0+}, \textit{Firefox 15}, \textit{Opera 12}. & H & M \\ \hline
Kompletność kontekstowa & Należy przygotować raport jak system zachowuje się na platformach mobilnych. & L & L \\ \hline
%
Łatwość adaptacji & Należy utworzyć raport z łatwości adaptacji oraz gdzie znajdują się adaptery. & M & L \\ \hline
%
Łatwość instalacji & System musi umożliwiać łatwą aktualizacje, przy założeniu, że wersja platformy \textit{Moodle} pozostaje bez zmian. & H & H \\ \hline
%
Łatwość nauczenia się & Interfejs użytkownika (dla ankietowanych) powinien być całkowicie intuicyjny. & H & M \\ \hline
Łatwość nauczenia się & Interfejs użytkownika (dla ankietera) może wymagać nieznacznego doszkolenia obsługujących. & M & L \\ \hline
%
Łatwość testowania & ,,Atrapy'' (ang.~\definicja{mock}) systemów zewnętrznych (m.in. \textit{eKonto}, \textit{ePoczta}). & M & M \\ \hline
Łatwość zamiany & Należy umożliwić przełączanie systemu między trybem testowym i produkcyjnym. & M & L \\ \hline
%
Łatwość zamiany & System powinien umożliwiać wczytanie wszystkich danych z poprzedniej wersji. & H & M \\ \hline
Łatwość zamiany & Procedura wymiany oprogramowania powinna trwać nie dłużej niż 2 dni i odbywać się w następujący sposób: zgodność z instrukcją. & M & L \\ \hline
Łatwość zamiany & Podczas projektowania systemu, należy brać pod uwagę możliwość wprowadzenia wielojęzyczności interfejsu. & M & M \\ \hline
%
Łatwość zmiany & System powinien być przygotowany na wprowadzenie następujących zmian: nowe typy raportów, nowe typy pytań, modyfikacje interfejsów systemów zewnętrznych. & M & M \\ \hline
%
Niezaprzeczalność & System musi posiadać logi (zalogowanie w systemie, stworzenie\slash edycja\slash usunięcie\slash wysłanie\slash wypełnienie ankiety), aby móc udokumentować skąd pochodzą dane. & H & L \\ \hline
%
Ochrona użytkownika przed błędami & Dodatkowe potwierdzenie chęci wykonania operacji nieodwracalnych (nawet dla administratora), lub możliwość przywrócenia usuniętych danych przez jakiś czas. & H & M \\ \hline
Ochrona użytkownika przed błędami & Dla dużych ankiet, zatwierdzenie odesłania jej przez ankietowanego. & M & L \\ \hline
Ochrona użytkownika przed błędami & Potwierdzenie przed rozesłaniem ankiet. & M & L \\ \hline
Ochrona użytkownika przed błędami & Lista operacji wykonywanych w tle. & M & L \\ \hline
%
Odporność na wady & Gdy nastąpi awaria innych systemów np. \textit{eKonto}, należy poinformować użytkownika o błędzie i uniemożliwić mu dalsze działanie w systemie. & H & L \\ \hline
%
Odtwarzalność & Odtwarzanie całego systemu w czasie nieprzekraczającym 3h. & M & L \\ \hline
Odtwarzalność & Kopia zapasowa bazy danych wykonywana z częstotliwością raz na dobę. & H & L \\ \hline
Odtwarzalność & Dostępność instrukcji odtwarzania. & H & L \\ \hline
Odtwarzalność & Wykonywanie operacji w transakcjach tam, gdzie to możliwe. & H & L \\ \hline
%
Poufność & Uwierzytelnianie i autoryzacja dla każdej operacji. & M & L \\ \hline
%
Współistnienie & Jedynym kryterium działania systemu ankiet (bez raportów) jest działająca przeglądarka. & H & L \\ \hline
Współistnienie & System generowania raportów może być osobną aplikacją, działającą na specjalnych: sprzęcie i oprogramowaniu. & L & L \\ \hline
Współistnienie & Wszystkie potrzebne dane z zewnętrznych systemów należy przechowywać lokalnie i synchronizować. & M & H \\ \hline
Współistnienie & Należy kolejkować zadania (np.~e-maile) w systemie, na czas braku możliwości ich wysłania (np.~awaria \textit{ePoczty}). & H & M \\ \hline
%
Zużycie zasobów & System nie powinien wykorzystywać więcej serwerów \textit{HTTP} niż jeden. & M & L \\ \hline
Zużycie zasobów & System powinien działać poprawnie przy nawet 10.000 użytkowników zalogowanych jednocześnie. & H & M \\ \hline
Zużycie zasobów & Zadania, które nie muszą zostać wykonane natychmiastowo, powinny być kolejkowane przez system. & M & M \\ \hline
Zużycie zasobów & Możliwość pracy przez ankietera\slash respondenta na średniej klasy laptopie (CPU 2GHz, RAM 4GB). & H & L \\ \hline
%
\caption{Wymagania pozafunkcjonalne}\label{tab:reqs}
\end{longtable}
\end{center}

Zważywszy, że projekt ten jest istotny dla Uczelni, dołożono wszelkich starań, aby spełnić wszystkie wymagania pozafunkcjonalne. Zadanie to zostało w sporej części wykonane dzięki zastosowaniu sprawdzonego i znanego narzędzia w roli podstawy dla całego systemu. Mowa o platformie \textit{Moodle}\footnote{Więcej informacji w rozdziale~\ref{Chapter6}, poświęconym implementacji i zastosowanym przy niej technologiom.}. Użycie jej pozwoliło m.in.~na zapewnienie modułowości systemowi. Spełniona została także adaptatywność i przenośność systemu, oparta na uniwersalności języka \textit{PHP}, jak też szerokiej gamie obsługiwanych przez \textit{Moodle} systemów baz danych. \\

Zalety wykorzystania tak powszechnej technologii były -- przynajmniej w teorii -- znacznie szersze. \textit{Moodle} posiada swój własny standard kodowania, którego przy realizowaniu projektu starano się w pełni przestrzegać. To sprawia, że w połączeniu z wytycznymi \textit{DRO} (\definicja{Dział Rozwoju Oprogramowania Politechniki Poznańskiej}), dokumentacją \textit{Moodle} i dokumentacją systemu \textit{iQuest}, nakład pracy wymagany do wdrożenia się, celem rozwoju projektu, został zmniejszony. \\

W trakcie prac nad projektem, niemały nacisk stawiano na prawidłowe zarządzanie uprawnieniami użytkowników. Wykorzystano w tym celu mechanizmy dostępne w platformie \textit{Moodle}. Są one weryfikowane przy każdym działaniu podejmowanym przez użytkowników i zależą od przydzielonej im w systemie roli. Szczegóły tej kwestii swobodnie definiować może administracja. Ważnym jest, że osoby o roli ankietera mogą modyfikować i analizować jedynie utworzone przez siebie, lub udostępnione im bezpośrednio, badania. Podobnie działają uprawnienia przydzielone do roli respondenta -- pozwalając dokładnie jeden raz odpowiedzieć na pytania w ankietach skierowanych do grup docelowych, do których należy użytkownik.