\chapter{Informacje uzupełniające}
\label{Chapter10}

\section{Wkład poszczególnych osób do przedsięwzięcia}
\label{Chapter101}

Skład zespołu pracującego nad projektem został przedstawiony w tablicy \ref{tab:roster}.

\begin{table}[H]
\centering
\begin{tabular}{ | c | c | }
\hline
\textbf{Stanowisko} & \textbf{Osoba} \\ \hline
Założyciel projektu, klient & prof. Jerzy Nawrocki \\ \hline
Główny użytkownik & prof. Jerzy Nawrocki \\ \hline
Główny dostawca & Tomasz Sawicki \\ \hline
Dostawca od strony DRO & Tomasz Sawicki \\ \hline
Starszy konsultant & Sylwia Kopczyńska \\ \hline
Konsultant & Sylwia Kopczyńska \\ \hline
Kierownik projektu & inż.~Marcin Domański \\ \hline
Analityk/Architekt & inż.~Błażej Matuszczyk \\ \hline
Programiści & Krzysztof Marian Borowiak \\ 
 & Maciej Trojan \\ 
 & Krzysztof Urbaniak \\ 
 & Łukasz Wieczorek \\
\hline
\end{tabular}
\caption{Osoby związane z przedsięwzięciem}\label{tab:roster}
\end{table}

\noindent
Odpowiedzialność za utworzenie treści niniejszej pracy dyplomowej została przedstawiona poniżej:

\begin{description}
\item Krzysztof Marian Borowiak

\begin{itemize}
\item Edycja i dostosowanie szablonu pracy w środowisku \LaTeX
\item Redakcja całej pracy, włącznie z częściami pozostałych autorów
\item Pozyskanie, przetworzenie i zamieszczenie materiałów zewnętrznych
\item Pozyskanie, przetworzenie i zamieszczenie materiałów pochodzących od zespołu zarządzającego
\item Rozdział 1 - Wprowadzenie
\item Rozdział 7 - Zapewnianie jakości i konserwacja systemu
\item Rozdział 6.2.11 - Testy jednostkowe i akceptacyjne
\item Rozdział 8 - Wnioski - część własna
\item Rozdział 9 - Zakończenie
\item Dodatki
\end{itemize}
\noindent

\item Maciej Trojan

\begin{itemize}
\item Rozdział 6.2.3; 6.2.4 - Napotkane problemy i ich rozwiązania - Inicjalizacja bazy danych; Inicjalizacja modułu
\item Rozdział 6.3.11 - Użyte technologie - JavaScript
\item Rozdział 8 - Wnioski - część własna
\end{itemize}
\noindent

\item Krzysztof Urbaniak

\begin{itemize}
\item Rozdział 6.2.5-10 - Napotkane problemy i ich rozwiązania - Formularze; Role; Formater kursu; Tworzenie badania; Tworzenie ankiety; Hierarchia CSS
\item Rozdział 6.5 - Interfejs
\item Rozdział 6.7 - Powiązanie logiki z interfejsem
\item Rozdział 8 - Wnioski - część własna
\end{itemize}
\noindent

\item Łukasz Wieczorek

\begin{itemize}
\item Rozdział 6.3.1-10 - Użyte technologie - Moodle, PHP, PHPUnit, Selenium, PostgreSQL,Eclipse IDE, SVN, Redmine, JasperReports, JetBrains PhpStorm
\item Rozdział 8 - Wnioski - część własna
\end{itemize}
\noindent

\item Zespół zarządzający projektem (Marcin Domański, Błażej Matuszyk)

\begin{itemize}
\item Materiały wskazane w bibliografii\cite{Redmine:ProjDocs}, zastosowane jako baza dla rozdziałów 1-5
\item Część grafik (z odpowiednim odniesieniem w etykiecie)
\end{itemize}
\noindent

\end{description}
\noindent

Odpowiedzialność za część implementacyjną systemu została przedstawiona poniżej:

\begin{description}
\item Krzysztof Marian Borowiak

\begin{itemize}
\item Testy jednostkowe (wydanie 1)
\item Testy akceptacyjne
\item Dokumentacja dla Użytkownika Końcowego (Administratora, Użytkownika)
\item Raport funkcjonowania na różnych platformach mobilnych
\item Raport funkcjonowania w różnych przeglądarkach internetowych
\end{itemize}
\noindent

\item Maciej Trojan

\begin{itemize}
\item Interfejs użytkownika
\item Obsługa Bazy Danych
\end{itemize}
\noindent

\item Krzysztof Urbaniak

\begin{itemize}
\item Interfejs użytkownika
\item Powiązanie interfejsu z logiką
\item Obsługa Bazy Danych
\end{itemize}
\noindent

\item Łukasz Wieczorek

\begin{itemize}
\item Logika
\item Testy jednostkowe (wydanie 2)
\item System raportowania
\item Obsługa Bazy Danych
\end{itemize}
\noindent

\end{description}

%\noindent
Serdeczne podziękowania należą się zespołowi zarządzającemu, przygotowane przez który materiały, choć wymagające znaczącej redakcji, pomogły w utworzeniu niniejszej pracy dyplomowej.

\section{Wykaz użytych narzędzi i technologii}
\label{ChapterA4}

\begin{itemize}
\item Apache (2.2.22)
\item bash (4.2.37)
\item Check Point's Linux SNX (800007027)
\item Chrome (24.0)
\item cron (3.0)
\item Eclipse IDE (3.7.2)
\item FastStone Capture (5.3)
\item Git (1:1.7.10.4)
\item JasperReports Studio (1.3.2)
\item JasperReports Server (5.0.1)
\item Java (1.6.0\_38)
\item JavaScript
\item JetBrains PhpStorm (5.0.4)
\item Kazam Screencaster (1.0.6)
\item meld (1.6.0)
\item Moodle (2.3.1)
\item Mozilla Firefox (18.0.1)
\item MySQL (14.14)
\item PHP (> 5.3)
\item PHPUnit (3.6.10)
\item psql (9.1.7)
\item PostgreSQL (9.1)
\item recordMyDesktop (0.3.8.1)
\item Redmine
\item Selenium IDE (1.10.0)
\item SSH (1:6.0)
\item SVN (1.7.5)
\item TexLive (20120611)
\item TexMaker (3.4)
\item Zend PHP Developer Tools for Eclipse IDE (3.0.2)
\end{itemize}

\section{Zawartość płyty CD}

Do dokumentu załączono płytę CD o następującej zawartości:

\begin{itemize}
\item Dokumentacja systemu iQuest
\item Niniejszy dokument w formacie PDF
\item Pliki źródłowe systemu iQuest
\item Pliki źródłowe wykorzystywanej wersji Moodle
\end{itemize}