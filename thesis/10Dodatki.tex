\chapter{Informacje uzupełniające}
\label{Chapter10}

\section{Wkład poszczególnych osób w przedsięwzięcie}
\label{Chapter101}

Skład zespołu pracującego nad projektem został przedstawiony w tablicy \ref{tab:roster}.

\begin{table}[H]
\centering
\begin{tabular}{ | c | c | }
\hline
\textbf{Stanowisko} & \textbf{Osoba} \\ \hline
Założyciel projektu, klient & prof. Jerzy Nawrocki \\ \hline
Główny użytkownik & prof. Jerzy Nawrocki \\ \hline
Główny dostawca & Tomasz Sawicki \\ \hline
Dostawca od strony DRO & Tomasz Sawicki \\ \hline
Starszy konsultant & Sylwia Kopczyńska \\ \hline
Konsultant & Sylwia Kopczyńska \\ \hline
Kierownik projektu & inż.~Marcin Domański \\ \hline
Analityk/Architekt & inż.~Błażej Matuszczyk \\ \hline
Programiści & Krzysztof Marian Borowiak \\ 
 & Maciej Trojan \\ 
 & Krzysztof Urbaniak \\ 
 & Łukasz Wieczorek \\
\hline
\end{tabular}
\caption{Osoby związane z przedsięwzięciem}\label{tab:roster}
\end{table}

\noindent
Odpowiedzialność za utworzenie treści niniejszej pracy dyplomowej została przedstawiona poniżej:

\begin{description}
\item Krzysztof Marian Borowiak

\begin{itemize}
\item Edycja i dostosowanie szablonu pracy w środowisku \LaTeX
\item Redakcja całej pracy, włącznie z częściami współautorów
\item Pozyskanie, przetworzenie i zamieszczenie materiałów zewnętrznych
\item Pozyskanie, przetworzenie i zamieszczenie materiałów pochodzących od zespołu zarządzającego
\item Rozdział 1 -- Wprowadzenie
\item Rozdział 7 -- Zapewnianie jakości i konserwacja systemu
\item Rozdział 6.2.11 -- Testy jednostkowe i akceptacyjne
\item Rozdział 8 -- Wnioski - część własna
\item Rozdział 9 -- Zakończenie
\item Dodatki
\item Zrzuty ekranowe
\end{itemize}
\noindent

\item Maciej Trojan

\begin{itemize}
\item Rozdział 6.2.3; 6.2.4 -- Napotkane problemy i ich rozwiązania -- Inicjalizacja bazy danych; Inicjalizacja modułu
\item Rozdział 6.3.11 -- Użyte technologie -- JavaScript
\item Rozdział 8 -- Wnioski -- część własna
\end{itemize}
\noindent

\item Krzysztof Urbaniak

\begin{itemize}
\item Rozdział 6.2.5-10 -- Napotkane problemy i ich rozwiązania - Formularze; Role; Formater kursu; Tworzenie badania; Tworzenie ankiety; Hierarchia CSS
\item Rozdział 6.5 -- Interfejs
\item Rozdział 6.7 -- Powiązanie logiki z interfejsem
\item Rozdział 8 -- Wnioski - część własna
\end{itemize}
\noindent

\item Łukasz Wieczorek

\begin{itemize}
\item Rozdział 6.2.12 -- Mapowanie obiektowo-relacyjne
\item Rozdział 6.3.1-10 -- Użyte technologie - Moodle, PHP, PHPUnit, Selenium, PostgreSQL,Eclipse IDE, SVN, Redmine, JasperReports, JetBrains PhpStorm
\item Rozdział 6.6 -- Logika (back-end)
\item Rozdział 7.1.2 -- Testy jednostkowe
\item Rozdział 8 -- Wnioski - część własna
\item Część grafik (z odpowiednim odniesieniem w etykiecie)
\end{itemize}
\noindent

\item Zespół zarządzający projektem (Marcin Domański, Błażej Matuszyk)

\begin{itemize}
\item Materiały\cite{Redmine:ProjDocs}, zastosowane jako podstawa dla rozdziałów 1-5
\item Część grafik (z odpowiednim odniesieniem w etykiecie)
\end{itemize}
\noindent

\end{description}
\noindent

Odpowiedzialność za część implementacyjną systemu została przedstawiona poniżej:

\begin{description}
\item Krzysztof Marian Borowiak

\begin{itemize}
\item Testy jednostkowe i akceptacyjne
\item Dokumentacja dla Użytkownika Końcowego (Administratora, Użytkownika)
\item Dokumentacja techniczna (raporty dot. funkcjonowania na platformach mobilnych oraz w różnych środowiskach)
\end{itemize}
\noindent

\item Maciej Trojan

\begin{itemize}
\item Interfejs użytkownika
\item Utworzenie Bazy Danych
\end{itemize}
\noindent

\item Krzysztof Urbaniak

\begin{itemize}
\item Interfejs użytkownika
\item Powiązanie interfejsu z logiką
\end{itemize}
\noindent

\item Łukasz Wieczorek

\begin{itemize}
\item Logika
\item Testy jednostkowe
\item System raportowania
\end{itemize}
\noindent

\end{description}

%\noindent
Autorzy niniejszej pracy dyplomowej inżynierskiej składają serdeczne podziękowania Promotorowi, dr. inż. Bartoszowi Walterowi, który wytrwale wspierał ich przy realizacji zadań związanych z projektem, oraz dr. inż. Grzegorzowi Pawlakowi, prowadzącemu przedmiot ,,Pracownia inżynierska'', za aktywne motywowanie ich do wytężonej pracy. \\

Podziękowania należą się także zespołowi zarządzającemu oraz opiekunom \textit{SDS}, w tym w szczególności mgr inż. Sylwii Kopczyńskiej, za przygotowanie wymaganych materiałów źródłowych oraz niewyczerpaną wiarę w możliwości zespołu programistów.

\section{Wykaz użytych narzędzi i technologii}
\label{ChapterA4}

Numery w nawiasach w poniższej liście oznaczają numer wersji.

\begin{itemize}
\item Apache (2.2.22)
\item BASH (4.2.37)
\item Check Point's Linux SNX (800007027)
\item Chrome (24.0)
\item CLOC (1.56)
\item CRON (3.0)
\item Eclipse IDE (3.7.2)
\item FastStone Capture (5.3)
\item Git (1:1.7.10.4)
\item JasperReports Studio (1.3.2)
\item JasperReports Server (5.0.1)
\item Java (1.6.0\_38)
\item JavaScript
\item JetBrains PhpStorm (5.0.4)
\item Kazam Screencaster (1.0.6)
\item Meld (1.6.0)
\item Moodle (2.3.1)
\item Mozilla Firefox (18.0.1)
\item MySQL (14.14)
\item PHP (> 5.3)
\item PHPUnit (3.6.10)
\item psql (9.1.7)
\item PostgreSQL (9.1)
\item recordMyDesktop (0.3.8.1)
\item Redmine
\item Selenium IDE (1.10.0)
\item SSH (1:6.0)
\item SVN (1.7.5)
\item TexLive (20120611)
\item TexMaker (3.4)
\item VIM (7.3)
\item Zend PHP Developer Tools for Eclipse IDE (3.0.2)
\end{itemize}

\section{Zawartość płyty CD}

Do dokumentu załączono płytę CD o następującej zawartości:

\begin{itemize}
\item Dokumentacja systemu iQuest
\item Niniejszy dokument w formacie PDF
\item Pliki źródłowe systemu iQuest
\item Pliki źródłowe wykorzystywanej wersji Moodle
\end{itemize}