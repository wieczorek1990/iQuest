\chapter{Informacje uzupełniające}
\label{Chapter10}

\section{Wkład poszczególnych osób do przedsięwzięcia}
\label{Chapter101}

Skład zespołu pracującego nad projektem został przedstawiony w tablicy \ref{tab:roster}.

\begin{table}[H]
\centering
\begin{tabular}{ | c | c | }
\hline
\textbf{Stanowisko} & \textbf{Osoba} \\ \hline
Założyciel projektu, klient & prof. Jerzy Nawrocki \\ \hline
Główny użytkownik & prof. Jerzy Nawrocki \\ \hline
Główny dostawca & Tomasz Sawicki \\ \hline
Dostawca od strony DRO & Tomasz Sawicki \\ \hline
Starszy konsultant & Sylwia Kopczyńska \\ \hline
Konsultant & Sylwia Kopczyńska \\ \hline
Kierownik projektu & inż.~Marcin Domański \\ \hline
Analityk/Architekt & inż.~Błażej Matuszczyk \\ \hline
Programiści & Krzysztof Marian Borowiak \\ 
 & Maciej Trojan \\ 
 & Krzysztof Urbaniak \\ 
 & Łukasz Wieczorek \\
\hline
\end{tabular}
\caption{Osoby związane z przedsięwzięciem}\label{tab:roster}
\end{table}

\noindent
%Teraz bardzo ważna rzecz -- w tym miejscu piszecie, co kto przygotowywał w tekście pracy inżynierskiej. Prawdopodobnie tutaj będziecie musieli wymienić, które rozdziały został dla Was przygotowane przez kierownika projektu, analityka, architekta lub inną osobę. Oczywiście, piszecie też, za które części dokumentu Wy jesteście odpowiedzialni. To jest ważna, aby ta część była tutaj precyzyjnie przygotowana -- takie są wymogi uczelni oraz też uwzględnienia pracy innych osób.
%Materiały pozyskane od zespołu zarządzającego - \cite{Redmine:ProjDocs}.

\noindent

Odpowiedzialność za część implementacyjną systemu została przedstawiona poniżej:

\begin{description}
\item Krzysztof Marian Borowiak

\begin{itemize}
\item Testy jednostkowe
\item Testy akceptacyjne
\item Dokumentacja dla Użytkownika Końcowego
\end{itemize}
\noindent

\item Maciej Trojan

\begin{itemize}
\item Interfejs użytkownika
\item Powiązanie interfejsu z back-endem
\item Obsługa Bazy Danych
\end{itemize}
\noindent

\item Krzysztof Urbaniak

\begin{itemize}
\item Interfejs użytkownika
\item Powiązanie interfejsu z back-endem
\item Obsługa Bazy Danych
\end{itemize}
\noindent

\item Łukasz Wieczorek

\begin{itemize}
\item Logika (back-end)
\item Powiązanie interfejsu z back-endem
\end{itemize}
\noindent

\end{description}

%\noindent
%Ewentualne podziękowania dla innych osób, które Wam pomagały, mowy dziękczynne, itd.

\section{Wykaz użytych narzędzi}
\label{Chapter104}

%Wprawdzie jest odpowiedni podrozdział w rozdziale \ref{Chapter5}, ale tutaj można wymienić nawet małe narzędzia i biblioteki, które wykorzystywaliście (np.~narzędzie do robienia makiet interfejsu) i które warto wymienić, także dla przyszłych roczników (można też dać linki).

\section{Zawartość płyty CD}

Do dokumentu załączono płytę CD o następującej zawartości:

\begin{itemize}
\item Zawartość 1
\item Zawartość 2
\item Zawartość 3
\item ...
\end{itemize}