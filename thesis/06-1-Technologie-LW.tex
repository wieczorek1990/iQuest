\subsection{\textit{Moodle}}
\label{Chapter631}

\textit{Moodle} (ang. \definicja{Modular Object-Oriented Dynamic Learning Environment}) -- stanowi podstawę systemu \textit{iQuest}. Jest to popularna (ponad 63 miliony użytkowników) platforma e-learningowa o otwartym kodzie źródłowym, napisana w języku \emph{PHP}. Wyboru dokonano ze względu na kilka czynników:
\begin{itemize}
\item{Propozycję Kierownika Projektu oraz następującą po niej decyzję Architekta, wynikającą z faktu, iż \textit{Moodle} posiada już implementację wielu wymaganych w \textit{iQuest} mechanizmów, jak np. konta użytkowników, system ról i uprawnień.}
\item{Oczekiwania Klienta, wynikające z popularności platformy \textit{Moodle} wśród systemów uczelnianych.}
\item{Modułowość \textit{Moodle}, umożliwiającą pisanie rozszerzeń.}
\end{itemize}

\subsection{\textit{PHP}}
\label{Chapter632}

\textit{PHP} -- platforma \textit{Moodle} jest napisana właśnie w tym języku programowania. Z tego względu, jest to technologia zastosowana w większości rozszerzeń utworzonych przez zespół textit{iQuest}, korzystających z interfejsów programowania aplikacji tej platformy. Ponadto \textit{PHP} jest jednym z najpopularniejszych języków programowania aplikacji internetowych, posiada doskonałą dokumentację\footnote{\cite{Man:PHP}} oraz jest cały czas rozwijany.

Dodatkowo, język \textit{PHP} jest dość przyjazny dla programisty, gdy chodzi o komunikację z bazą danych. Przy realizacji zadań z tym związanych, odnoszono się zarówno do wspomnianej wyżej dokumentacji, jak też do literatury fachowej, w tym ,,PHP i MySQL - Księga przykładów'' autorstwa E. Quigley oraz M. Gargenta\footnote{\cite{EQMG:PiMKp07}}.

\subsection{\textit{PHPUnit}}
\label{Chapter633}

Ze względu na fakt, iż programiści \textit{Moodle'a} wykonują testy jednostkowe kodu wykorzystując do tego celu \textit{PHPUnit}, zdecydowano się skorzystać z przygotowanego przez nich oprogramowania. \textit{Moodle} udostępnia dwie klasy do testowania -- \textit{basic\_testcase} i \textit{advanced\_testcase}, przy czym druga wymieniona służy do testów, które wchodzą w interakcję z bazą danych. Korzystanie z tych klas dodatkowo uprasza fakt istnienia świetnej dokumentacji technicznej\footnote{\cite{Man:PHPUnit}}.

\subsection{\textit{Selenium}}
\label{Chapter634}

\textit{Selenium} -- szybko rozwijający się zestaw narzędzi do testów akceptacyjnych. Był to naturalny wybór zwłaszcza, że zostało ono przybliżone programistom na zajęciach z Inżynierii Oprogramowania w trakcie toku studiów. \textit{Selenium} składa się m.in. z następującego oprogramowania\footnote{\cite{Man:Selenium}}:
\begin{itemize}
\item{\textit{Selenium IDE} -- zintegrowane środowisko programistyczne dla skryptów \textit{Selenium} -- zaimplementowane jako rozszerzenie dla przeglądarki internetowej \textit{Firefox}. Pozwala na: nagrywanie i odtwarzanie sekwencji kroków, wykonywanych podczas pracy z przeglądarką, eksport skryptów do kodu języków programowania (np. \textit{Java}).}
\item{\textit{Selenium Client Drivers} (\textit{Java}) -- sterownik klienta dla języka \textit{Java}, pozwalający na wykonywanie skryptów \textit{Selenium} z poziomu języka \textit{Java}},
\item{\textit{HtmlUnit Driver} -- Implementacja klasy \textit{WebDriver}, która emuluje zachowanie przeglądarki. Pozwala na uruchamianie skryptów \textit{Selenium} bez korzystania z przeglądarki internetowej.}
\end{itemize}

\subsection{\textit{PostgreSQL}}
\label{Chapter635}

System zarządzania bazą danych \textit{PostgreSQL} został wybrany ze względu na wymaganie pozafunkcjonalne -- pracownicy \textit{Działu Rozwoju Oprogramowania Politechniki Poznańskiej} korzystają z tej właśnie bazy danych. Jest to baza danych o otwartym kodzie źródłowym, zgodna ze standardami, ciągle rozwijana, wysoce konfigurowalna.

\subsection{\textit{Eclipse IDE}}
\label{Chapter636}

Wybór \textit{Eclipse IDE} jako stosowanego dla projektu \textit{iQuest} zintegrowanego środowiska programistycznego wynika z faktu, iż oprogramowanie to jest dostępne za darmo. Dodatkową zaletą \textit{Eclipse} jest modularność tego rozwiązania, dzięki czemu dostępny jest w nim dodatek \textit{PHP Development Tools}, upraszczający pracę z technologią \textit{PHP}. Udostępnia m.in. narzędzia do analizy poprawności składniowej pisanego kodu, formatery kodu, wyszukiwanie fraz w wielu plikach, kontekstowe podpowiedzi i nawigację.

\subsection{\textit{SVN}}
\label{Chapter637}

\textit{Subversion} został wybrany jako podstawowy system kontroli wersji ze względu na wymagania pozafunkcjonalne. Zespół eksploatacji, który docelowo przejmie zarządzanie artefaktami związanymi z projektem, wykorzystuje właśnie \textit{SVN}. Główne funkcjonalności tego systemu to: atomowe publikowanie zmian, historia operacji na plikach (zmiana nazwy, skopiowanie, przeniesienie, modyfikacja, usunięcie), wersjonowanie plików i folderów, łatwy dostęp do informacji o zmianach.

\subsection{\textit{Redmine}}
\label{Chapter638}

Systemu zarządzania projektami \textit{Redmine} wykorzystywany był od samego początku istnienia projektu. Jest to narzędzie bardzo przydatne w wymianie informacji pomiędzy członkami zespołu, integrujące się m.in. z repozytorium kodu, bazą wiedzy o projekcie, listą zagadnień, forum.  Technologia ta została narzucona, ze względu na sposób organizacji pracy w \textit{Software Development Studio} na Politechnice Poznańskiej.

\subsection{\textit{JasperReports}}
\label{Chapter639}

Ze względu na wymagania pozafunkcjonalne, zdecydowano się skorzystać z mechanizmów raportowania oferowanych przez \textit{JasperReports}. Jest to najbardziej popularny silnik raportowania o otwartym kodzie źródłowym (wersja Community). Pozwala na generację raportów, których treść jest określona z dokładnością co do piksela. Generowane raporty można eksportować do popularnych formatów dokumentów, np. \textit{HTML}, \textit{PDF}, \textit{Excel}, \textit{Word}.

\subsection{\textit{JetBrains PhpStorm}}
\label{Chapter63a}

\textit{PhpStorm} jest komercyjnym \textit{IDE} dla języka \textit{PHP} tworzonym przez firmę \textit{JetBrains}, dostępnym dla studentów na licencji edukacyjnej. Dla celów niniejszej pracy dyplomowej, wykorzystano funkcjonalność odpowiedzialną za tworzenie diagramów UML z kodu. Wygenerowane w trybie \textit{Organic} diagramy można zobaczyć w dalszej części pracy (schematy \ref{rys:back-end1}., \ref{rys:back-end2}., \ref{fig:tests1}., \ref{fig:tests2}., zamieszczone w rozdziałach \ref{Chapter6}. i \ref{Chapter7}.).

\subsection{\textit{HTML} oraz \textit{CSS}}
\label{Chapter63b}
Po stronie użytkownika, system \textit{iQuest} prezentowany jest za pośrednictwem interpretowanego przez jego przeglądarkę internetową kodu w języku \textit{HTML}. Jego wygląd, wraz z umiejscowieniem elementów, określa natomiast arkusz stylów \textit{CSS}. Realizując zadania związane z tymi technologiami, odnoszono się do profesjonalnej literatury branżowej\footnote{M.in. ,,Wstępu do CSS3 i HTML5'' autorstwa Bartosza Danowskiego\cite{BD:WdCiH11}}.