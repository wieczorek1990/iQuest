\subsection{Moodle}
\emph{Moodle} (roz. \textit{Modular Object-Oriented Dynamic Learning Environment}) -- stanowi podstawę systemu \textit{iQuest}. Wyboru dokonano ze względu na kilka czynników:
\begin{itemize}
\item{Propozycję Architekta, wynikającą z faktu, iż Moodle posiada już implementację wielu wymaganych w \textit{iQuest} mechanizmów.}
\item{Oczekiwania Klienta, wynikające z popularności platformy Moodle wśród systemów uczelnianych.}
\end{itemize}

\subsection{PHP}
\emph{PHP} -- platforma Moodle opiera się właśnie na tym języku programowania. Z tego względu, jest to też technologia zastosowana w większości rozszerzeń utworzonych przez zespół /textit{iQuest}, korzystających z wielu interfejsów programowania aplikacji tej platformy.

\subsection{PHPUnit}
Ze względu na fakt, iż programiści \textit{Moodle'a} wykonują testy jednostkowe kodu wykorzystując do tego celu \emph{PHPUnit}, zdecydowano o wzorowaniu się na tym działaniu. \textit{Moodle} udostępnia dwie klasy do testowania -- \textit{basic\_testcase} i \textit{advanced\_testcase}, przy czym ta druga służy do testów, które wchodzą w interakcję z bazą danych.

\subsection{Selenium}
\emph{Selenium} -- szybko rozwijające się narzędzie do testów akceptacyjnych. Był to naturalny wybór zwłaszcza, że zostało ono przybliżone zespołowi \textit{iQuest} na zajęciach z Inżynierii Oprogramowania w trakcie toku studiów.

\subsection{PostgreSQL}
System zarządzania bazą danych \emph{PostgreSQL} został wybrany ze względu na wymagania pozafunkcjonalne.

\subsection{Eclipse IDE}
\label{Chapter621}

Wybór \emph{Eclipse IDE} jako stosowanego dla projektu \textit{iQuest} zintegrowanego środowiska programistycznego wynika z faktu, iż oprogramowanie to jest dostępne za darmo. Dodatkową zaletą \textit{Eclipse} jest modularność tego rozwiązania, dzięki czemu dostępny jest w nim dodatek \emph{PHP Development Tools}, znacząco upraszczający pracę z technologią PHP.

\subsection{SVN}
\label{Chapter632}

\emph{Subversion} został wybrany jako podstawowy system zarządzania treścią ze względu na wymagania pozafunkcjonalne. Dodatkową zaletą jego użycia jest fakt, że zespół eksploatacji, który docelowo przejmie zarządzanie artefaktami związanymi z projektem, wykorzystuje właśnie \textit{SVN}.

\subsection{Redmine}
\label{Chapter633}

Systemu zarządzania projektami \emph{Redmine} wykorzystywany był od samego początku istnienia projektu. Jest to narzędzie bardzo przydatne w wymianie informacji pomiędzy członkami zespołu, integrujące się m.in. z repozytorium kodu. Technologia ta została narzucona, ze względu na sposób organizacji pracy w \textit{Software Development Studio} na Politechnice Poznańskiej.

\subsection{JasperReports}
\label{Chapter634}

Ze względu na wymagania pozafunkcjonalne, zdecydowano się skorzystać z mechanizmów raportowania oferowanych przez \emph{JasperReports}.