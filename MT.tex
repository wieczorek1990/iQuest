\documentclass[12pt]{article}

% kodowanie: latin2, utf8 lub cp1250
%\usepackage[latin2]{inputenc}
\usepackage[polish]{babel}
\usepackage[utf8]{inputenc}
\usepackage[T1]{fontenc}
\usepackage[MeX]{polski}
\begin{document}


\section{Implementacja}
\subsection{Użyte technologie}
\subsubsection{JavaScript}
Formularze wymagające częstej interakcji z klientem, np. formularz umożliwiający tworzenie nowej ankiety, oraz funkcje związane z walidacją pól uzupełnianych przez klienta zostały napisane w \emph{JavaScript}. Obsługa strony po stronie użytkownika zapobiega frustracji, związanej z częstym przeładowywaniem całej strony.

\section{Problemy i ich rozwiązania}
\subsection{Inicjalizacja bazy danych}
Moduł iQuest do działania wymaga rozszerzenie istniejącej bazy danych platformy \emph{Moodle} o dodatkowe tabele, przechowujące niezbędne dane wykorzystywane do spełnienia założonej funkcjonalności.

Do zaimportowania bazy danych przygotowanej przez architekta wykorzystano narzędzie, wbudowane w platformę \emph{Moodle}, \emph{XMLDB}, które gwarantuje bezobsługową instalację modułu w przyszłości. Jak się później okazało narzędzie to posiada błąd, który uniemożliwia zaimportowania kluczy obcych. Wymagało to od programistów ręcznego utworzenia wszystkich kluczy obcych, przewidzianych przez architekta.

\subsection{Instalacja modułu}
Postanowiono, że wraz z instalacją modułu powinień automatycznie tworzyć się odpowiedni \emph{kurs} związany jedynie z modułem \emph{iQuest}. Zmniejsza to potrzebny czas na przygotowanie platformy do użytku, oraz zapobiega pomyłkom związanych z ręcznym tworzeniem i konfiguracji \emph{kursu}. 

Niestety okazało się, że podejście to uniemożliwia instalację modułu jednocześnie z całą platformą \emph{Moodle}, ponieważ dodatkowe moduły instalowane są przed mechanizmami pozwalającymi na tworzenie \emph{kursu}. Doinstalowanie modułu do zainstalowanej platformy nie powoduje żadnych komplikacji.
\end{document}